\clearpage
\setcounter{page}{1}
\maketitlesupplementary

\section{Supplementary Materials}
\label{sec:supplementary}

\subsection{Detailed Search Strategy}
\label{sec:search_strategy}

\subsubsection{Search Documentation}


A number of search strings were constructed using relevant terms deduced from the research. The
initial search strings are shown in the Appendix in Table \ref{tab:search_terms}. These terms are
combined in Table \ref{tab:search_documentation}, using boolean operators to generate query
strings, which are then used to search each individual database.

% \begin{table*}[htbp!]
%     \centering
%     \caption{Core Search Terms for Distributed Deep Learning and GPU Programming}
%     \label{tab:search_terms}
%     \begin{tabularx}{\textwidth}{|l|X|X|}
%         \hline
%         \textbf{Category} & \textbf{Distributed Learning} & \textbf{GPU Computing} \\
%         \hline
%         Core Terms & 
%         \textbf{"Distributed Deep Learning"},
%         \textbf{"Parallel Deep Learning"},
%         "Large-Scale Deep Learning" &
%         \textbf{"GPU Programming"},
%         \textbf{"CUDA Programming"} \\
%         \hline
%         Technical Approach & 333
%         \textbf{"Data Parallelism"},
%         \textbf{"Model Parallelism"},
%         \textbf{"Hybrid Parallelism"} &
%         \textbf{"CUDA"},
%         "GPU Optimization",
%         "Parallel Computing" \\
%         \hline
%         Implementation & 
%         \textbf{"Parameter Server"},
%         \textbf{"All-Reduce"},
%         \textbf{"SGD"} &
%         \textbf{"CUDA Toolkit"},
%         \textbf{"cuDNN"},
%         "Multi-GPU" \\
%         \hline
%         Frameworks & 
%         "TensorFlow",
%         "PyTorch", 
%         "Horovod" &
%         "TensorRT",
%         "PyCUDA",
%         "Numba" \\
%         \hline
%     \end{tabularx}
%     \caption*{Note: Bold terms indicate primary search terms that will be prioritized.}
% \end{table*}

The starting date used for the analysis starts from 2014 till 2024. The reason for utilizing this range
is due to the publishing date of \cite{SierraCanto2010ParallelTO}, which acts as a key paper in the domain. As a result,
I'd like to analyze the advancement of the field starting from this age. The key The following table 
documents our complete search process:

% TODO Razvan: remove this table
% \begin{table*}[htbp!]
%     \centering
%     \caption{Detailed Search Documentation}
%     \label{tab:search_documentation}
%     \begin{tabularx}{\textwidth}{|l|X|c|c|c|}
%         \hline
%         \textbf{Database} & \textbf{Search String} & \textbf{Years Covered} & \textbf{Results} & \textbf{Filtered} \\
%         \hline
%         Scopus & ("Distributed Deep Learning" OR "Parallel Deep Learning") AND "Data Parallelism" & 2017-2024 & 82 & 11 \\
%                & ( "machine learning" OR "deep learning" ) AND ( "Data parallelism" OR "model parallelism" OR "pipeline parallelism" OR "hybrid parallelism" ) AND ( "framework" OR "implementation" ) & 2012-2024 & 206 & 11 \\
%         \hline
%         ACM Digital Library & ("GPU Programming" OR "GPGPU Programming") AND ("CUDA" OR "CUDA Programming") AND ("Parallel Computing") & 2015-2022 & 2024-05-10 & 210 \\
%         \hline
%         Science Direct & [Exact search string] & 2015-2022 & 2024-05-10 & XXX \\
%         \hline
%         arXiv & [Exact search string] & 2015-2022 & 2024-05-10 & XXX \\
%         \hline
%     \end{tabularx}
% \end{table*}


\subsubsection{Search Process Details}
\begin{itemize}
    \item \textbf{Years Covered:} 2015-2022
    \item \textbf{Language Restrictions:} English only
    \item \textbf{Document Types:} Journal articles, conference papers, and high-quality preprints
\end{itemize}

\subsubsection{Manual Searches}
The following additional sources were manually searched:
\begin{itemize}
    \item Key conference proceedings (e.g., NeurIPS, ICLR, ICML)
    \item Reference lists of included studies (snowballing)
    \item Citations of included studies (forward snowballing)
\end{itemize}

\subsection{Study Selection Details}
\label{sec:study_selection}

% TODO Razvan: remove this table: excluded studies are not needed
% \begin{table*}[htbp!]
%     \centering
%     \caption{List of Excluded Studies with Reasons}
%     \label{tab:excluded_studies}
%     \begin{tabularx}{\textwidth}{|l|l|X|c|X|}
%         \hline
%         \textbf{ID} & \textbf{Authors} & \textbf{Title} & \textbf{Year} & \textbf{Reason for Exclusion} \\
%         \hline
%         E1 & Author et al. & Title of excluded study 1 & 20XX & Does not meet inclusion criterion 1 \\
%         \hline
%         E2 & Author et al. & Title of excluded study 2 & 20XX & Insufficient technical details \\
%         \hline
%     \end{tabularx}
% \end{table*}


\subsection{Data Extraction Forms}
\label{sec:data_extraction}

The following template was used for data extraction:
\begin{itemize}
    \item \textbf{Study ID:} [Unique identifier]
    \item \textbf{Authors:} [Author names]
    \item \textbf{Year:} [Publication year]
    \item \textbf{Venue:} [Publication venue]
    \item \textbf{Distribution Strategy:} [Description]
    \item \textbf{Implementation Details:} [Technical details]
    \item \textbf{Evaluation Metrics:} [Performance measures]
    \item \textbf{Results:} [Key findings]
\end{itemize}

\subsection{Quality Assessment Checklist}
\label{sec:quality_checklist}

Each study was evaluated using the following criteria:
\begin{enumerate}
    \item \textbf{Problem Definition} (0-2 points)
        \begin{itemize}
            \item 2: Clear and well-motivated problem statement
            \item 1: Partially clear problem statement
            \item 0: Unclear problem statement
        \end{itemize}
    \item \textbf{Methodology Description} (0-2 points)
        \begin{itemize}
            \item 2: Detailed and replicable methodology
            \item 1: Partial methodology description
            \item 0: Insufficient methodology description
        \end{itemize}
    % Add more quality criteria as needed
\end{enumerate}

\subsection{Raw Data}
\label{sec:raw_data}

\subsubsection{Performance Metrics}
[Tables or figures showing raw performance data]

\subsubsection{Statistical Analysis}
[Detailed statistical analysis of the results]

\section{Conflicts of Interest}
\label{sec:conflicts}

The authors declare no conflicts of interest that could have appeared to influence the work reported in this paper. This research did not receive any specific grant from funding agencies in the public, commercial, or not-for-profit sectors.

% NOTE not needed
% \section{Author Contributions}
% \label{sec:contributions}

% \begin{itemize}
%     \item \textbf{First Author:} Conceptualization, Methodology, Writing - Original draft
%     \item \textbf{Second Author:} Data curation, Formal analysis, Writing - Review \& editing
% \end{itemize}

% All authors have read and agreed to the published version of the manuscript.


\subsection{Review Protocol}
This systematic review follows the guidelines proposed by Kitchenham and Charters for software engineering research. The protocol was developed and reviewed by all authors before beginning the review process.

\subsection{Data Sources and Search Strategy}
We searched the following digital libraries:
\begin{itemize}
    \item IEEE Xplore
    \item ACM Digital Library
    \item Science Direct
    \item arXiv (for preprints)
\end{itemize}


% TODO Razvan: remove this table
% \begin{table*}[htbp]
%     \centering
%     \caption{Search Terms by Category for Distributed Deep Learning and GPU Programming}
%     \label{tab:search_terms}
%     \begin{tabularx}{\textwidth}{|l|X|X|}
%         \hline
%         \textbf{Facet} & \textbf{Distributed Deep Learning Terms} & \textbf{GPU Programming Terms} \\
%         \hline
%         Core Concept & 
%         \textbf{"Distributed Deep Learning"}, \textbf{"Parallel Deep Learning"}, 
%         "Deep Learning on Clusters", "Large-Scale Deep Learning", 
%         "Scalable Deep Learning" &
%         \textbf{"GPU Programming"}, \textbf{"General-Purpose GPU Programming"}, 
%         \textbf{"GPGPU Programming"} \\
%         \hline
%         Specific Technology / 
%         Parallelization Techniques & 
%         \textbf{"Data Parallelism"}, \textbf{"Model Parallelism"}, 
%         \textbf{"Hybrid Parallelism"}, "Data-Parallel", "Model-Parallel" &
%         \textbf{"CUDA"}, \textbf{"CUDA Programming"}, "Nvidia CUDA", 
%         "Compute Unified Device Architecture" \\
%         \hline
%         Training Methods / 
%         Programming Aspects & 
%         \textbf{"Stochastic Gradient Descent"}, \textbf{"SGD"}, "Mini-batch SGD", 
%         "Asynchronous SGD", "Synchronous SGD", "Distributed Stochastic Gradient Descent", 
%         "Elastic Averaging SGD", "Byzantine-tolerant gradient descent" &
%         "Parallel Computing", "Parallel Programming", "High-Performance Computing", 
%         "Kernel Programming", "GPU Memory Management", "GPU Optimisation", 
%         "CUDA Libraries" \\
%         \hline
%         Communication Strategies & 
%         \textbf{"Parameter Server"}, \textbf{"All-Reduce"}, 
%         \textbf{"Collective Communication"}, "Decentralized Optimization", 
%         "Decentralized Parameter Sharing", "Gradient Compression", 
%         "Sparse Communication" & -- \\
%         \hline
%         Frameworks & 
%         "TensorFlow", "PyTorch", "Horovod", "DistBelief", "Parameter Server", 
%         "SparkNet", "Petuum", "BigDL", "MXNet", "CaffeOnSpark" &
%         \textbf{"CUDA Toolkit"}, \textbf{"cuDNN"}, "TensorRT", "Thrust", 
%         "OpenACC", "RAPIDS", "PyCUDA", "Numba", "JAX", "TensorFlow with CUDA", 
%         "PyTorch with CUDA", "Caffe with CUDA", "Theano with CUDA", 
%         "MxNet with CUDA", "Darknet with CUDA" \\
%         \hline
%         Hardware & 
%         "GPUs", "CPUs", "Accelerators", "Cluster Computing", "Supercomputers", 
%         "Multi-GPU" &
%         "GPUs", "Nvidia GPUs", "Multi-GPU", "CUDA-enabled GPUs" \\
%         \hline
%         Performance Aspects & 
%         "Scalability", "Convergence", "Latency", "Communication Overhead", 
%         "Fault Tolerance" &
%         "GPU Acceleration", "Parallel Speedup", "Throughput", 
%         "Memory Bandwidth", "Latency", "Performance Optimisation" \\
%         \hline
%     \end{tabularx}
%     \caption*{Note: Bold terms indicate primary search terms that will be prioritized in the search strategy. A dash (--) indicates no specific terms for that category.}
% \end{table*}


\clearpage
\onecolumn

{\tiny
\begin{longtable}{|l|p{0.6cm}|p{11.8cm}|p{0.6cm}|p{2cm}|}
	\caption{The passages on distributed neural networks}\label{tab:dnn_passages}                                                                                                                                                                                                                                                                                                                                                                                                                                                              \\

	\toprule
	Cat. & \centering ID & Text Passages                                                                                                                                                                                                                                                                                                                                                                                                                                                                                              & Ref. & Codes \\
	\midrule
	\endfirsthead

	\multicolumn{5}{c}{Table \thetable{} -- continued from previous page}                                                                                                                                                                                                                                                                                                                                                                                                                                                                  \\
	\toprule
	Cat. & \centering ID & Text Passages                                                                                                                                                                                                                                                                                                                                                                                                                                                                                              & Ref. & Codes \\
	\midrule
	\endhead
    \hline
	\multirow{44}{*}{\rotatebox[origin=c]{90}{RQ\textsubscript{1}: Key Motivating Factors}}
	     & D101\label{D101}\newline\centering\cite{abadi_tensorflow_2016} & In addition, often in close collaboration with the Google Brain team, more than 50 teams at Google and other Alphabet companies have deployed deep neural networks using DistBelief in a wide variety of products, including Google Search [11], our advertising products, our speech recognition systems [50, 6, 46], Google Photos [43], Google Maps and StreetView [19], Google Translate [18], YouTube, and many others.
	     & \cite{abadi_tensorflow_2016,li_pytorch_2020}
	     & \textbullet\ Internal need to scale existing products \\

	\cline{2-5}
	     & \label{D102}D102\newline\centering\cite{chen_mxnet_2015} & The scale and complexity of machine learning (ML) algorithms are becoming increasingly large. Almost all recent ImageNet challenge [12] winners employ neural networks with very deep layers, requiring billions of floating-point operations to process one single sample. The rise of structural and computational complexity poses interesting challenges to ML system design and implementation.
	     & \cite{chen_mxnet_2015,lepikhin_gshard_2020,shoeybi_megatron-lm_2020}
	     & \textbullet\ Increasingly complex models/datasets\newline \textbullet\ Keen interest for scientific inquiry \\
	\cline{2-5}
	     & \label{D103}D103\newline\centering\cite{huang_gpipe_2019} & We scale the architecture along two dimensions to stress the flexibility of GPipe: (i) along the depth by increasing the number of layers in the model and (ii) along the width by increasing the hidden dimension in the feed-forward layers and the number of attention heads (...) We notice that increasing the model capacity, from 400M params (T (6, 8192, 16)) to 1.3B (T (24, 8192, 16)), and further, to 6B (T (64, 16384, 32)), leads to significant quality improvements across all languages.
	     & \cite{huang_gpipe_2019,lepikhin_gshard_2020}
	     & \textbullet\ Improve performance \newline \textbullet\ Critical in many domains \\
	\cline{2-5}
	     & \label{D104}D104\newline\centering\cite{jiang_unified_nodate} & Data center clusters that run DNN training jobs are inherently heterogeneous. They have GPUs and CPUs for computation and network bandwidth for distributed training. However, existing distributed DNN training architectures, all-reduce and Parameter Server (PS), cannot fully utilize such heterogeneous resources. In this paper, we present a new distributed DNN training architecture called BytePS. BytePS can leverage spare CPU and bandwidth resources in the cluster to accelerate distributed DNN training tasks running on GPUs.
	     & \cite{jiang_unified_nodate}
	     & \textbullet\ Utilization of heterogenous hardware \\
	\cline{2-5}
	     & \label{D105}D105\newline\centering\cite{lepikhin_gshard_2020} & Neural network scaling has been critical for improving the model quality in many real-world machine learning applications with vast amounts of training data and compute. Although this trend of scaling is affirmed to be a sure-fire approach for better model quality, there are challenges on the path such as the computation cost, ease of programming, and efficient implementation on parallel devices. 
	     & \cite{lepikhin_gshard_2020,chen_mxnet_2015, huang_gpipe_2019, shoeybi_megatron-lm_2020}
	     & \textbullet\ Increasingly complex models/datasets \newline \textbullet\ Improve performance \newline \textbullet\ Critical in many domains \\	\cline{2-5}
	     & \label{D106}D106\newline\centering\cite{li_pytorch_2020} & Deep Neural Networks (DNN) have powered a wide spectrum of applications, ranging from image recognition [20], language translation [15], anomaly detection [16], content recommendation [38], to drug discovery [33], art generation [28], game play [18], and self-driving cars [13]. Many applications pursue higher intelligence by optimizing larger models using larger datasets, craving advances in distributed training systems. Among existing solutions, distributed data parallel is a dominant strategy due to its minimally intrusive nature. (...) During the past year, we have seen significant adoption both internally and externally. Within Facebook, a workload study from 05/11/20 to 06/05/20 shows that more than 60\% of production GPU hours during that period were spent on the PyTorch distributed data parallel package across a wide variety of applications, including speech, vision, mobile vision, translation, etc.
	     & \cite{li_pytorch_2020,abadi_tensorflow_2016}
	     & \textbullet\ Meeting user's requirements \newline \textbullet\ Emerging applications \newline \textbullet\ Internal need to scale existing products \\	\cline{2-5}
	     & \label{D107}D107\newline\centering\cite{li_colossal-ai_2023} & Methods such as PipeDream [25], GPipe [16], and Chimera [20] were proposed to split the model into several chunks of consecutive layers and each chunk is allocated to a device as shown in Figure 3c. Intermediate activations and gradients are passed between pipeline stages to complete the forward and backward pass. As a result, our method reduces cross-node communication. Pipeline parallelism allows multiple devices to compute simultaneously, leading to a higher throughput. (...) Inspired by Alpa, Colossal-AI has included an experimental automatic parallelism feature to improve upon the Alpa project.
	     & \cite{li_colossal-ai_2023}
	     & \textbullet\ Improving/Building on existing frameworks \\ \cline{2-5}
         & D108\newline\centering\cite{moritz_ray_2018} & In our evaluation, we study the following questions: (...) 2. What overheads are imposed on distributed primitives (e.g., allreduce) written using Ray's API? (Section 5.1) 3. In the context of RL workloads, how does Ray compare against specialized systems for training, serving, and simulation? (Section 5.2) 4. What advantages does Ray provide for RL applications, compared to custom systems? (Section 5.3)
	     & \cite{moritz_ray_2018}
	     & \textbullet\ Extending existing tools to new domains i.e. Reinforcement Learning \\ \cline{2-5}
         & D109\newline\centering\cite{rasley_deepspeed_2020} & DeepSpeed is compatible with PyTorch. One piece of our library, called ZeRO, is a new parallelized optimizer that greatly reduces the resources needed for model and data parallelism while massively increasing the number of parameters that can be trained. Researchers have used these breakthroughs to create Turing Natural Language Generation (Turing-NLG), which at the time of its release was the largest publicly known language model at 17 billion parameters.
	     & \cite{rasley_deepspeed_2020,sergeev_horovod_2018}
	     & \textbullet\ Cross-framework compatibility \newline \textbullet\ Large-scale training \\ \cline{2-5}
         & \label{D110}D110\newline\centering\cite{sergeev_horovod_2018} & Existing methods for enabling multi-GPU training under the TensorFlow library entail non-negligible communication overhead and require users to heavily modify their model-building code, leading many researchers to avoid the whole mess and stick with slower single-GPU training. In this paper we introduce Horovod, an open source library that improves on both obstructions to scaling: it employs efficient inter-GPU communication via ring reduction and requires only a few lines of modification to user code, enabling faster, easier distributed training in TensorFlow.
	     & \cite{sergeev_horovod_2018,rasley_deepspeed_2020,wolf_huggingfaces_2020}
	     & \textbullet\ Cross-framework compatibility \newline \textbullet\ Large-scale training\newline \textbullet\ Ease of use \\ \cline{2-5}
         & \label{D111}D111\newline\centering\cite{shoeybi_megatron-lm_2020} & Natural Language Processing (NLP) is advancing quickly in part due to an increase in available compute and dataset size. The abundance of compute and data enables training increasingly larger language models via unsupervised pretraining... Empirical evidence indicates that larger language models are dramatically more useful for NLP tasks such as article completion, ques-tion answering, and natural language inference (...) In summary, our approach as de-scribed above is simple to implement, requiring only a few extra all-reduce operations added to the forward and back-ward pass. It does not require a compiler, and is orthogonal and complementary to the pipeline model parallelism advo-cated by approaches such as (Huang et al., 2018).
	     & \cite{shoeybi_megatron-lm_2020,chen_mxnet_2015,lepikhin_gshard_2020}
	     & \textbullet\ Increasingly complex models/datasets \newline \textbullet\ No need for compilers \\ \cline{2-5}
         & \label{D112}D112\newline\centering\cite{wolf_huggingfaces_2020} & An increasingly important goal of Transformers is to make it easy to efficiently deploy model to pro-duction. Different users have different production needs, and deployment often requires solving significantly different challenges than training. The library thereforce allows for several different strategies for production deployment.
         One core propery of the libary is that models are available both in PyTorch and TensorFlow, and there is interoperability between both frameworks. 
	     & \cite{wolf_huggingfaces_2020,sergeev_horovod_2018,rasley_deepspeed_2020}
	     & \textbullet\ Ease of use \newline \textbullet\ Cross-framework compatibility \\ \hline

    \multirow{33}{*}{\rotatebox[origin=c]{90}{RQ\textsubscript{3}: Critical Factors}}
         & \label{D201}D201\newline\centering\cite{abadi_tensorflow_2016} & A computation expressed using TensorFlow can be executed with little or no change on a wide variety of heterogeneous systems, ranging from mobile devices such as phones and tablets up to large-scale distributed systems of hundreds of machines and thousands of computational devices such as GPU cards.
         & \cite{abadi_tensorflow_2016,jiang_unified_nodate}
	     & \textbullet\ Utilization of heterogenous hardware \\ \cline{2-5}
         & \label{D202}D202\newline\centering\cite{chen_mxnet_2015} & Most ML systems embed a domain-specific language (DSL) into a host language (e.g. Python, Lua, C++). Possible programming paradigms range from imperative, where the user specifies exactly "how" computation needs to be performed, and declarative, where the user specification focuses on "what" to be done.
         & \cite{chen_mxnet_2015,lepikhin_gshard_2020}
	     & \textbullet\ Programming paradigms \\ \cline{2-5}
        
         & \label{D203}D203\newline\centering\cite{huang_gpipe_2019} & In many cases, increasing model capacity beyond the memory limit of a single accelerator has required developing special algorithms or infrastructure. These solutions are often architecture-specific and do not transfer to other tasks. To address the need for efficient and task-independent model parallelism, we introduce GPipe, a pipeline parallelism library that allows scaling any network that can be expressed as a sequence of layers.
         & \cite{huang_gpipe_2019,rasley_deepspeed_2020,shoeybi_megatron-lm_2020}
	     & \textbullet\ Scaling \\ \cline{2-5}

         & \label{D204}D204\newline\centering\cite{jiang_unified_nodate} & BytePS is a unified distributed DNN training acceleration system that achieves optimal communication efficiency in heterogeneous GPU/CPU clusters.
         & \cite{jiang_unified_nodate,li_colossal-ai_2023,sergeev_horovod_2018}
	     & \textbullet\ Communication efficiency \\ \cline{2-5}

         & \label{D205}D205\newline\centering\cite{lepikhin_gshard_2020} & In automatic sharding model description should be separated from the partitioning implementation and optimization. This separation of concerns let model developers focus on the network architecture and flexibly change the partitioning strategy, while the underlying system applies semantic-preserving transformations and implements efficient parallel execution. 
         & \cite{lepikhin_gshard_2020,chen_mxnet_2015}
	     & \textbullet\ Separation of concerns \newline \textbullet\ Programming ease \\ \cline{2-5}

         & \label{D206}D206\newline\centering\cite{li_pytorch_2020} & PyTorch natively provides several techniques to accelerate distributed data parallel, including bucketing gradients, overlapping compu-tation with communication, and skipping gradient synchronization.
         & \cite{li_pytorch_2020,li_colossal-ai_2023,rasley_deepspeed_2020,shoeybi_megatron-lm_2020}
	     & \textbullet\ Performance \\ \cline{2-5}

         & \label{D207}D207\newline\centering\cite{li_colossal-ai_2023} & Methods such as PipeDream [25], GPipe [16], and Chimera [20] were proposed to split the model into several chunks of consecutive layers and each chunk is allocated to a device as shown in Figure 3c. Intermediate activations and gradients are passed between pipeline stages to complete the forward and backward pass. As a result, this method reduces cross-node communication. Pipeline parallelism allows multiple devices to compute simultaneously, leading to a higher throughput.
         & \cite{li_pytorch_2020,li_colossal-ai_2023,rasley_deepspeed_2020,shoeybi_megatron-lm_2020}
	     & \textbullet\ Performance \\ \cline{2-5}

         & \label{D208}D208\newline\centering\cite{moritz_ray_2018} & To learn a policy, an agent typically employs a two-step process: (1) policy evaluation and (2) policy improvement. To evaluate the policy, the agent interacts with the environment (e.g., with a simulation of the environment) to generate trajectories, where a trajectory consists of a sequence of (state, reward) tuples produced by the current policy.
         & \cite{moritz_ray_2018}
	     & \textbullet\ Policy learning in reinforcement learning \\ \cline{2-5}

         & \label{D209}D209\newline\centering\cite{rasley_deepspeed_2020} & The latest trend in AI is that larger natural language models provide better accuracy; however, larger models are difficult to train because of cost, time, and ease of code integration. With the goal of advancing large model training by improving scale, speed, cost, and usability for model developers across the world, Since then, the DeepSpeed team has been hard at work extending the library to continue pushing the boundaries of scale and speed of deep learning training.
         & \cite{rasley_deepspeed_2020,huang_gpipe_2019,shoeybi_megatron-lm_2020}
	     & \textbullet\ Performance \newline \textbullet\ Cost \newline \textbullet\ Scalability \newline \textbullet\ Usability \\ \cline{2-5}

         & \label{D210}D210\newline\centering\cite{sergeev_horovod_2018} & In early 2017 Baidu published an article [8] evangelizing a different algorithm for averaging gradients and communicating those gradients to all nodes (Steps 2 and 3 above), called ring-allreduce (...) allows worker nodes to average gradients and disperse them to all nodes without the need for a parameter server (...)  This algorithm is bandwidth-optimal, meaning that if the buffer is large enough, it will optimally utilize the available network.
         & \cite{sergeev_horovod_2018,li_colossal-ai_2023}
	     & \textbullet\ Network Latency \\ \cline{2-5}

         & \label{D211}D211\newline\centering\cite{shoeybi_megatron-lm_2020} & Our approach is to utilize model parallelism to split the model across multiple accelerators. This not only alleviates the memory pressure, but also increases the amount of parallelism independently of the microbatch size. (...) By increasing the minibatch size proportionally to the number of available workers (i.e. weak scaling), one observes near linear scaling in training data throughput. (...) We exploit the inherent structure in transformer based language models to make a simple model-parallel implementation that trains efficiently in PyTorch, with no custom C++ code or compiler required.
         & \cite{shoeybi_megatron-lm_2020,rasley_deepspeed_2020}
	     & \textbullet\ Performance \newline \textbullet\ Cross-framework compatibility \newline \textbullet\ Scalability \newline \textbullet\ Usability \\ \cline{2-5}
         
         & \label{D212}D212\newline\centering\cite{wolf_huggingfaces_2020} & Each model is made up of a Tokenizer, Transformer, and Head. The model is pretrained with a fixed head and can then be further fine-tuned with alternate heads for different tasks.
         & \cite{wolf_huggingfaces_2020,rasley_deepspeed_2020,shoeybi_megatron-lm_2020}
	     & \textbullet\ Ease of use \\ \hline


    \multirow{20}{*}{\rotatebox[origin=c]{90}{RQ\textsubscript{2}: Evaluation Metrics}}
         & D301\newline\centering\cite{abadi_tensorflow_2016} & In addition, often in close collaboration with the Google Brain team, more than 50 teams at Google and other Alphabet companies have deployed deep neural networks using DistBelief in a wide variety of products, including Google Search [11], our advertising products, our speech recognition systems [50, 6, 46], Google Photos [43], Google Maps and StreetView [19], Google Translate [18], YouTube, and many others.
         & \cite{abadi_tensorflow_2016}
	     & \textbullet\ Deployment via Google Apps \\ \cline{2-5}
         
         & D303\newline\centering\cite{huang_gpipe_2019} & We demonstrate the advantages of GPipe by training large-scale neural networks on two different tasks with distinct network architectures: (i) Image Classification: We train a 557-million-parameter AmoebaNet model and attain a top-1 accuracy of 84.4\% on ImageNet-2012, (ii) Multilingual Neural Machine Translation: We train a single 6-billion-parameter, 128-layer Transformer model on a corpus spanning over 100 languages and achieve better quality than all bilingual models.
         & \cite{huang_gpipe_2019,lepikhin_gshard_2020}
	     & \textbullet\ Image Classification\newline\textbullet\ Multilingual Neural Machine Translation \\ \cline{2-5}

         & D304\newline\centering\cite{jiang_unified_nodate} & We evaluate BytePS using six DNN models and three training frameworks (TensorFlow, PyTorch, MXNet) in production data centers. The results show that with 256 GPUs, BytePS consistently outperform existing all-reduce and PS solutions by up to 84\% and 245\%, respectively.
         & \cite{jiang_unified_nodate}
	     & \textbullet\ Cross-platform evaluation (TensorFlow, PyTorch, MXNet) \\ \cline{2-5}

         & D305\newline\centering\cite{lepikhin_gshard_2020} & GShard enabled us to scale up multilingual neural machine translation Transformer model with Sparsely-Gated Mixture-of-Experts beyond 600 billion parameters using automatic sharding. We demonstrate that such a giant model can efficienctly be trained on 2048 TPU v3 accelerators in 4 days to achieve far superior quality for translation from 100 languages to English compared to the prior art.
         & \cite{lepikhin_gshard_2020,huang_gpipe_2019}
	     & \textbullet\ Models: MoEs \\ \cline{2-5}

         & D306\newline\centering\cite{li_pytorch_2020} & We measure DDP per iteration latency and scalability using two popular models, ResNet50 [20] and BERT [15], to represent typical vision and NLP applications.
         & \cite{li_pytorch_2020,li_colossal-ai_2023}
	     & \textbullet\ Tasks: NLP, Vision \newline \textbullet\ Performance \\ \cline{2-5}
         
         & D307\newline\centering\cite{li_colossal-ai_2023} & To demonstrate the capability of dynamic tensor placement in ColossalAI, we trained GPT-2 model with 10 billion parameters on the Wikipedia dataset on System II. We set the batch size to 4 and scaled the data parallel training from 1 GPU to 8 GPU.
         & \cite{li_colossal-ai_2023,li_pytorch_2020,moritz_ray_2018,shoeybi_megatron-lm_2020}
	     & \textbullet\ Performance \\ \cline{2-5}
         
         & D308\newline\centering\cite{moritz_ray_2018} & In our experiments, we demo-strate scaling beyond 1.8 million tasks per second and better performance than existing specialized systems for several challenging reinforcement learning applications.
         & \cite{moritz_ray_2018,li_colossal-ai_2023,li_pytorch_2020,shoeybi_megatron-lm_2020}
	     & \textbullet\ Performance \newline \textbullet\ Tasks: RL \\ \cline{2-5}
         
         & D311\newline\centering\cite{shoeybi_megatron-lm_2020} & Using the GPT-2 model we achieve SOTA results on the WikiText103 (10.8 compared to SOTA perplexity of 15.8) and LAMBADA (66.5\% compared to SOTA accuracy of 63.2\%) datasets. Our BERT model achieves SOTA results on the RACE dataset (90.9\% compared to SOTA accuracy of 89.4\%) (...) We demonstrate that scaling the model size results in improved accuracies for both GPT-2 (studied up to 8.3 billion parameters) and BERT (studied up to 3.9B parameters) models.
         & \cite{shoeybi_megatron-lm_2020,li_pytorch_2020,li_colossal-ai_2023,moritz_ray_2018}
	     & \textbullet\ Tasks: NLP \newline \textbullet\ Performance \newline \textbullet\ Scaling \\ \hline
         
    \multirow{30}{*}{\rotatebox[origin=c]{90}{RQ\textsubscript{2}: Tool limitations and challenges}} 
         & D401\newline\centering\cite{abadi_tensorflow_2016} & 
		 Once a system has multiple devices, there are two main complications: deciding which device to place the computation for each node in the graph, and then managing the required communication of data across device boundaries implied by these placement decisions. (...) A future version of this white paper will have a comprehensive performance evaluation section of both the single machine and distributed implementations.
         & \cite{abadi_tensorflow_2016,li_colossal-ai_2023,sergeev_horovod_2018}
	     & \textbullet\ Communication overhead \\ \cline{2-5}
         
         & D402\newline\centering\cite{chen_mxnet_2015} & 
		 Most ML systems embed a domain-specific language (DSL) into a host language (e.g. Python, Lua, C++).  (...)  Comparing to other open-source ML systems, MXNet provides a superset programming interface to Torch7, Theano, Chainer and Caffe, and supports more systems such as GPU clusters.
         & \cite{chen_mxnet_2015,huang_gpipe_2019}
	     & \textbullet\ Ease of use \newline \textbullet\ Common API with other frameworks \\ \cline{2-5}
         
         & D403\newline\centering\cite{huang_gpipe_2019} & 
		 (Other) naive model parallelism strategies lead to severe under-utilization due to the sequential dependency of the network. (...) In many cases, increasing model capacity beyond the memory limit of a single accelerator has required developing special algorithms or infrastructure. These solutions are often architecture-specific and do not transfer to other tasks.
         & \cite{huang_gpipe_2019,lepikhin_gshard_2020,jiang_unified_nodate,chen_mxnet_2015}
	     & \textbullet\ Resource under-utilization \newline \textbullet\ No shared API with common frameworks \\ \cline{2-5}
         
         & D404\newline\centering\cite{jiang_unified_nodate} & 
		 For distributed training, there are two families of data parallelism approaches, i.e., all-reduce and Parameter Server (PS). In all-reduce, no additional CPU machine is involved [to aggregate results from different accelerators] . Ring is the most popular all-reduce algorithm. (...) All-reduce has no way to utilize additional non-worker nodes, since it was designed for homogeneous setup.
         & \cite{jiang_unified_nodate,huang_gpipe_2019,lepikhin_gshard_2020}
	     & \textbullet\ Designed for homogeneous setup \newline \textbullet\ Resource under-utilization \\ \cline{2-5}
         
         & D405\newline\centering\cite{lepikhin_gshard_2020} & 
		 There is a lack of support for efficient model parallelism algorithms under commonly used deep learning frameworks such as TensorFlow [21] and PyTorch [22]. Naive model parallelism with graph partition is supported but it would lead to severe under-utilization due to the sequential dependency of the network and gradient based optimization.
         & \cite{lepikhin_gshard_2020,huang_gpipe_2019,jiang_unified_nodate}
	     & \textbullet\ Resource under-utilization \newline \textbullet\ No efficient model parallelism algorithms \\ \cline{2-5}
         
         & D406\newline\centering\cite{li_pytorch_2020} & 
		 Despite the conceptual simplicity of the technique, the subtle dependencies between computation and communication make it non-trivial to optimize the distributed training efficiency. (...) Based on our observations, there is no single configuration that would work for all use cases, as it would highly depend on the model size, model structure, network link bandwidth, etc.
         & \cite{li_pytorch_2020,abadi_tensorflow_2016,sergeev_horovod_2018,shoeybi_megatron-lm_2020}
	     & \textbullet\ No optimal algorithm for all use cases \newline \textbullet\ Optimization challenges \\ \cline{2-5}
         
         & D407\newline\centering\cite{li_colossal-ai_2023} & 
		 One drawback of pipeline parallel training is that there will be some bubble time, where some devices are idle when others are engaged in computation, leading to the waste of computational resources. (...) DeepSpeed's static policy will still offload all model data to the CPU memory, leading to low memory efficiency and high communication over-head. 
         & \cite{li_colossal-ai_2023,abadi_tensorflow_2016,sergeev_horovod_2018}
	     & \textbullet\ Redundant computation \newline \textbullet\ Communication overhead \\ \cline{2-5}
        
         & D408\newline\centering\cite{moritz_ray_2018} & 
		 While in principle one could develop an end-to-end solution by stitching together several existing systems (e.g., Horovod [53] for distributed training, Clipper [19] for serving, and CIEL [40] for simulation), in practice this approach is untenable due to the tight coupling of these components within applications. As a result, researchers and practitioners today build one-off systems for specialized RL applications [58, 41, 54, 44, 49, 5]. (...) To satisfy these requirements, Ray implements a unified interface that can express both task-parallel and actorbased computations.
         & \cite{moritz_ray_2018,chen_mxnet_2015}
	     & \textbullet\ Tight coupling of components \newline \textbullet\ Multiple programming paradigms \\ \cline{2-5}
         
         & D410\newline\centering\cite{sergeev_horovod_2018} & 
		 There are a few areas that we are actively working on to improve Horovod, including: Collecting and sharing learnings about adjusting model parameters for distributed deep learning: Facebook's paper [6] describes the adjustments needed to model hyperparameters to achieve the same or greater accuracy in a distributed training job compared to training the same model on a single GPU, demonstrating the feasibility of training a TensorFlow model on 256 GPUs. We believe this area of deep learning research is still in its early stages and hope to collaborate with other teams about approaches to further scale deep learning training.
         & \cite{sergeev_horovod_2018,abadi_tensorflow_2016,li_colossal-ai_2023}
	     & \textbullet\ Cross-node communication challenges \newline \textbullet\ Collaboration with external teams \\ \cline{2-5}
         
         & D411\newline\centering\cite{shoeybi_megatron-lm_2020} & 
		 However, large batch training introduces complications into the optimization process that can result in reduced accuracy or longer time to convergence, offsetting the benefit of increased training throughput. (...) for BERT models, careful attention to the placement of layer normalization in BERT-like models is critical to achieving increased accuracies as the model size increases.
         & \cite{shoeybi_megatron-lm_2020,li_pytorch_2020}
	     & \textbullet\ Error prone utilization \newline \textbullet\ Manual hyperparameter tuning \\  \cline{2-5}
         
         
	\bottomrule
\end{longtable}
}
\clearpage
\twocolumn


\clearpage
\onecolumn

{\tiny
\begin{longtable}{|l|p{0.6cm}|p{11.8cm}|p{0.6cm}|p{2cm}|}
	\caption{The passages on GPU programming}\label{tab:gpu_passages}                                                                                                                                                                                                                                                                                                                                                                                                                                                              \\

	\toprule
	\textbf{Cat.} & \textbf{ID} & \textbf{Text Passages}                                                                                                                                                                                                                                                                                                                                                                                                                                                                                              & \textbf{Ref.} & \textbf{Codes} \\
	\midrule
	\endfirsthead

	\multicolumn{5}{c}{Table \thetable{} -- continued from previous page}                                                                                                                                                                                                                                                                                                                                                                                                                                                                  \\
	\toprule
	\textbf{Cat.} & \textbf{ID} & \textbf{Text Passages}                                                                                                                                                                                                                                                                                                                                                                                                                                                                                              & \textbf{Ref.} & \textbf{Codes} \\
	\midrule
	\endhead
    \hline
	\multirow{32}{*}{\rotatebox[origin=c]{90}{RQ\textsubscript{4}: Key Motivating Factors}}
	     & \label{G1011} G1011 \newline\centering\cite{chetlur_cudnn_2014} 
         & Deep learning workloads are computationally intensive, and optimizing their kernels is difficult and time-consuming. As parallel architectures evolve, kernels must be reoptimized, which makes maintaining codebases difficult over time.  The computations that arise when training and using deep neural networks lend themselves naturally to efficient parallel implementations. 
	     & \cite{chetlur_cudnn_2014,krizhevsky_imagenet_2012}
	     & \textbullet\ Optimizing deep-learning kernels \newline \textbullet\ Surging demand for scalability \\
        
         \cline{2-5}
        
         & \label{G1012} G1012 \newline\centering\cite{chetlur_cudnn_2014}
         & Parallel processors such as GPUs have played a significant role in the practical implementation of deep neural net-works. The computations that arise when training and using deep neural networks lend themselves naturally to efficient parallel implementations. The efficiency provided by these implementations al-lows researchers to explore significantly higher capacity networks, training them on larger datasets [7]. 
         & \cite{chetlur_cudnn_2014,okuta_cupy_2017}
         & \textbullet\ Breakthroughs that provide computational resources \newline \textbullet\ Natural parallelizability \newline \textbullet\ Data availability \\

         \cline{2-5}
        
         & \label{G1013} G1013 \newline\centering\cite{chetlur_cudnn_2014}
         & The deep learning community has been successful in finding optimized implementations of these kernels, but as the underlying architectures evolve, these kernels must be re-optimized, which is a significant investment. (...) To address this problem, we have created a library similar in intent to BLAS, with optimized routines for deep learning workloads. Our implementation contains routines for GPUs, although similarly to the BLAS library, these routines could be implemented for other platforms. The library is easy to integrate into existing frameworks, and provides optimized performance and memory usage. 
         & \cite{chetlur_cudnn_2014,okuta_cupy_2017}
         & \textbullet\ Optimizing deep-learning kernels (investment) \newline \textbullet\ Easy integration into existing frameworks \\

         \cline{2-5}
        
         & \label{G1014} G1014 \newline\centering\cite{chetlur_cudnn_2014}
         & Several deep learning projects at Baidu have integrated cuDNN. For example, it has been integrated into PADDLE, Baidu’s internal deep 
         learning framework. (...) cuDNN computation is transparent to the user through drop-in integration. The model schema and framework interfaces 
         are completely unchanged. Setting a single compilation flag during installation equips Caffe with cuDNN layer implementations and sets cuDNN 
         as the default computation engine.
         & \cite{chetlur_cudnn_2014,okuta_cupy_2017,Jia.EtAl_2014a}
         & \textbullet\ Integration into existing frameworks \newline \textbullet\ Transparent integration \\

         \cline{2-5}
         & \label{G1015} G1015 \newline\centering\cite{chetlur_cudnn_2014}
         & Our implementation contains routines for GPUs, although similarly to the BLAS library, these routines could be implemented for other platforms. (...) The library exposes a host-callable C language API, but requires that input and output data be resi-dent on the GPU, analogously to cuBLAS.
         & \cite{chetlur_cudnn_2014}
         & \textbullet\ Interaction between GPU and CPU \\

         \cline{2-5}
         & \label{G1016} G1016 \newline\centering\cite{chetlur_cudnn_2014}
         &  With cuDNN, it is possible to write programs that train standard convolutional neural networks without writing any parallel code, but simply using cuDNN and cuBLAS. (...) Firstly, deep learning frameworks can focus on higher-level issues rather than close optimization of parallel kernels to specific hardware platforms. Secondly, as parallel architectures evolve, library providers can provide performance portability, in much the same way as the BLAS routines provide performance portability to diverse applications on diverse hardware. Thirdly, a clearer separation of concerns allows specialization: library providers can take advantage of their deep understanding of parallel architectures to provide optimal efficiency. Our goal is to make it much easier for deep learning frameworks to take advantage of parallel hardware.
         & \cite{chetlur_cudnn_2014,Goodfellow.EtAl_2013}
         & \textbullet\ Meeting user requirements\\

         \cline{2-5}
         & \label{G1017} G1017 \newline\centering\cite{chetlur_cudnn_2014}
         & One of the primary goals of cuDNN is to enable the community of neural network frameworks to benefit equally from its APIs. Accordingly, users of cuDNN are not required to adopt any particular software framework, or even data layout. (...) 
         Rather than providing a layer abstraction, we provide lower-level computational primitives, in order to simplify integration with existing deep learning frameworks, each with their own abstractions.
         & \cite{chetlur_cudnn_2014,Collobert.EtAl_}
         & \textbullet\ Self-contained framework \newline \textbullet\ Lower-level abstractions \\

         \cline{2-5}
         & \label{G1061} G1061 \newline\centering\cite{okuta_cupy_2017}
         & NumPy provides multi-dimensional arrays, the fundamental data structure for scientific computing, and a variety of operations and functions. 
         (...) Deep learning computations principally require linear algebra computations, which is one of NumPy’s strengths. However, NumPy does not 
         support calculations on GPUs. This was the motivation to develop CuPy – to fully benefit from fast computations using the latest GPUs with a 
         NumPy-compatible interface.
         & \cite{okuta_cupy_2017,chetlur_cudnn_2014}
         & \textbullet\ Existing tools not GPU compatible \newline \textbullet\ Deep learning involves linear algebra computations \\

         \cline{2-5}
         & \label{G1062} G1062 \newline\centering\cite{okuta_cupy_2017}
         & CuPy 1 is an open-source library with NumPy syntax that increases speed by doing matrix operations on NVIDIA GPUs. It is accelerated with 
         the CUDA platform from NVIDIA and also uses CUDA-related libraries, including cuBLAS, cuDNN, cuRAND, cuSOLVER, cuSPARSE, and NCCL, to make 
         full use of the GPU architecture. CuPy’s interface is highly compatible with NumPy.
         & \cite{okuta_cupy_2017}
         & \textbullet\ Building on existing tools \\

         \cline{2-5}
         & \label{G1041} G1041 \newline\centering\cite{Jia.EtAl_2014a}
         & While deep neural networks have attracted enthusiastic interest within computer vision and beyond, replication of published results can involve months of work by a researcher or engineer. (...) But trained models alone are not sufficient for rapid research progress and emerging commercial applications, and few toolboxes offer truly off-the-shelf deployment of state-of-the-art models—and those that do are often not computationally efficient and thus unsuitable for commercial deployment.
         & \cite{Jia.EtAl_2014a,Collobert.EtAl_,Goodfellow.EtAl_2013}
         & \textbullet\ Scalability and deployment \newline \textbullet\ Usability\\

         \cline{2-5}

         & \label{G1071} G1071 \newline\centering\cite{Collobert.EtAl_}
         & With Torch7, we aim at providing a framework with three main advantages: (1) it should ease the development of numerical algorithms, (2) it 
         should be easily extended (including the use of other libraries), and (3) it should be fast. (...) We found that a scripting (interpreted) 
         language with a good C API appears as a convenient solution to “satisfy” the constraint (2). (...) Among existing scripting languages1 finding 
         the ones that satisfy condition (3) severely restricted our choice. We chose Lua, the fastest interpreted language (with also the fastest Just 
         In Time (JIT) compiler2) we could find.
         & \cite{Collobert.EtAl_,Goodfellow.EtAl_2013,Jia.EtAl_2014a}
         & \textbullet\ Performance \newline \textbullet\ Usability \newline \textbullet\ Extensibility \\
         \cline{2-5}


         & \label{G1031} G1031 \newline\centering\cite{Goodfellow.EtAl_2013}
         & The goal of the library is to facilitate machine learning research. This means that the library has a focus on flexibility and extensibility, in order to make sure that nearly any research idea is feasible to implement in the library. The target user base is machine learning researchers. Being "user friendly" for a research user means that it should be easy to understand exactly what the code is doing and configure it very precisely for any desired experiment.
         & \cite{Goodfellow.EtAl_2013,Collobert.EtAl_}
         & \textbullet\ Performance \newline \textbullet\ Expert user-base \newline \textbullet\ Flexible and extensible \\
         \cline{2-5}


         & \label{G1051} G1051 \newline\centering\cite{krizhevsky_imagenet_2012}
         & We trained a large, deep convolutional neural network to classify the 1.2 million high-resolution images (...) The specific contributions of this paper are as follows: we trained one of the largest convolutional neural networks to date on the subsets of ImageNet used in the ILSVRC-2010 and ILSVRC-2012 competitions [2] and achieved by far the best results ever reported on these datasets. We wrote a highly-optimized GPU implementation of 2D convolution and all the other operations inherent in training convolutional neural networks, which we make available publicly.
         & \cite{krizhevsky_imagenet_2012,chetlur_cudnn_2014}
         & \textbullet\ Task-specific optimizations \newline \textbullet\ Object-recognition \\
         \hline


         \multirow{15}{*}{\rotatebox[origin=c]{90}{RQ\textsubscript{3}: Critical Factors}}
         & \label{G2011} G2011 \newline\centering\cite{chetlur_cudnn_2014}
         & It can provide immediate efficiency gains, and it is rigorously tested and maintained in order to be reliable and performant across a range of different processor architectures. Importantly, our library is designed to use the minimum possible amount of auxiliary memory, which frees up scarce memory for larger models and datasets. We also optimize performance across a wide range of potential use cases, including small mini-batch sizes.
         & \cite{chetlur_cudnn_2014, Jia.EtAl_2014a, Collobert.EtAl_}
         & \textbullet\ Scalability \newline \textbullet\ Performance \\

         \cline{2-5}
         & \label{G2012} G2012 \newline\centering\cite{chetlur_cudnn_2014}
         & Firstly, deep learning frameworks can focus on higher-level issues rather than close optimization of parallel kernels to specific hardware platforms. Secondly, as parallel architectures evolve, library providers can provide performance portability, in much the same way as the BLAS routines provide performance portability to diverse applications on diverse hardware. Thirdly, a clearer separation of concerns allows specialization: library providers can take advantage of their deep understanding of parallel architectures to provide optimal efficiency. Our goal is to make it much easier for deep learning frameworks to take advantage of parallel hardware.
         & \cite{chetlur_cudnn_2014, Jia.EtAl_2014a}
         & \textbullet\ Separation of concerns \newline \textbullet\ Focus on higher-level design \newline \textbullet\ Performance portability \\
         \cline{2-5}

         & \label{G2021} G2021 \newline\centering\cite{Collobert.EtAl_} 
         & Torch7 has been designed with efficiency in mind, leveraging SSE when possible and supporting two ways of parallelization: OpenMP and CUDA. Open Multi-Processing (OpenMP) provides a shared memory CPU parallelization framework on C/C++ and Fortran languages on almost every operating system and compiler toolset. (...) Torch7 is designed to be easily interfaced with third-party software thanks to Lua's interface
	     & \cite{Collobert.EtAl_, Jia.EtAl_2014a}
	     & \textbullet\ Performance \newline \textbullet\ Cross-framework compatibility \newline \textbullet\ Heterogenous hardware \\
         \cline{2-5}

         & \label{G2041} G2041 \newline\centering\cite{Jia.EtAl_2014a} 
         &  Caffe fits industry and internet-scale media needs by CUDA GPU computation, processing over 40 million images a day on a single K40 or Titan GPU (...) Switching between a CPU and GPU implementation is exactly one function call. (...) Separation of representation and implementation. Caffe model definitions are written as config files using the Protocol Buffer language. (...) The code is written in clean, efficient C++, with CUDA used for GPU computation, and nearly complete, well-supported bindings to Python/Numpy and MATLAB.
	     & \cite{Jia.EtAl_2014a, chetlur_cudnn_2014, Collobert.EtAl_}
	     & \textbullet\ Heterogenous hardware \newline \textbullet\ Scalability \newline \textbullet\ Separation of concerns \newline \textbullet\ Usability \\

         \cline{2-5}
         & \label{G2051} G2051 \newline\centering\cite{krizhevsky_imagenet_2012} 
         & Current GPUs are particularly well-suited to cross-GPU parallelization, as they are able to read from and write to one another's memory directly, without going through host machine memory. The parallelization scheme that we employ essentially puts half of the kernels (or neurons) on each GPU, with one additional trick: the GPUs communicate only in certain layers.
	     & \cite{krizhevsky_imagenet_2012, chetlur_cudnn_2014}
	     & \textbullet\ Inter-GPU communication \\

         \cline{2-5}

	     & \label{G2061} G2061 \newline\centering\cite{okuta_cupy_2017} 
         & CuPy implements many functions on cupy.ndarray objects. See the reference 2 for the supported subset of NumPy API. Since CuPy covers most NumPy features, reading the NumPy documentation can be helpful for using CuPy.
	     & \cite{okuta_cupy_2017}
	     & \textbullet\ Declarative programming\\


         \hline

         \multirow{23}{*}{\rotatebox[origin=c]{90}{RQ\textsubscript{2}: Evaluation Metrics}}
         & \label{G3011} G3011 \newline\centering\cite{chetlur_cudnn_2014}
         & The convolution routines in cuDNN provide competitive performance with zero auxiliary memory required.
         Figure 2 shows the performance on an NVIDIA Tesla K40 of three convolution implementations: cuDNN, Caffe, and cuda-convnet2. We evaluated these implementations using the layer configurations shown in table 2, which are commonly used for benchmarking convolution performance [1], and quote average throughput for all these layers. (...) Compared to Caffe, cuDNN per-formance ranges from 1.0× to 1.41×. Importantly, even with a small mini-batch size of only 16, cuDNN performance is still 86\% of maximum performance, which shows that our implementation performs well across the convolution parameter space.
         & \cite{chetlur_cudnn_2014,Goodfellow.EtAl_2013,Jia.EtAl_2014a}
         & \textbullet\ Model Architecture (convolution performance, mini-batch evaluation, efficiency) \\

         \cline{2-5}

         & \label{G3012} G3012 \newline\centering\cite{chetlur_cudnn_2014}
         & We present a library of efficient implementations of deep learning primitives. (...) Convolutional Neural Networks (CNNs) [14] are an important and successful class of deep networks. (...) We are also using cuDNN in other domains besides image processing, such as speech and language. cuDNN's ability to convolve non-square inputs with asymmetric padding is particularly useful for these other domains. 
         & \cite{chetlur_cudnn_2014,okuta_cupy_2017}
         & \textbullet\ Domains: deep learning, CNNs, general purpose\\

         \cline{2-5}

         & \label{G3013} G3013 \newline\centering\cite{chetlur_cudnn_2014}
         & NVIDIA provides a matrix multiplication routine that achieves a substantial fraction of floating-point throughput on GPUs. (...) Over-all, training time for 200 iterations improved by 36\%, when training the bvlc reference caffenet model, using cuDNN R1 on an NVIDIA Tesla K40. (...) Figure 2 shows the performance on an NVIDIA Tesla K40 of three convolution implementations: cuDNN, Caffe, and cuda-convnet2. cuDNN performance ranges from 0.8× to 2.25× that of cuda-convnet2, with an advantage at smaller batch sizes. Compared to Caffe, cuDNN per-formance ranges from 1.0× to 1.41×. (...) This data illustrates how cuDNN provides performance portability across GPU architectures, with no need for users to retune their code as GPU architectures evolve.
         & \cite{chetlur_cudnn_2014}
         & \textbullet\ Evaluation: performance, benchmarking and portability\\

         \cline{2-5}
         & \label{G3031} G3031 \newline\centering\cite{Goodfellow.EtAl_2013}
         & Because the philosophy that Pylearn2 developers should write features when they are needed, and because most Pylearn2 developers so far have been deep learning researchers, Pylearn2 mostly con-tains deep learning models or models that are used as building blocks for deep architectures. This includes autoencoders [6], RBMs [47] including RBMs with Gaussian visible units [54], DBMs [45], MLPs [44], convolutional networks [33], and local coordinate coding [56]. 
         & \cite{Goodfellow.EtAl_2013,chetlur_cudnn_2014,Jia.EtAl_2014a}
         & \textbullet\ Model architectures \\

         \cline{2-5}
         & \label{G3041} G3041 \newline\centering\cite{Jia.EtAl_2014a}
         & It powers on-going research projects, large-scale industrial applications, and startup prototypes in vision, speech, and multimedia. (...)  The same models can be run in CPU or GPU mode on a variety of hardware (...) The code is written in clean, efficient C++, with CUDA used for GPU computation, and nearly complete, well-supported bindings to Python/Numpy and MATLAB. (...) Separation of representation and implementation. Caffe model definitions are written as config files using the Protocol Buffer language.
         & \cite{Jia.EtAl_2014a,chetlur_cudnn_2014,Goodfellow.EtAl_2013}
         & \textbullet\ Deployment \newline \textbullet\ Model architectures \\

         \cline{2-5}
         & \label{G3051} G3051 \newline\centering\cite{krizhevsky_imagenet_2012}
         & In the left panel of Figure 4 we qualitatively assess what the network has learned by computing its top-5 predictions on eight test images. Notice that even off-center objects, such as the mite in the top-left, can be recognized by the net. Most of the top-5 labels appear reasonable. 
         & \cite{krizhevsky_imagenet_2012}
         & \textbullet\ Qualitative evaluation \\

         \cline{2-5}
         & \label{G3061} G3061 \newline\centering\cite{okuta_cupy_2017}
         & For example, a Python-based probabilistic modeling software, Pomegranate [4], uses CuPy as its GPU backend. We believe this is thanks to CuPy's NumPy-like design and strong performance based on NVIDIA libraries. (...) CuPy runs NumPy code at GPU calculation speeds. Though developed as the array back-end for the deep learning framework Chainer, it can also be used for general purpose, scientific computing on GPU.
         & \cite{okuta_cupy_2017,chetlur_cudnn_2014}
         & \textbullet\ Domain: scientific computing, probabilistic modelling\\


         \hline

         %\multirow{23}{*}{\rotatebox[origin=c]{90}{RQ\textsubscript{2}: Limitations and Challenges}}
         & \label{G4011} G4011 \newline\centering\cite{chetlur_cudnn_2014}
         & Optimizing and maintaining all these specializations is a difficult task. As we envision this library being maintained for some time, and being ported to yet-to-be-conceived future architectures, we searched for something simpler that would perform more robustly across the parameter space and be easier to port to new architectures. (...) We are considering several avenues for expanding the performance and functionality of cuDNN. (...) Finally, we would like this library to help people use multiple GPUs to accelerate training.
         & \cite{chetlur_cudnn_2014,krizhevsky_imagenet_2012}
         & \textbullet\ Outstanding challenges due to future architectures \newline \textbullet\ Multi-GPU training \\
         \cline{2-5}

         & \label{G4012} G4012 \newline\centering\cite{chetlur_cudnn_2014}
         & The deep learning community has been successful in finding optimized implementations of these kernels, but as the underlying architectures evolve, these kernels must be re-optimized, which is a significant investment. Optimizing these kernels requires a deep understanding of the underlying processor architecture, with careful scheduling of data movement, on-chip memory placement, register blocking, and other optimizations in order to get acceptable performance.
         & \cite{chetlur_cudnn_2014}
         & \textbullet\ Usability: Optimization of kernels is time-consuming \\
         \cline{2-5}
         
         \multirow{15}{*}{\rotatebox[origin=c]{90}{RQ\textsubscript{2}: Limitations \& Challenges}}
         & \label{G4013} G4013 \newline\centering\cite{chetlur_cudnn_2014}
         & There are several ways to implement convolutions efficiently. Our goal is to provide performance as close as possible to matrix multiplication, while using no auxiliary memory. GPU memory is high bandwidth, but low capacity, and is therefore a scarce resource. (...) Some approaches, like lowering convolutions to matrix multiplication or using Fast Fourier Transform (FFT), have drawbacks (...) Another common approach is to compute the convolutions directly. This can be very efficient, but requires a large number of specialized implementations to handle the many corner cases implicit in the 11-dimensional parameter space of the convolutions. (...) Additionally, the FFT based approach does not per-form efficiently when the striding parameters u and v are greater than 1, which is common in many state-of-the art networks
         & \cite{chetlur_cudnn_2014,chetlur_cudnn_2014}
         & \textbullet\ Algorithmic limitation: high memory usage (FFT, matrix multiplication), direct compute (specialized implementations) \\
         \cline{2-5}

        
         & \label{G4041} G4041 \newline\centering\cite{Jia.EtAl_2014a}
         & While deep neural networks have attracted enthusiastic interest within computer vision and beyond, replication of published results can involve months of work by a researcher or engineer. Sometimes researchers deem it worthwhile to release trained models along with the paper advertising their performance. But trained models alone are not sufficient for rapid research progress and emerging commercial applications, and few toolboxes offer truly off-the-shelf deployment of state-of-the-art models—and those that do are often not computationally efficient and thus unsuitable for commercial deployment.
         & \cite{Jia.EtAl_2014a}
         & \textbullet\ Usability: Replication of results is challenging \\
         \cline{2-5}

         & \label{G4051} G4051 \newline\centering\cite{krizhevsky_imagenet_2012}
         & The parallelization scheme that we employ essentially puts half of the kernels (or neurons) on each GPU, with one additional trick: the GPUs communicate only in certain layers. This means that, for example, the kernels of layer 3 take input from all kernel maps in layer 2. However, kernels in layer 4 take input only from those kernel maps in layer 3 which reside on the same GPU. Choosing the pattern of connectivity is a problem for cross-validation, but this allows us to precisely tune the amount of communication until it is an acceptable fraction of the amount of computation.
         & \cite{krizhevsky_imagenet_2012,chetlur_cudnn_2014}
         & \textbullet\ Communication overhead \\
         \cline{2-5}

         & \label{G4061} G4061 \newline\centering\cite{okuta_cupy_2017}
         & It is accelerated with the CUDA platform from NVIDIA and also uses CUDA-related libraries, including cuBLAS, cuDNN, cuRAND, cuSOLVER, cuSPARSE, and NCCL, to make full use of the GPU architecture. For small matrices, CuPy is slower than NumPy since there is some overhead in CuPy from the CUDA kernel launch. For larger matrices, the overhead is small compared to the actual GPU computation, and CuPy is up to six times faster than CPU-based NumPy.
         & \cite{okuta_cupy_2017,chetlur_cudnn_2014}
         & \textbullet\ Limitations: small matrix performance \\
         \cline{2-5}


	\bottomrule
\end{longtable}
}
\clearpage
\twocolumn


\begin{table*}[htbp]
    \centering 
    \caption{Artifacts}
    \label{tab:artefacts}
    \begin{tabularx}{\textwidth}{lXX}
        \hline
        \textbf{Link} & \textbf{Comment} \\
        \hline
        \href{https://1drv.ms/x/c/a7d18f02247e70f7/EU6EIrW0ubBBlxLX8JvVxjgBcGvPiTlZgRLTdLrM9-S4bw?e=os2ugX}{M.4 Reading the Studies} & 
        The information extracted while evaluating the studies. \\
        \hline
    \end{tabularx}
\end{table*}

\subsubsection{Inclusion Criteria}
\begin{itemize}
    \item Studies published between 2013 and 2023
    \item Peer-reviewed articles
    \item Studies focusing on distributed training techniques
    \item Articles written in English
\end{itemize}

\subsubsection{Exclusion Criteria}
\begin{itemize}
    \item Studies not focused on neural network training
    \item Pure theoretical papers without implementation
    \item Secondary studies (surveys, reviews)
\end{itemize}

\subsection{Quality Assessment}
Studies were evaluated using the following quality criteria:
\begin{enumerate}
    \item Clear description of the distributed technique
    \item Empirical evaluation of the proposed method
    \item Comparison with existing approaches
    \item Discussion of limitations and threats to validity
\end{enumerate}

\subsection{Data Extraction}
Data was extracted using a standardized form capturing:
\begin{itemize}
    \item Publication details
    \item Distributed training approach
    \item Implementation details
    \item Performance metrics
    \item Experimental setup
\end{itemize} 


\section{Notes}

\subsection{Conducting the Review}
The stages associated with conducting the review are:
\begin{itemize}
	\item Identification of research (See Section \TODO{Reference section number}).
	      \begin{itemize}
		      \item \textbf{Initial Search:} This stage involves using the defined search terms within selected
		            databases to identify relevant studies, as further detailed in section \ref{sec:search_process_documentation} (Search Process Documentation). The process of how these terms are combined to create search
		            strings is described in section \ref{sec:search_process_documentation}, and the search results will be stored using \textbf{Zotero}.
	      \end{itemize}
	\item Selection of primary studies (See Section \TODO{Reference section number}).
	      \begin{itemize}
		      \item \textbf{Screening:} This stage involves an initial screening of titles and abstracts to remove
		            irrelevant studies, which is part of the study selection process described in section \ref{sec:study-selection-criteria} (Study Selection Criteria).
		      \item \textbf{Full-Text Review:} All potentially relevant studies will have their full texts retrieved,
		            and the full texts will then be assessed against pre-defined inclusion and exclusion criteria (see section \ref{sec:study-selection-criteria}).
	      \end{itemize}
	\item Study quality assessment (See Section \TODO{Reference section number}).
	\item Data extraction and monitoring (See Section \TODO{Reference section number}).
	      \begin{itemize}
		      \item \textbf{Data Extraction:} The final step is data extraction, where relevant information will be
		            extracted from the included studies using a predefined data extraction form (detailed in section \ref{sec:data-extraction-strategy}).
	      \end{itemize}
	\item Data synthesis (See Section \TODO{Reference section number}).
\end{itemize}

\subsection{Reporting the Review}
The stages associated with reporting the review are:
\begin{itemize}
	\item Specifying dissemination mechanisms (See Section \TODO{Reference section number}).
	\item Formatting the main report (See Section \TODO{Reference section number}).
	\item Evaluating the report (See Section \TODO{Reference section number}).
\end{itemize}

We consider all the above stages to be mandatory except:
\begin{itemize}
	\item Commissioning a review which depends on whether or not the systematic review is being done on a
	      commercial basis.
	\item Evaluating the review protocol and Evaluating the report which are optional and depend on the
	      quality assurance procedures decided by the systematic review team (and any other stakeholders).
\end{itemize}

The stages listed above may appear to be sequential, but it is important to recognise that many of
the stages involve iteration. In particular, many activities are initiated during the protocol
development stage, and refined when the review proper takes place. For example:
\begin{itemize}
	\item The selection of primary studies is governed by inclusion and exclusion criteria. These criteria
	      are initially specified when the protocol is drafted but may be refined after quality criteria are
	      defined.
	\item Data extraction forms initially prepared during construction of the protocol will be amended when
	      quality criteria are agreed.
	\item Data synthesis methods defined in the protocol may be amended once data has been collected.
\end{itemize}

