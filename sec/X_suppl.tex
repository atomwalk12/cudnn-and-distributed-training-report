\clearpage
\setcounter{page}{1}
\maketitlesupplementary

\section{Supplementary Materials}
\label{sec:supplementary}

\subsection{Detailed Search Strategy}
\label{sec:search_strategy}

\subsubsection{Search Documentation}

The search string was constructed using the terms shown in Table \ref{tab:search_terms}.
\begin{table*}[htbp!]
    \centering
    \caption{Core Search Terms for Distributed Deep Learning and GPU Programming}
    \label{tab:search_terms}
    \begin{tabularx}{\textwidth}{|l|X|X|}
        \hline
        \textbf{Category} & \textbf{Distributed Learning} & \textbf{GPU Computing} \\
        \hline
        Core Terms & 
        \textbf{"Distributed Deep Learning"},
        \textbf{"Parallel Deep Learning"},
        "Large-Scale Deep Learning" &
        \textbf{"GPU Programming"},
        \textbf{"CUDA Programming"} \\
        \hline
        Technical Approach & 
        \textbf{"Data Parallelism"},
        \textbf{"Model Parallelism"},
        \textbf{"Hybrid Parallelism"} &
        \textbf{"CUDA"},
        "GPU Optimization",
        "Parallel Computing" \\
        \hline
        Implementation & 
        \textbf{"Parameter Server"},
        \textbf{"All-Reduce"},
        \textbf{"SGD"} &
        \textbf{"CUDA Toolkit"},
        \textbf{"cuDNN"},
        "Multi-GPU" \\
        \hline
        Frameworks & 
        "TensorFlow",
        "PyTorch",
        "Horovod" &
        "TensorRT",
        "PyCUDA",
        "Numba" \\
        \hline
    \end{tabularx}
    \caption*{Note: Bold terms indicate primary search terms that will be prioritized.}
\end{table*}

The starting date used for the analysis starts from 2014 till 2024. The reason for utilizing this range
is due to the publishing date of \cite{SierraCanto2010ParallelTO}, which acts as a key paper in the domain. As a result,
I'd like to analyze the advancement of the field starting from this age. The key The following table 
documents our complete search process:

\begin{table*}[htbp!]
    \centering
    \caption{Detailed Search Documentation}
    \label{tab:search_documentation}
    \begin{tabularx}{\textwidth}{|l|X|c|c|c|}
        \hline
        \textbf{Database} & \textbf{Search String} & \textbf{Years Covered} & \textbf{Results} & \textbf{Filtered} \\
        \hline
        Scopus & ("Distributed Deep Learning" OR "Parallel Deep Learning") AND "Data Parallelism" & 2017-2024 & 82 & 11 \\
               & ( "machine learning" OR "deep learning" ) AND ( "Data parallelism" OR "model parallelism" OR "pipeline parallelism" OR "hybrid parallelism" ) AND ( "framework" OR "implementation" ) & 2012-2024 & 206 & 11 \\
        \hline
        ACM Digital Library & ("GPU Programming" OR "GPGPU Programming") AND ("CUDA" OR "CUDA Programming") AND ("Parallel Computing") & 2015-2022 & 2024-05-10 & 210 \\
        \hline
        Science Direct & [Exact search string] & 2015-2022 & 2024-05-10 & XXX \\
        \hline
        arXiv & [Exact search string] & 2015-2022 & 2024-05-10 & XXX \\
        \hline
    \end{tabularx}
\end{table*}


\subsubsection{Search Process Details}
\begin{itemize}
    \item \textbf{Years Covered:} 2015-2022
    \item \textbf{Language Restrictions:} English only
    \item \textbf{Document Types:} Journal articles, conference papers, and high-quality preprints
\end{itemize}

\subsubsection{Manual Searches}
The following additional sources were manually searched:
\begin{itemize}
    \item Key conference proceedings (e.g., NeurIPS, ICLR, ICML)
    \item Reference lists of included studies (snowballing)
    \item Citations of included studies (forward snowballing)
\end{itemize}

\subsection{Study Selection Details}
\label{sec:study_selection}

\subsubsection{Included Studies}
Table \ref{tab:included_studies} lists all studies that met our inclusion criteria:

\begin{table*}[htbp!]
    \centering
    \caption{Detailed List of Included Studies}
    \label{tab:included_studies}
    \begin{tabularx}{\textwidth}{|l|l|X|c|c|X|}
        \hline
        \textbf{ID} & \textbf{Authors} & \textbf{Title} & \textbf{Year} & \textbf{Quality Score} & \textbf{Key Findings} \\
        \hline
        S1 & Author et al. & Title of study 1 & 20XX & X.X & Brief summary of main findings \\
        \hline
        S2 & Author et al. & Title of study 2 & 20XX & X.X & Brief summary of main findings \\
        \hline
    \end{tabularx}
\end{table*}

\subsubsection{Excluded Studies}
Table \ref{tab:excluded_studies} lists studies that were excluded during the screening process:

\begin{table*}[htbp!]
    \centering
    \caption{List of Excluded Studies with Reasons}
    \label{tab:excluded_studies}
    \begin{tabularx}{\textwidth}{|l|l|X|c|X|}
        \hline
        \textbf{ID} & \textbf{Authors} & \textbf{Title} & \textbf{Year} & \textbf{Reason for Exclusion} \\
        \hline
        E1 & Author et al. & Title of excluded study 1 & 20XX & Does not meet inclusion criterion 1 \\
        \hline
        E2 & Author et al. & Title of excluded study 2 & 20XX & Insufficient technical details \\
        \hline
    \end{tabularx}
\end{table*}

\subsubsection{Study Quality Assessment}
Table \ref{tab:quality_assessment} provides detailed quality scores for included studies:

\begin{table*}[htbp!]
    \centering
    \caption{Quality Assessment Scores for Included Studies}
    \label{tab:quality_assessment}
    \begin{tabularx}{\textwidth}{|l|c|c|c|c|X|}
        \hline
        \textbf{Study ID} & \textbf{Methodology} & \textbf{Implementation} & \textbf{Evaluation} & \textbf{Total Score} & \textbf{Notes} \\
        \hline
        S1 & X.X & X.X & X.X & X.X & Brief quality notes \\
        \hline
        S2 & X.X & X.X & X.X & X.X & Brief quality notes \\
        \hline
    \end{tabularx}
\end{table*}

\subsection{Data Extraction Forms}
\label{sec:data_extraction}

The following template was used for data extraction:
\begin{itemize}
    \item \textbf{Study ID:} [Unique identifier]
    \item \textbf{Authors:} [Author names]
    \item \textbf{Year:} [Publication year]
    \item \textbf{Venue:} [Publication venue]
    \item \textbf{Distribution Strategy:} [Description]
    \item \textbf{Implementation Details:} [Technical details]
    \item \textbf{Evaluation Metrics:} [Performance measures]
    \item \textbf{Results:} [Key findings]
\end{itemize}

\subsection{Quality Assessment Checklist}
\label{sec:quality_checklist}

Each study was evaluated using the following criteria:
\begin{enumerate}
    \item \textbf{Problem Definition} (0-2 points)
        \begin{itemize}
            \item 2: Clear and well-motivated problem statement
            \item 1: Partially clear problem statement
            \item 0: Unclear problem statement
        \end{itemize}
    \item \textbf{Methodology Description} (0-2 points)
        \begin{itemize}
            \item 2: Detailed and replicable methodology
            \item 1: Partial methodology description
            \item 0: Insufficient methodology description
        \end{itemize}
    % Add more quality criteria as needed
\end{enumerate}

\subsection{Raw Data}
\label{sec:raw_data}

\subsubsection{Performance Metrics}
[Tables or figures showing raw performance data]

\subsubsection{Statistical Analysis}
[Detailed statistical analysis of the results]

\section{Conflicts of Interest}
\label{sec:conflicts}

The authors declare no conflicts of interest that could have appeared to influence the work reported in this paper. This research did not receive any specific grant from funding agencies in the public, commercial, or not-for-profit sectors.

% NOTE not needed
% \section{Author Contributions}
% \label{sec:contributions}

% \begin{itemize}
%     \item \textbf{First Author:} Conceptualization, Methodology, Writing - Original draft
%     \item \textbf{Second Author:} Data curation, Formal analysis, Writing - Review \& editing
% \end{itemize}

% All authors have read and agreed to the published version of the manuscript.


\subsection{Review Protocol}
This systematic review follows the guidelines proposed by Kitchenham and Charters for software engineering research. The protocol was developed and reviewed by all authors before beginning the review process.

\subsection{Data Sources and Search Strategy}
We searched the following digital libraries:
\begin{itemize}
    \item IEEE Xplore
    \item ACM Digital Library
    \item Science Direct
    \item arXiv (for preprints)
\end{itemize}


% TODO Razvan: remove this table
% \begin{table*}[htbp]
%     \centering
%     \caption{Search Terms by Category for Distributed Deep Learning and GPU Programming}
%     \label{tab:search_terms}
%     \begin{tabularx}{\textwidth}{|l|X|X|}
%         \hline
%         \textbf{Facet} & \textbf{Distributed Deep Learning Terms} & \textbf{GPU Programming Terms} \\
%         \hline
%         Core Concept & 
%         \textbf{"Distributed Deep Learning"}, \textbf{"Parallel Deep Learning"}, 
%         "Deep Learning on Clusters", "Large-Scale Deep Learning", 
%         "Scalable Deep Learning" &
%         \textbf{"GPU Programming"}, \textbf{"General-Purpose GPU Programming"}, 
%         \textbf{"GPGPU Programming"} \\
%         \hline
%         Specific Technology / 
%         Parallelization Techniques & 
%         \textbf{"Data Parallelism"}, \textbf{"Model Parallelism"}, 
%         \textbf{"Hybrid Parallelism"}, "Data-Parallel", "Model-Parallel" &
%         \textbf{"CUDA"}, \textbf{"CUDA Programming"}, "Nvidia CUDA", 
%         "Compute Unified Device Architecture" \\
%         \hline
%         Training Methods / 
%         Programming Aspects & 
%         \textbf{"Stochastic Gradient Descent"}, \textbf{"SGD"}, "Mini-batch SGD", 
%         "Asynchronous SGD", "Synchronous SGD", "Distributed Stochastic Gradient Descent", 
%         "Elastic Averaging SGD", "Byzantine-tolerant gradient descent" &
%         "Parallel Computing", "Parallel Programming", "High-Performance Computing", 
%         "Kernel Programming", "GPU Memory Management", "GPU Optimisation", 
%         "CUDA Libraries" \\
%         \hline
%         Communication Strategies & 
%         \textbf{"Parameter Server"}, \textbf{"All-Reduce"}, 
%         \textbf{"Collective Communication"}, "Decentralized Optimization", 
%         "Decentralized Parameter Sharing", "Gradient Compression", 
%         "Sparse Communication" & -- \\
%         \hline
%         Frameworks & 
%         "TensorFlow", "PyTorch", "Horovod", "DistBelief", "Parameter Server", 
%         "SparkNet", "Petuum", "BigDL", "MXNet", "CaffeOnSpark" &
%         \textbf{"CUDA Toolkit"}, \textbf{"cuDNN"}, "TensorRT", "Thrust", 
%         "OpenACC", "RAPIDS", "PyCUDA", "Numba", "JAX", "TensorFlow with CUDA", 
%         "PyTorch with CUDA", "Caffe with CUDA", "Theano with CUDA", 
%         "MxNet with CUDA", "Darknet with CUDA" \\
%         \hline
%         Hardware & 
%         "GPUs", "CPUs", "Accelerators", "Cluster Computing", "Supercomputers", 
%         "Multi-GPU" &
%         "GPUs", "Nvidia GPUs", "Multi-GPU", "CUDA-enabled GPUs" \\
%         \hline
%         Performance Aspects & 
%         "Scalability", "Convergence", "Latency", "Communication Overhead", 
%         "Fault Tolerance" &
%         "GPU Acceleration", "Parallel Speedup", "Throughput", 
%         "Memory Bandwidth", "Latency", "Performance Optimisation" \\
%         \hline
%     \end{tabularx}
%     \caption*{Note: Bold terms indicate primary search terms that will be prioritized in the search strategy. A dash (--) indicates no specific terms for that category.}
% \end{table*}


\subsection{Study Selection}
\subsubsection{Inclusion Criteria}
\begin{itemize}
    \item Studies published between 2013 and 2023
    \item Peer-reviewed articles
    \item Studies focusing on distributed training techniques
    \item Articles written in English
\end{itemize}

\subsubsection{Exclusion Criteria}
\begin{itemize}
    \item Studies not focused on neural network training
    \item Pure theoretical papers without implementation
    \item Secondary studies (surveys, reviews)
\end{itemize}

\subsection{Quality Assessment}
Studies were evaluated using the following quality criteria:
\begin{enumerate}
    \item Clear description of the distributed technique
    \item Empirical evaluation of the proposed method
    \item Comparison with existing approaches
    \item Discussion of limitations and threats to validity
\end{enumerate}

\subsection{Data Extraction}
Data was extracted using a standardized form capturing:
\begin{itemize}
    \item Publication details
    \item Distributed training approach
    \item Implementation details
    \item Performance metrics
    \item Experimental setup
\end{itemize} 


\section{Notes}

\subsection{Conducting the Review}
The stages associated with conducting the review are:
\begin{itemize}
	\item Identification of research (See Section \TODO{Reference section number}).
	      \begin{itemize}
		      \item \textbf{Initial Search:} This stage involves using the defined search terms within selected
		            databases to identify relevant studies, as further detailed in section \ref{sec:search_process_documentation} (Search Process Documentation). The process of how these terms are combined to create search
		            strings is described in section \ref{sec:search_process_documentation}, and the search results will be stored using \textbf{Zotero}.
	      \end{itemize}
	\item Selection of primary studies (See Section \TODO{Reference section number}).
	      \begin{itemize}
		      \item \textbf{Screening:} This stage involves an initial screening of titles and abstracts to remove
		            irrelevant studies, which is part of the study selection process described in section \ref{sec:study-selection-criteria} (Study Selection Criteria).
		      \item \textbf{Full-Text Review:} All potentially relevant studies will have their full texts retrieved,
		            and the full texts will then be assessed against pre-defined inclusion and exclusion criteria (see section \ref{sec:study-selection-criteria}).
	      \end{itemize}
	\item Study quality assessment (See Section \TODO{Reference section number}).
	\item Data extraction and monitoring (See Section \TODO{Reference section number}).
	      \begin{itemize}
		      \item \textbf{Data Extraction:} The final step is data extraction, where relevant information will be
		            extracted from the included studies using a predefined data extraction form (detailed in section \ref{sec:data-extraction-strategy}).
	      \end{itemize}
	\item Data synthesis (See Section \TODO{Reference section number}).
\end{itemize}

\subsection{Reporting the Review}
The stages associated with reporting the review are:
\begin{itemize}
	\item Specifying dissemination mechanisms (See Section \TODO{Reference section number}).
	\item Formatting the main report (See Section \TODO{Reference section number}).
	\item Evaluating the report (See Section \TODO{Reference section number}).
\end{itemize}

We consider all the above stages to be mandatory except:
\begin{itemize}
	\item Commissioning a review which depends on whether or not the systematic review is being done on a
	      commercial basis.
	\item Evaluating the review protocol and Evaluating the report which are optional and depend on the
	      quality assurance procedures decided by the systematic review team (and any other stakeholders).
\end{itemize}

The stages listed above may appear to be sequential, but it is important to recognise that many of
the stages involve iteration. In particular, many activities are initiated during the protocol
development stage, and refined when the review proper takes place. For example:
\begin{itemize}
	\item The selection of primary studies is governed by inclusion and exclusion criteria. These criteria
	      are initially specified when the protocol is drafted but may be refined after quality criteria are
	      defined.
	\item Data extraction forms initially prepared during construction of the protocol will be amended when
	      quality criteria are agreed.
	\item Data synthesis methods defined in the protocol may be amended once data has been collected.
\end{itemize}
