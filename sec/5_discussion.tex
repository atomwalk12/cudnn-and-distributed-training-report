% ===== STEP 7: Synthesize Findings =====
% This section covers:
% - Re-express ideas, concepts and findings
% - Integration of both studies
\section{Discussion}
\label{sec:discussion}

% PLACEHOLDER: Synthesis of Studies
\TODO{PLACEHOLDER: Synthesis of Studies}
\subsection{Integration of Approaches}
\subsubsection{Complementary Aspects}
\begin{itemize}
    \item Combining distributed training with GPU acceleration
    \item Trade-offs between communication and computation
    \item Hybrid approaches for large-scale deployment
    \item Optimization strategies across scales
\end{itemize}

\subsection{Key Findings Synthesis}
\subsubsection{Common Themes}
\begin{itemize}
    \item Scalability challenges in both approaches
    \item Impact of hardware architecture
    \item Importance of efficient memory management
    \item Balance between parallelism and overhead
\end{itemize}

\subsection{Practical Applications}
\begin{itemize}
    \item Guidelines for implementation choices
    \item Best practices for hybrid systems
    \item Performance optimization strategies
    \item Resource allocation recommendations
\end{itemize}
\TODO{Placeholder-end}

\subsection{Interpretation of Findings}
Our systematic review answers the research questions posed in Section \ref{sec:research_questions} and reveals several key patterns in distributed training approaches:
\begin{itemize}
    \item Data parallelism remains the dominant paradigm for distributed training
    \item Communication overhead is a critical bottleneck in most implementations
    \item Synchronous approaches generally provide better convergence guarantees
    \item Asynchronous methods offer better scaling but with potential accuracy trade-offs
\end{itemize}

\subsection{Implications}
\subsubsection{Theoretical Implications}
The findings suggest several theoretical implications:
\begin{itemize}
    \item Need for better theoretical understanding of convergence in asynchronous settings
    \item Importance of communication-computation trade-offs
    \item Role of batch size in distributed training
\end{itemize}

\subsubsection{Practical Implications}
For practitioners, our review suggests:
\begin{itemize}
    \item Guidelines for choosing between synchronous and asynchronous approaches
    \item Best practices for implementing distributed training systems
    \item Common pitfalls and their solutions
\end{itemize}

\subsection{Limitations}
This review has several limitations:
\begin{itemize}
    \item Focus on published literature may miss industrial implementations
    \item Rapid pace of development in the field
    \item Limited access to implementation details in some studies
    \item Potential publication bias towards successful approaches
\end{itemize}

\subsection{Future Research Directions}
Based on our analysis, we identify several promising directions for future research:
\begin{enumerate}
    \item Development of adaptive communication strategies
    \item Integration of federated learning approaches
    \item Improved fault tolerance mechanisms
    \item Novel compression techniques for gradient communication
\end{enumerate} 