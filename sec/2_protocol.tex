% ===== STEP 2: Define Search Strategy =====
% This section covers:
% - Search strategy development
% - Documentation of search process
\section{Review Protocol and Process}
\label{sec:protocol}

This section outlines the methodology that will be used to conduct the systematic literature review, following the guidelines described in \cite{brereton_lessons_2007-1,kitchenham_procedures_nodate,budgen_reporting_2018,dos_santos_sustainable_2024}.
A systematic literature review involves several discrete activities, and existing guidelines suggest slightly different numbers and orders of these activities.
However, medical guidelines and sociological textbooks generally agree on the major stages.
This document summarizes these stages into three main phases: Planning the Review, Conducting the Review, and Reporting the Review.
\textbf{The primary goal is to ensure that the review process is both transparent and replicable}.
This includes defining the search strategy, study selection criteria, quality assessment process,
and data extraction methods. The review will investigate both distributed deep learning techniques
and their parallel implementations using CUDA.

\subsection{Review Workflow}
Figure \ref{fig:workflow} visually outlines our systematic review process, divided into three key phases: the main workflow, studies selection, and validation. The diagram clearly presents the seven main steps of our process. This section will detail how we implement these steps.

\begin{figure*}[th]
    \centering
    \includegraphics[width=\linewidth]{figures/workflow}
    \caption{Systematic review workflow showing the main steps, documentation artifacts, and validation processes.
    The workflow is divided into three main phases: main workflow (top), studies selection (middle), and
    validation (bottom). Dashed lines indicate documentation and communication flows.}
    \label{fig:workflow}
\end{figure*}

% The workflow also shows the documentation artifacts and validation processes that occur throughout the review, ensuring rigor and transparency.

\subsection{The Review Process}
% Moved the explanation of the phases here
% A systematic literature review involves several discrete activities. Existing guidelines for systematic reviews have slightly different suggestions about the number and order of activities. This document summarises the stages in a systematic review into three main phases: Planning the Review, Conducting the Review, Reporting the Review.

\subsubsection{Planning the Review}
The stages associated with planning the review are:
\begin{itemize}
    \item Identification of the need for a review (See Section \TODO{Reference section number}).
    \item Commissioning a review (See Section \TODO{Reference section number}).
    \item Specifying the research question(s) (See Section \ref{sec:research_questions}).
    \item Developing a review protocol (See Section \ref{sec:protocol}).
    \item Evaluating the review protocol (See Section \TODO{Reference section number}).
\end{itemize}

\subsubsection{Conducting the Review}
The stages associated with conducting the review are:
\begin{itemize}
    \item Identification of research (See Section \TODO{Reference section number}).
        \begin{itemize}
            \item \textbf{Initial Search:} This stage involves using the defined search terms within selected
            databases to identify relevant studies, as further detailed in section \ref{sec:search_process_documentation} (Search Process Documentation). The process of how these terms are combined to create search
            strings is described in section \ref{sec:search_process_documentation}, and the search results will be stored using \textbf{Zotero}.
        \end{itemize}
    \item Selection of primary studies (See Section \TODO{Reference section number}).
        \begin{itemize}
            \item \textbf{Screening:} This stage involves an initial screening of titles and abstracts to remove
            irrelevant studies, which is part of the study selection process described in section \ref{sec:study-selection-criteria} (Study Selection Criteria). 
            \item \textbf{Full-Text Review:} All potentially relevant studies will have their full texts retrieved,
            and the full texts will then be assessed against pre-defined inclusion and exclusion criteria (see section \ref{sec:study-selection-criteria}).
        \end{itemize}
    \item Study quality assessment (See Section \TODO{Reference section number}).
    \item Data extraction and monitoring (See Section \TODO{Reference section number}).
        \begin{itemize}
            \item \textbf{Data Extraction:} The final step is data extraction, where relevant information will be
            extracted from the included studies using a predefined data extraction form (detailed in section \ref{sec:data-extraction-strategy}).
        \end{itemize}
    \item Data synthesis (See Section \TODO{Reference section number}).
\end{itemize}

\subsubsection{Reporting the Review}
The stages associated with reporting the review are:
\begin{itemize}
    \item Specifying dissemination mechanisms (See Section \TODO{Reference section number}).
    \item Formatting the main report (See Section \TODO{Reference section number}).
    \item Evaluating the report (See Section \TODO{Reference section number}).
\end{itemize}

We consider all the above stages to be mandatory except:
\begin{itemize}
    \item Commissioning a review which depends on whether or not the systematic review is being done on a commercial basis.
    \item Evaluating the review protocol and Evaluating the report which are optional and depend on the quality assurance procedures decided by the systematic review team (and any other stakeholders).
\end{itemize}

The stages listed above may appear to be sequential, but it is important to recognise that many of the stages involve iteration. In particular, many activities are initiated during the protocol development stage, and refined when the review proper takes place. For example:
\begin{itemize}
    \item The selection of primary studies is governed by inclusion and exclusion criteria. These criteria are initially specified when the protocol is drafted but may be refined after quality criteria are defined.
    \item Data extraction forms initially prepared during construction of the protocol will be amended when quality criteria are agreed.
    \item Data synthesis methods defined in the protocol may be amended once data has been collected.
\end{itemize}

% ===== STEP 3: Selection of Relevant Studies =====
% This section details:
% - Study 1: Distributed learning techniques
% - Study 2: CUDA implementations
\subsection{Study Selection Framework}
\label{sec:study-selection}

% Combined "Purpose," "Scope," and "Relationship Between Study Types"
This section defines the aims and scope of the review, clarifying the types of studies to be included and the rationale for exploring distributed learning and CUDA implementations together. The specific objectives of the review are aligned with the research questions to provide a clear focus. The boundaries of the review concern the types of distributed learning techniques and CUDA implementations considered, with a time period between 2015 and 2022 to ensure currency. The review includes studies focusing on the design and analysis of distributed learning algorithms and those focusing on CUDA-based parallel implementations to understand the translation of theoretical aspects into practical implementations. This approach allows for the exploration of patterns and challenges in mapping distributed algorithms onto parallel architectures \TODO{Different topics explored together}.

\subsubsection{Distributed Learning Techniques Review}
The review will focus on distributed learning approaches, aligning with "Study 1" in Figure \ref{fig:workflow},
with the following considerations:
\begin{itemize}
    \item Types of algorithms including \textbf{data parallelism, model parallelism, and asynchronous
    Stochastic Gradient Descent (SGD)} \cite{ben-nun_demystifying_2020,langer_distributed_2020}.
    \item Different distributed architectures including parameter servers and peer-to-peer systems \cite{verbraeken_survey_2021,ben-nun_demystifying_2020,langer_distributed_2020}.
    \item Specific machine learning models such as neural networks and support vector machines.
\end{itemize}

\subsubsection{CUDA-based Parallel Implementation Review}
For CUDA implementations, the review will consider aspects relevant to "Study 2" in Figure \ref{fig:workflow}:
\begin{itemize}
    \item Implementation of distributed methods on NVIDIA GPUs using the CUDA framework
    \item Different CUDA libraries and architectures
    \item Specific hardware considerations including GPUs and Tensor Processing Units (TPUs)
\end{itemize}

\subsubsection{Justification for Inclusion}
Both distributed learning techniques and CUDA implementation studies will be included to provide a
complete picture of the current state-of-the-art research in the area. By including both study types,
a deeper understanding of both theoretical approaches and implementation techniques for practical
applications can be reached.

\subsection{Preliminary Protocol Development}
This systematic review follows the guidelines proposed by Kitchenham and Charters for software
engineering research. The preliminary review protocol was developed to establish the foundation for the steps visualized in Figure \ref{fig:workflow}, particularly in the initial stages. An overview of the papers included after the initial selection phase (corresponding to the output of the "Studies Selection" phase in Figure \ref{fig:workflow}) will be presented in Table 2.1.

\subsubsection{Background and Rationale}
This section provides the necessary context for the review, outlining the research gaps that will be
addressed \cite{ben-nun_demystifying_2020}. It explicitly states the need for a systematic review of the current literature to address
this gap and provide a focused analysis.

\subsubsection{Initial Search Strategy}
The initial search strategy involves combining keywords using Boolean and proximity operators to generate search strings, based on the "Goals, expected outputs, constraints, search terms and keywords" documented as an input to Step 1 in Figure \ref{fig:workflow}. Databases like Scopus, Google Scholar, and ACM Digital Library are selected for their coverage of computer science, engineering, and applied mathematics literature. Studies published between 2015-2022 will be considered to ensure recent advancements are included while maintaining a consistent period for analysis.
\begin{itemize}
    \item \textbf{Search Terms:} Details of the search terms will be provided in Section \ref{sec:search_process_documentation}.
    \item \textbf{Database Justification:} Rationale for selecting specific databases is detailed in Section \ref{sec:search_process_documentation}.
    \item \textbf{Timeline:} The timeframe for including studies is 2015-2022.
\end{itemize}

\subsubsection{Preliminary Selection Criteria}
Preliminary criteria for inclusion will use specific examples such as ``studies
that evaluate the performance of synchronous distributed SGD in deep learning models'' rather than
general terms like ``distributed computing'' \cite{ben-nun_demystifying_2020}. Preliminary quality thresholds will ensure only high-quality studies are included in the final analysis. Specific inclusion and exclusion criteria are detailed in Section \ref{sec:study-selection-criteria}.

\subsubsection{Initial Data Extraction Plan}
The following information will be extracted from each study:
\begin{itemize}
    \item Details of distributed systems \cite{ben-nun_demystifying_2020,langer_distributed_2020}:
        \begin{enumerate}
            \item Number of nodes
            \item Communication network
            \item Communication method
            \item Topology
        \end{enumerate}
    \item Machine learning algorithms and models used \cite{xing_strategies_2015}.
    \item Datasets and benchmarks \cite{ben-nun_demystifying_2020}.
    \item Performance metrics (training time, accuracy, speedup) \cite{ben-nun_demystifying_2020,langer_distributed_2020,xing_strategies_2015}.
    \item CUDA implementation details (libraries, optimizations) \cite{verbraeken_survey_2021,ben-nun_demystifying_2020,xing_strategies_2015}.
\end{itemize}
Further details on the data extraction strategy can be found in Section \ref{sec:data-extraction-strategy}.

\subsubsection{Quality Assessment Framework}
Preliminary quality assessment will use specific criteria to evaluate the validity and reliability of methods, using established checklists from the literature \cite{ben-nun_demystifying_2020}. A Likert scale will be used for a standardized approach. Detailed guidelines for reviewers will be established to ensure consistency and prevent bias, as described further in Section \ref{sec:quality-assessment-process}.
\begin{itemize}
    \item \textbf{Criteria:} Specific criteria are detailed in Section \ref{sec:quality-assessment-process}.
    \item \textbf{Scoring System:} A Likert scale will be used.
    \item \textbf{Guidelines:} Guidelines for reviewers are detailed in Section \ref{sec:quality-assessment-process}.
\end{itemize}

\subsubsection{Synthesis Approach}
The synthesis approach will involve meta-analysis where appropriate, using statistical analysis to combine results from included studies with clearly defined methods \cite{ben-nun_demystifying_2020}. Thematic synthesis will be used for narrative synthesis, allowing an in-depth understanding of themes present in selected studies.
\begin{itemize}
    \item \textbf{Meta-analysis:} Details of the methods will be defined later.
    \item \textbf{Narrative synthesis:} Thematic synthesis will be employed.
\end{itemize}

\subsection{Search Process Documentation}
\label{sec:search_process_documentation}
This section provides an overview of our search strategy. Detailed search documentation, including exact search strings and results for each database, can be found in the supplementary materials (Section \ref{sec:search_strategy}).

\subsubsection{Search Strategy Overview}
Our search strategy combines terms from two main categories as shown in Table \ref{tab:search_terms}. The number of articles retrieved from each database is presented in Table \ref{tab:search_results}.

\begin{table*}[h]
    \centering
    \caption{Number of Retrieved Articles by Database}
    \label{tab:search_results}
    \begin{tabular}{|l|c|c|c|}
        \hline
        \textbf{Database} & \textbf{Initial Results} & \textbf{After Filtering} & \textbf{Final Selection} \\
        \hline
        Scopus & XXX & XXX & XXX \\
        \hline
        Google Scholar & XXX & XXX & XXX \\
        \hline
        ACM Digital Library & XXX & XXX & XXX \\
        \hline
        IEEE Xplore & XXX & XXX & XXX \\
        \hline
        Science Direct & XXX & XXX & XXX \\
        \hline
        arXiv & XXX & XXX & XXX \\
        \hline
        \textbf{Total} & XXX & XXX & XXX \\
        \hline
    \end{tabular}
\end{table*}

The complete search strings for each database, including any database-specific adaptations, are documented in Section \ref{sec:search_strategy} of the supplementary materials.

\subsection{Study Selection Criteria}
\label{sec:study-selection-criteria}
This section specifies the detailed inclusion and exclusion criteria for the studies to be included
in the review. These will be specific, measurable, and objective to ensure that all studies are
assessed consistently and fairly \cite{ben-nun_demystifying_2020}.

% TODO: Razvan. Is this fine?
\subsubsection{Inclusion Criteria}
The following criteria will be used for including studies:
\begin{itemize}
    \item Studies published between 2013 and 2023
    \item Peer-reviewed articles and high-quality preprints
    \item Studies focusing on distributed training techniques
    \item Articles written in English
    \item Implementation details available
\end{itemize}
\cite{verbraeken_survey_2021,ben-nun_demystifying_2020}.

\subsubsection{Exclusion Criteria}
Studies will be excluded based on the following criteria:
\begin{itemize}
    \item Studies not focused on neural network training
    \item Pure theoretical papers without implementation
    \item Secondary studies (surveys, reviews)
    \item Insufficient technical details or results
\end{itemize}

\subsection{Quality Assessment Process}
\label{sec:quality-assessment-process}
This part of the methodology details the process used to assess the quality of the selected studies, aligning with the "Validation" phase shown in the bottom section of Figure \ref{fig:workflow}.

\subsubsection{Quality Criteria}
The quality of the studies will be evaluated based on the methodological rigour, clarity of reporting, limitations of the studies, and potential for bias. Established checklists, such as those provided by the CASP, will be used to address bias and validity in a rigorous and systematic way.

\subsection{Data Extraction Strategy}
\label{sec:data-extraction-strategy}
This section details how data will be extracted from the included studies. The data extraction form will be designed to capture all necessary information, including study details, methodology, implementation specifics, dataset details, and results. The form will be piloted to ensure it captures the information effectively \cite{ben-nun_demystifying_2020}.

% \subsubsection{Extraction Form}
% The data extraction form was iteratively refined through:

% TODO Razvan
% \begin{itemize}
%     \item Publication metadata
%     \begin{itemize}
%         \item Authors, venue, year
%         \item Citation count and impact
%     \end{itemize}
%     \item Technical details
%     \begin{itemize}
%         \item Distributed training approach
%         \item Implementation specifications
%         \item Hardware configurations
%     \end{itemize}
%     \item Performance metrics
%     \begin{itemize}
%         \item Training time and convergence
%         \item Resource utilization
%         \item Scalability measures
%     \end{itemize}
%     \item Experimental setup
%     \begin{itemize}
%         \item Dataset characteristics
%         \item Hardware specifications
%         \item Software frameworks used
%     \end{itemize}
% \end{itemize}

\begin{itemize}
    \item \textbf{Details:} The data extraction form will capture study design, participants, interventions, and outcomes, as well as details of the implementation in distributed systems and parallel CUDA frameworks \cite{ben-nun_demystifying_2020}.
    \item \textbf{Piloting:} The extraction form will be piloted to ensure effectiveness \cite{ben-nun_demystifying_2020}.
    \item \textbf{Study Details:} The form will capture relevant details including implementation specifics.
\end{itemize}

% Final paragraph about methodology
By using this methodology, the systematic review will aim to provide a comprehensive and reliable 
analysis of the current state of research in distributed deep learning and CUDA implementations. 
The approach used here will enable the review to identify any trends and gaps in current research 
and make recommendations for future study.


\subsection{Study Selection Results}
\label{sec:study_selection_results}
The complete lists of included and excluded studies, along with detailed information about each study, can be found in the supplementary materials (Section \ref{sec:study_selection}). A total of XXX studies were initially identified, with XXX studies meeting our inclusion criteria after screening. Table \ref{tab:study_types} provides an overview of the types of studies included in our review.

\begin{table}[h]
    \centering
    \caption{Overview of Included Study Types}
    \label{tab:study_types}
    \begin{tabular}{|l|c|p{8cm}|}
        \hline
        \textbf{Study Type} & \textbf{Count} & \textbf{Description} \\
        \hline
        Empirical Studies & XXX & Studies with experimental evaluations of distributed learning techniques \\
        \hline
        Implementation Studies & XXX & Studies focusing on CUDA implementations and optimizations \\
        \hline
        Hybrid Studies & XXX & Studies covering both theoretical and implementation aspects \\
        \hline
        \textbf{Total} & XXX & \\
        \hline
    \end{tabular}
\end{table}

