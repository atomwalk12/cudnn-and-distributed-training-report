% ======== CORE THEME 1: SCALABILITY ========
\section*{Scalability Relationships}
\subsection*{Motivation (MF1)}
\textbf{MF1. Scalability.} The connection is that scalability is a major shared motivating factor for both DNNs and GPU programming.
The increasing scale of data and complexity of DNNs necessitates scalable solutions. GPU programming is
motivated by providing the tools and optimizations needed to achieve this scalability, enabling DNNs to
handle larger workloads, improve productivity, and become more cost-effective.

\subsection*{Critical Factors (CF2)}
\textbf{CF2. Scalability.} Scalability is achieved by implementing a modular programming style. DNN frameworks
abstract away the distributed infrastructure complexity -- by being able to easily select distributed
strategies when executing the code -- while ML frameworks leveraging GPU acceleration to hide
low-level hardware details, allowing developers to focus on the application logic.
% Related to: MF1 (motivation), LF3 (limitation)

\subsection*{Limitations (LF3)}
\textbf{LF3. Communication Overhead and Scalability.} Another limitation is the communication overhead and its impact on scalability. Both areas struggle
with this, leading to performance bottlenecks that are challenging to overcome. There are not universally
optimal solutions, as the best approaches are dependent on model architectures and hardware configurations.
Community involvement is essential for progress as this can promote innovation.
% Emerges from: CF2 implementation challenges

% ======== CORE THEME 2: HARDWARE COMMUNICATION ======== 
\section{Hardware Communication Challenges}
\subsection*{Motivation (MF4)}
\textbf{MF4. Heterogeneous hardware.} While DNNs are motivated to use heterogenous hardware to ensure broader applicability and performance, GPU programming
acknowledge its importance by providing C APIs for CPU-GPU communication. Nonetheless, there do exist limitations
due to latency and sub-optimal bandwidth utilization in both domains. A related concern is that GPU libraries like cuDNN do not provide
integrated support for multi-GPU training, which must be achieved manually by the user. This highlights existing challenges
in both domains.

\subsection*{Critical Factors (CF4)}
\textbf{CF4. Network and hardware communication.} Multi-GPU training is particularly relevant in DNNs. There exist multiple algorithms to minimize
network latency, however GPU programming frameworks still struggle with multi-GPU communication
training, as libraries like cuDNN leave this management to the user. This indicates an ongoing
challenge in this area.
% Direct progression from MF4 motivation

% ======== CORE THEME 3: PERFORMANCE OPTIMIZATION ========
\section{Performance Considerations}
\subsection*{Motivation (MF2)}
\textbf{MF2. Complexity and performance.} The key shared motivator is to manage computational complexity while at the same time ensure higher accuracy
in common applications (i.e. NLP tasks). The increased accuracy is due to the guarantee that performance
is likely to increase thanks to the scaling laws that neural networks exhibit.
GPU programming directly provides the necessary primitives to facilitate the scaling laws and DNNs build on
top of them to improve performance.

\subsection*{Critical Factors (CF3)}
\textbf{CF3. Performance.} Performance is a key concern for both DNNs and GPU programming. DNNs are motivated by the need to
minimize network bandwidth latency and achieve better scalability, while GPU programming provides
optimized primitives by implementing large-matrix operations. To do this, thorough
knowledge of the GPU architecture is necessary.
% Implements MF2 motivations

\subsection*{Evaluation (EM2)}
\textbf{EM2. Model architectures.} In some evaluating scenarios, special care was payed to assess performance across different architectures.
For DNNs, this involves assessing scalability with increasingly complex model architectures. On the other hand,
for GPU programming, evaluation is geared towards optimizing performance by providing efficient primitives
for the most popular operations (convolutions, self-attention, fully connected layers, etc.).
% Assesses CF3 implementations

% ======== CORE THEME 4: COMMUNITY & ECOSYSTEM ========
\section{Community and Tooling Landscape}
\subsection*{Motivation (MF3, MF6, MF7)}
\textbf{MF3. Critical in many domains.}
GPU programming itself might be considered a more specialized domain, but its motivation is connected to the
critical applicability of DNNs across many fields. GPU programming enable DNNs to function efficiently in
these domains by providing the building blocks. A specific example involves Nvidia providing the optimized libraries,
as they can optimize the code better than the general community due to better knowledge of the underlying GPU architecture.

\textbf{MF6. Leveraging existing tools...}
DNN development benefits from open-source frameworks, promoting community-driven innovation. Conversely, GPU programming,
while often proprietary, builds on-top of other low-level libraries like cuBLAS. This shows a reliance on other
proprietary software, which ensures high-level performance. This can stifle innovation in the long term due
to the lack of competition.

\textbf{MF7. Cross-framework use...}
Open-source DNN framework aim for cross-framework compatibility and usability to foster community innovation.
GPU programming libraries show a trade-off between low-level, highly optimized primitives (like cuDNN), and offering
user-friendly interfaces (like CuPy and Torch7). Community involvement is important in both areas, however
less so for GPU programming due to the proprietary nature of the frameworks.

\subsection*{Critical Factors (CF1)}
\textbf{CF1. Paradigms, programming ease...}
DNNs are generally flexible and often use popular interpreted programming languages like Python to
promote ease of use. On the other hand, GPU programming usually relies on C++ and CUDA, which are
critical in areas where speed is a concern. Many GPU programming libraries provide bindings to
popular languages like Python to promote broad user involvement. This shows that community is
important in both areas, however GPU programming must rely on low-level languages to squeeze out
the best performance.
% Connects to MF7's cross-framework focus

% ======== CORE THEME 5: USABILITY & FLEXIBILITY ========
\section{Usability Tradeoffs}
\subsection*{Motivation (MF5)}
\textbf{MF5. Applications...} A shared motivation is to simplify development and improve the practical utility of both DNNs and GPU programming.
Both fields are driven by the need to make life easier for developers to leverage parallel hardware effectively,
and the motivation is to fulfill practical requirements of simplified deployment and reproducible research.

\subsection*{Critical Factors (CF5)}
\textbf{CF5. Ease of use and hardware flexibility...}
The main challenge here resides between ease of use to the developer and hardware flexibility.
Considering the broad community of developers, DNN libraries prioritize modularity and ease of
extension to facilitate broader community involvement. Some GPU programming libraries sacrifice ease of use for lower-level control
and potentially higher performance (cuDNN), while others strive for more user-friendly APIs (CuPy, Caffe).

\subsection*{Limitations (LF1)}
\textbf{LF1. Usability...}
DNN libraries attempt to improve usability through common APIs to facilitate broader
reproducibility. On the other hand, GPU programming frameworks have performance optimizations as
the main objective, which sacrifices ease of use and hides internal implementation details.
To circumvent this, many GPU frameworks provide profiling tools and specialized debuggers to aid
developer productivity.
% Result of CF5 tensions

% ======== CORE THEME 6: DEPLOYMENT & EVALUATION ========
\section{Deployment Strategies}
\subsection*{Evaluation (EM1, EM3, EM4)}
\textbf{EM1. Deployment...}
DNN libraries follow a staged deployment process for safe evaluation (i.e. Google products), which
offers a degree of safety in real world scenarios. Conversely, GPU programming frameworks are
designed with configurable deployment in mind, providing features like compile-time flags to
integrate more easily into ML frameworks.

\textbf{EM3. Task domains...}
Evaluation metrics are strongly dependent on the task domains. DNNs are assessed across
a broad range of deep learning applications, while GPU programming libraries are geared towards
a more narrow field  (like cuDNN), with others being more general purpose (like CuPy in scientific
computing).

\textbf{EM4. Evaluation...}
DNN evaluation focuses on overall performance gains and broad applicability in different domains (vision,
NLP, etc.). Conversely, GPU programming focuses on the potential performance gains achievable through
hardware-specific optimizations. Both fields are continuously refined when new hardware architectures
become available and new deep learning algorithms are developed.

% ======== CORE THEME 7: ALGORITHMIC CHALLENGES ========
\section{Algorithmic Limitations}
\subsection*{Limitations (LF2)}
\textbf{LF2. Algorithmic limitations...}
The primary algorithmic limitation concerns memory management in both domains.
DNNs pose problems related to data and model parallelism, while GPU programming faces issues
related to matrix multiplication algorithms and hyperparameter choices being suboptimal in some edge cases
(i.e. small batch size).
% Connects to CF3 (Performance)

% ======== UPDATED CROSS-THEME CONNECTIONS ========
\begin{itemize}
	\item \textbf{MF1 → CF2 → LF3}: Scalability motivation (MF1) leads to modular implementation approaches (CF2), ultimately resulting in communication limitations (LF3)
	\item \textbf{MF4 → CF4 → LF3}: Hardware heterogeneity motivation leads to multi-GPU challenges (CF4), ultimately creating scalability limitations (LF3)
	\item \textbf{MF2 → CF3 → EM2}: Performance motivation requires architectural knowledge (CF3), evaluated through model-specific metrics (EM2)
	\item \textbf{MF6 → CF1 → LF1}: Tool leverage (MF6) enables programming paradigms (CF1) but creates usability limitations (LF1)
	\item \textbf{MF5 → CF5 → EM1}: Application motivation (MF5) drives flexibility needs (CF5) evaluated through deployment (EM1)
	\item \textbf{MF3 → EM3}: Domain criticality (MF3) influences task-specific evaluation (EM3)
	\item \textbf{CF3 → LF2}: Performance requirements (CF3) reveal algorithmic limitations (LF2)
\end{itemize}

% ======== PRESERVED ORIGINAL STRUCTURE ========
\clearpage
\onecolumn

{\footnotesize
	\begin{longtable}{|l|p{5cm}|p{5cm}|p{5cm}|}
		\caption{Translations of the motivating factors}\label{tab:translations}   \\

		\toprule
		ID & Distributed Neural Networks & GPU Programming & Translation \\
		\midrule
		\endfirsthead

		\multicolumn{4}{c}{Table \thetable{} -- continued from previous page}           \\
		\toprule
		ID & Distributed Neural Networks & GPU Programming & Translation \\
		\midrule
		\endhead
		\midrule
		MF1
		   & \textbullet\ Google internally requires their deep learning frameworks to be scalable. \cellref{D101} \newline
             \textbullet\ Internally, other organizations (i.e. Facebook) become more and more reliant on neural networks. \cellref{D106}
           & \textbullet\ Optimizing kernels is difficult and time-consuming. \cellref{G1011}              
           & \uline{\textbf{Scalability}}\newline 
           %Internal need for scalability
           There is a surging \textbf{need for scalability}, likely due to the increasingly abundant \textbf{data availability} which time consuming to process. 
             The resulting reliance of neural networks has \textbf{increased productivity} and \textbf{reduced costs}.            \\
           \midrule
		   MF2 
           & \textbullet\ The trend to scale up datasets/computational resources is successful in competitions like ImageNet. \cellref{D102}, \cellref{D105}, \cellref{D103}
            \newline
            \textbullet\ The abundance of computation and data are particularly effective in Natural Language Processing (NLP) tasks. \cellref{D111}
           & \textbullet\ Natural parallelizability of deep learning techniques enables training higher capacity networks on larger datasets. \cellref{G1012} \newline
             \textbullet\ Early open-source GPU implementations of CNNs set precedent for code sharing. \cellref{G1051}
           & \uline{\textbf{Complexity and performance}}\newline 
           The GPU programming field enabled DNNs through \textbf{easy parallelization} of training. Larger networks consistently provide better \textbf{performance}, leading to widespread adoption, especially in \textbf{NLP tasks}. Open-source implementations have accelerated progress. \\
           \midrule
		   MF3 
           &
             \textbullet\ The deep learning applications are critical in many domains. \cellref{D103}, \cellref{D105} \newline
             \textbullet\ Frameworks have been extended reinforcement learning \cellref{D208}

           & \textbullet\ Deep learning frameworks (Caffe and PADDLE) rely on GPU programming libraries such as cuDNN. \cellref{G1014} \newline
             \textbullet\ As architectures evolve, underlying code needs to be re-optimized. This is standardized by NVidia as they understand better how the GPU architecture works.\cellref{G1013}
           & \uline{\textbf{Critical in many domains}}\newline 
           \textbullet\ GPU programming libraries are used in a more narrow domain, however DNNs have broader applicability in areas such as reinforcement learning. 
           \newline
           \textbullet\ NVidia understands well the GPU architecture and can provide better optimizations in critical areas.\\

           \midrule
		   MF4 
           & \textbullet\ Data centers are inherently homogenous. BytePS can leverage spare CPU and bandwidth resources to accelerate distributed training running on GPUs. \cellref{D104} \newline
             \textbullet\ Modern systems can leverage mobile devices, tablets, and thousands of GPU cards. \cellref{D201}
           & GPU programming libraries expose a C language API to communicate with the host CPU. \cellref{G1015}
           & \uline{\textbf{Heterogenous hardware}}\newline 
           There is limited support for CPU-GPU interaction in GPU programming libraries. However, heterogeneous hardware plays a more important role in DNNs as the ability to fully utilize available resources is critical. \\
 
           \midrule
		   MF5
           & DDNs have powered a wide range of applications including image recognition, language translation, anomaly detection, and more. \cellref{D106} \newline
             \textbullet\ Replication of published results can involve months of work by researchers. \cellref{G1041}
           & GPU libraries meet user's needs by reducing the need to write custom code, allowing developers to focus on higher-level issues, improved portability. \cellref{G1016} \newline
             \textbullet\ Few toolboxes offer truly off-the-shelf deployment of state-of-the-art models that are computationally efficient. \cellref{G1041}
           & \uline{\textbf{Requirements and applications}}\newline 
           Both topics aim to make it easier for developers to take advantage of parallel hardware. GPU programming facilitates the development of new architectures and DNNs scale models to achieve better accuracy. The need for efficient deployment and replication of research results drives development in both areas. \\

           \midrule
		   MF6
           & Colossal-AI builds on existing open-source frameworks such as PipeDream, GPipe and Chimera. \cellref{D107}, \cellref{D207}
           & \textbullet\ CuDNN relies on the CUDA toolkit, specifically the cuBLAS library. \cellref{G1016} \newline \textbullet\ CuPy is specifically designed to work with NVidia GPUs. \cellref{G1062}
           & \uline{\textbf{Leverage existing frameworks}}\newline 
           Since DNN frameworks are generally open-source, this encourages community involvement, which leads to innovation. \newline 
           GPU programming libraries are proprietary. Nonetheless, internally libraries such as cuDNN rely on the CUDA toolkit. \\

           \midrule
		   MF7
           & \textbullet\ Inter-GPU communication frameworks require minimal code changes and support multiple frontends. \cellref{D110}, \cellref{D112} \newline
            \textbullet\ Libraries can generally be integrated into existing frontend frameworks (i.e. PyTorch). \cellref{D211}
           & \textbullet\ GPU programming libraries provide lower-level primitives and are generally self-contained. \cellref{G1017} \newline
             \textbullet\ Libraries emphasize compatibility (CuPy with NumPy) or ease of development (Torch7). \cellref{G1062}, \cellref{G1071} \newline
             \textbullet\ Research-focused libraries prioritize configurability and flexibility. \cellref{G1031}
           & \uline{\textbf{Cross-framework use}}\newline 
           Open-source DNN libraries promote usability and cross-framework compatibility, fostering innovation. GPU libraries vary between low-level primitives (cuDNN) and user-friendly interfaces (CuPy, Torch7), though being closed-source limits community-driven innovation. This highlights the importance of code sharing for research. \\

           % recent advancements in deep learning
           % rapidly evolving field

        

		\bottomrule
	\end{longtable}
}

\twocolumn

% DONE D101 internal need for scalability
% DONE D102 increasingly complex datasets (Done also from D105 and D111), Improved performance
% DONE D103 critical in many domains (emerging applications)
% DOING D104 utilization of heterogeneous hardware
% D105
% howpublished = \{(https?://.*?)\}
% howpublished = {\url{$1}}
% url = \{(https?://.*?)\}
% howpublished = {\url{$1}}
% DONE D201 utilization of heterogenous hardware


% Not taken into account:
% D108
\clearpage
\onecolumn

{\footnotesize
	\begin{longtable}{|l|p{5cm}|p{5cm}|p{5cm}|}
		\caption{Translations of the critical factors}\label{tab:translations_critical_factors}   \\

		\toprule
		ID & Distributed Neural Networks & GPU Programming & Translation \\
		\midrule
		\endfirsthead

		\multicolumn{4}{c}{Table \thetable{} -- continued from previous page}           \\
		\toprule
		ID & Distributed Neural Networks & GPU Programming & Translation \\
		\midrule
		\endhead
		\midrule
		CF1
		   & Generally DNNs are most flexible and use the most common programming style supported by the host language. \cellref{D202}, \cellref{D205}
           & \textbullet\ Although most GPU frameworks work using the host language as C++, there exist frontend frameworks that enable users to use Python. \cellref{G2061} \newline
             \textbullet\ Cudnn exposes a C API \cellref{G1015} \newline
             \textbullet\ Torch7 is designed for ease of programming through Lua \cellref{G2021} \newline
             \textbullet\ CuPy implements NumPy-like API \cellref{G2061}
           & \uline{\textbf{Paradigms, programming ease.}}\newline
           Imperative and declarative programming styles are both supported. DNNs use Python as the most popular host language, while GPU frameworks generally work with C++ and CUDA. 
           Nonetheless, there do exist frontend frameworks that enable users to write code in higher-level languages (i.e. Lua, Python), and compatibility libraries (NumPy-like API) to improve accessability.\\
           \midrule

    CF2
    & \textbullet\ Distributing neural network layers across multiple GPUs is architecture-specific. \cellref{D203} \newline
      \textbullet\ Scaling is expensive in terms of cost, time and code integration. \cellref{D209} \newline
      \textbullet\ There is a separation of concerns between GPU programming and DNNs, as there is no need for custom C++ code or compiler required to distribute neural networks over cluster nodes. \cellref{D211}
        & \textbullet\ Optimized code using NVIDIA GPUs ensures high performance (freeing up auxiliary memory) \cellref{G2011} \newline
          \textbullet\ CuDNN provides separation of concerns by enabling developers to focus on higher-level optimizations instead of low-level architecture specific code \cellref{G2012} \newline
          \textbullet\ Caffe implements separation of representation and implementation \cellref{G2041}
        & \uline{\textbf{Scalability, separation of concerns.}}\newline
        \textbullet\ Scalability challenges are related to distributing parts of the network or dataset across multiple nodes. Frameworks emphasize separation of concerns, allowing developers to focus on higher-level optimizations while library providers handle hardware-specific optimizations.\\
        \midrule

    CF3
    & \textbullet\ Specialized techniques for distributed training include: bucketing, overlapping communication with computation, etc. \cellref{D206} \newline
      \textbullet\ Megatron-LM extends optimization techniques to the transformer model. \cellref{D211}
        & \textbullet\ Libraries optimize for wide range of use cases \cellref{G2011} \newline
          \textbullet\ Frameworks like Torch7 leverage SSE and support multiple parallelization methods \cellref{G2021}
        & \uline{\textbf{Performance optimization.}}\newline
        Both domains focus heavily on performance optimization through various techniques. DNNs use specialized distributed training techniques, while GPU frameworks optimize for different architectures and use cases through various parallelization methods.\\
        \midrule
    
    CF4
    & There exist algorithms that can optimize network latency. \cellref{D210} \cellref{D204}
        & \textbullet\ Multi-GPU training is an outstanding challenge \cellref{G4011} \newline
          \textbullet\ GPUs can read/write directly to each other's memory \cellref{G2051} \newline
          \textbullet\ Inter-GPU communication is optimized for specific layers \cellref{G2051}
        & \uline{\textbf{Network and hardware communication.}}\newline
        While optimal algorithms exist for network latency optimization, multi-GPU training presents ongoing challenges. AlexNet optimized inter-GPU communication through direct memory access and selective layer communication. The CuDNN leaves multi-GPU communication to the user.\\
        \midrule

    CF5
    & The Transformers library provides modular components that greatly simplify the extension and ease of use of the library \cellref{D212}
        & \textbullet\ CuPy is NumPy compatible \cellref{G1062} \newline
          \textbullet\ CuDNN requires more specialized C and CUDA knowledge \cellref{G1015} \newline
          \textbullet\ Caffe provides easy CPU/GPU switching and clean Python/MATLAB bindings \cellref{G2041}
        & \uline{\textbf{Ease of use and hardware flexibility.}}\newline
          \textbullet\ DNN libraries emphasize modularity and ease of extension. \newline
          \textbullet\ GPU frameworks vary in accessibility - some require specialized knowledge while others provide familiar APIs and easy hardware switching capabilities.\\
        \midrule

		\bottomrule
	\end{longtable}
}

\twocolumn


\clearpage
\onecolumn

{\footnotesize
	\begin{longtable}{|l|p{5cm}|p{5cm}|p{5cm}|}
		\caption{Translations of the evaluation metrics}\label{tab:translations_evaluation_metrics}   \\

		\toprule
		\textbf{ID} & \textbf{Distributed Neural Networks} & \textbf{GPU Programming} & \textbf{Translation} \\
		\midrule
		\endfirsthead

		\multicolumn{4}{c}{Table \thetable{} -- continued from previous page}           \\
		\toprule
		\textbf{ID} & \textbf{Distributed Neural Networks} & \textbf{GPU Programming} & \textbf{Translation} \\
		\midrule
		\endhead
		\midrule
    EM1
        & \textbullet\ Evaluation was initially performed behind closed doors for internal processes (speech recognition systems) and subsequently for external applications (Google Search). \cellref{D301}
        & \textbullet\ In many frameworks, the GPU libraries can be switched on and off at compile time using a single flag. \cellref{G1014} \newline
          \textbullet\ Some are designed for both research and industry, run on both CPU and GPU, have bindings for both Python and Matlab. For model architecture portability, Protocol Buffer files are used. \cellref{G3041}
        & \uline{\textbf{Deployment:}} \newline
          \textbullet\ Large companies employ staged deployment: first testing internally before external applications, enabling safe evaluation. \newline
          \textbullet\ Framework designers can rollback through compile-time flags. CUDA code written in C, with Python and Matlab bindings ensure portability.
        \\
        \midrule

    EM2
        & \textbullet\ The evaluation was done by scaling complex networks -- based on Mixture of Experts -- to 600B parameters using automatic sharding. \cellref{D305}
        & \textbullet\ The cuDNN library is assessed by measuring time and memory usage for convolutional layers. 
        Mini-batch performance is also assessed.
        Scalability is not as much of a concern as different GPU architectures are benchmarked instead of assessing performance across GPU clusters. \cellref{G3011}
        & \uline{\textbf{Model Architectures:}} \newline
          \textbullet\ Scalability testing is a concern for DNNs, as these are the type of problems that are encountered in the real world. \newline
          \textbullet\ However, for GPU programming, performance is measured by optimizing resource usage on a single GPU. 
          The difficulty stands in optimizing performance as new NN architectures are developed.
        \\
        \midrule

    EM3
        & \textbullet\ Evaluation tasks include: image classification \cellref{D303}, machine translation \cellref{D303}, \cellref{D305}, 
          NLP \cellref{D306} \cellref{D311}, RL \cellref{D308}.\newline
          \textbullet\ MoE models can be scaled up to 600 billion parameters for machine translation.
        & \textbullet\ CuDNN can be used in deep learning, CNNs, speech and language. \cellref{G3012} \newline
          \textbullet\ CuPy can be extended to scientific computing and probabilistic modelling. \cellref{G3061}
        & \uline{\textbf{Task Domains:}} \newline
          \textbullet\ DNN libraries focus on deep learning tasks. \newline
          \textbullet\ CuDNN was designed for deep learning. CuPy can be used in broader domains.
        \\
        \midrule

    EM4
        & \textbullet\ Impressive improvements in performance over older methods. \cellref{D304} \newline
          \textbullet\ Evaluation is done against vision and NLP tasks \cellref{D306} and RL \cellref{D308}. \newline
          \textbullet\ Wikipedia dataset is often used for evaluation. \cellref{D307} \newline
          \textbullet\ Scaling gives consistent improvements in performance. \cellref{D311}
        & \textbullet\ Evaluation metric involves assessing performance and matrix multiplication. Speedup reaches up to 36\% improvements. \cellref{G3013} \newline
          \textbullet\ CuDNN is assessed against libraries like cuda-convnet2 and Caffe. Achieves portability across GPU architectures. \cellref{G3013} \newline
          \textbullet\ Qualitative evaluation can offer valuable insights into performance. \cellref{G3051}
        & \uline{\textbf{Evaluation:}} \newline
          \textbullet\ Potential for gains through hardware-specific optimizations. \newline
          \textbullet\ Broad applicability of DNNs, while GPU programming focuses on specific architectures. \newline
          \textbullet\ Consistent scaling suggests promising future for DNNs, while GPU advances focus on specific implementations.
        \\
        \midrule

    LF1
      & \textbullet\ To address usability, many libraries provide common APIs with other frameworks. \cellref{D402} \newline
        \textbullet\ Training DNNs require special algorithms that are often architecture-specific. \cellref{D403} \newline
        \textbullet\ Optimizations are challenging and error prone, requirement intimate knowledge of the network architecture. \cellref{D411} \newline
        \textbullet\ Reinforcement learning libraries have bindings that allow both task-parallel and actor-based parallelism. \cellref{D408}
      & \textbullet\ Replication of results is challenging (can take months of work). \cellref{G4041} \newline
        \textbullet\ The requirement to manually fine-tune architectures is time-consuming and requires deep knowledge of the GPU architecture. \cellref{G4012} \newline
        \textbullet\ The memory profiles are used to assess performance. \cellref{G4012}
      & \uline{\textbf{Usability:}} \newline
      \textbullet\ To address the replication of SOTA results common APIs are used (Transformers~\cite{wolf_huggingfaces_2020} addresses this issue as it interfaces with \cite{falcon_pytorch_2019}). \newline
      \textbullet\ Being closed source, open-source frameworks cannot reliably match NVidia SOTA performance, as they have better knowledge of the GPU architecture. \newline
      \textbullet\ Profiles are key to efficient debugging and optimization in both cases.\\
      \midrule

    LF2
        & \textbullet\ No single algorithm that can perform optimally across all cases. \cellref{D406} \newline
          \textbullet\ Communication overhead leads to resource under-utilization and solutions do not transfer across architectures. \cellref{D403}, \cellref{D405} \newline
          \textbullet\ Data parallelism models are designed for homogeneous setups. \cellref{D404}
        & \textbullet\ Main issues relate to memory management around matrix multiplication algorithms. Problems also relate to hyperparameter choice, as some stride sizes perform sub-optimally. \cellref{G4013}
        & \uline{\textbf{Algorithmic Limitations:}} \newline
          \textbullet\ No algorithms perform optimally across all cases. \newline
          \textbullet\ Memory optimizations remain a challenge for both GPU programming and DNNs.
        \\
        \midrule
        
    LF3
        & \textbullet\ Tensorflow performs node placement and communication management which results in overhead. \cellref{D401} \newline
          \textbullet\ Deepspeed incurs communication overhead by allocating data to CPU memory. \cellref{D407} \newline
          \textbullet\ Some papers emphasize collaboration in the research community to ensure innovation. \cellref{D410}
        & \textbullet\ Sophisticated techniques to manage communication overhead by not updating parameters across GPUs on each layer. \cellref{G4051} \newline
          \textbullet\ The cost of transferring data to the GPU outweighs the benefits of using a GPU. \cellref{G4061} \newline
          \textbullet\ The very existence of CuDNN implies that cross-GPU programming is challenging which requires thorough understanding of the GPU architecture. \cellref{G4012}, \cellref{G4011}
        & \uline{\textbf{Communication Overhead \& Scalability:}} \newline
          \textbullet\ Tradeoffs between communication overhead and performance. \newline
          \textbullet\ Both GPUs and DNNs face similar bottlenecks in terms of memory allocation that impacts performance. \newline
          \textbullet\ There are no simple universal solutions and choosing the right approach depends on model architectures and hardware. \newline
          \textbullet\ Community is key to success for DNNs.
        \\
        \midrule


        
		\bottomrule
	\end{longtable}
}

\twocolumn






% ======== ORIGINAL STRUCTURE ========
\subsection{M.6 -- Relationship between concepts}

\textbf{MF1. Scalability.}
The connection is that scalability is a major shared motivating factor for both DNNs and GPU programming.
The increasing scale of data and complexity of DNNs necessitates scalable solutions. GPU programming is
motivated by providing the tools and optimizations needed to achieve this scalability, enabling DNNs to
handle larger workloads, improve productivity, and become more cost-effective.

\textbf{MF2. Complexity and performance.}
The key shared motivator is to manage computational complexity while at the same time ensure higher accuracy
in common applications (i.e. NLP tasks). The increased accuracy is due to the guarantee that performance
is likely to increase thanks to the scaling laws that neural networks exhibit.
GPU programming directly provides the necessary primitives to facilitate the scaling laws and DNNs build on 
top of them to improve performance.

\textbf{MF3. Critical in many domains.}
GPU programming itself might be considered a more specialized domain, but its motivation is connected to the
critical applicability of DNNs across many fields. GPU programming enable DNNs to function efficiently in
these domains by providing the building blocks. A specific example involves Nvidia providing the optimized libraries,
as they can optimize the code better than the general community due to better knowledge of the underlying GPU architecture.

\textbf{MF4. Heterogeneous hardware.}
While DNNs are motivated to use heterogenous hardware to ensure broader applicability and performance, GPU programming
acknowledge its importance by providing C APIs for CPU-GPU communication. Nonetheless, there do exist limitations
due to latency and sub-optimal bandwidth utilization in both domains. A related concern is that GPU libraries like cuDNN do not provide
integrated support for multi-GPU training, which must be achieved manually by the user. This highlights existing challenges
in both domains.

\textbf{MF5. Applications.}
A shared motivation is to simplify development and improve the practical utility of both DNNs and GPU programming.
Both fields are driven by the need to make life easier for developers to leverage parallel hardware effectively,
and the motivation is to fulfill practical requirements of simplified deployment and reproducible research.

\textbf{MF6. Leveraging existing tools.}
DNN development benefits from open-source frameworks, promoting community-driven innovation. Conversely, GPU programming,
while often proprietary, builds on-top of other low-level libraries like cuBLAS. This shows a reliance on other
proprietary software, which ensures high-level performance. This can stifle innovation in the long term due
to the lack of competition.

\textbf{MF7. Cross-framework use.}
Open-source DNN framework aim for cross-framework compatibility and usability to foster community innovation.
GPU programming libraries show a trade-off between low-level, highly optimized primitives (like cuDNN), and offering
user-friendly interfaces (like CuPy and Torch7). Community involvement is important in both areas, however
less so for GPU programming due to the proprietary nature of the frameworks.

\paragraph{CF1. Paradigms, programming ease.}
DNNs are generally flexible and often use popular interpreted programming languages like Python to
promote ease of use. On the other hand, GPU programming usually relies on C++ and CUDA, which are
critical in areas where speed is a concern. Many GPU programming libraries provide bindings to
popular languages like Python to promote broad user involvement. This shows that community is
important in both areas, however GPU programming must rely on low-level languages to squeeze out
the best performance.

\textbf{CF2. Scalability.}
Scalability is achieved by implementing a modular programming style. DNN frameworks
abstract away the distributed infrastructure complexity -- by being able to easily select distributed
strategies when executing the code -- while ML frameworks leveraging GPU acceleration to hide
low-level hardware details, allowing developers to focus on the application logic.

\clearpage
\onecolumn

{\footnotesize
	\begin{longtable}{|l|p{5cm}|p{5cm}|p{5cm}|}
		\caption{Translations of the motivating factors}\label{tab:translations}   \\

		\toprule
		ID & Distributed Neural Networks & GPU Programming & Translation \\
		\midrule
		\endfirsthead

		\multicolumn{4}{c}{Table \thetable{} -- continued from previous page}           \\
		\toprule
		ID & Distributed Neural Networks & GPU Programming & Translation \\
		\midrule
		\endhead
		\midrule
		MF1
		   & \textbullet\ Google internally requires their deep learning frameworks to be scalable. \cellref{D101} \newline
             \textbullet\ Internally, other organizations (i.e. Facebook) become more and more reliant on neural networks. \cellref{D106}
           & \textbullet\ Optimizing kernels is difficult and time-consuming. \cellref{G1011}              
           & \uline{\textbf{Scalability}}\newline 
           %Internal need for scalability
           There is a surging \textbf{need for scalability}, likely due to the increasingly abundant \textbf{data availability} which time consuming to process. 
             The resulting reliance of neural networks has \textbf{increased productivity} and \textbf{reduced costs}.            \\
           \midrule
		   MF2 
           & \textbullet\ The trend to scale up datasets/computational resources is successful in competitions like ImageNet. \cellref{D102}, \cellref{D105}, \cellref{D103}
            \newline
            \textbullet\ The abundance of computation and data are particularly effective in Natural Language Processing (NLP) tasks. \cellref{D111}
           & \textbullet\ Natural parallelizability of deep learning techniques enables training higher capacity networks on larger datasets. \cellref{G1012} \newline
             \textbullet\ Early open-source GPU implementations of CNNs set precedent for code sharing. \cellref{G1051}
           & \uline{\textbf{Complexity and performance}}\newline 
           The GPU programming field enabled DNNs through \textbf{easy parallelization} of training. Larger networks consistently provide better \textbf{performance}, leading to widespread adoption, especially in \textbf{NLP tasks}. Open-source implementations have accelerated progress. \\
           \midrule
		   MF3 
           &
             \textbullet\ The deep learning applications are critical in many domains. \cellref{D103}, \cellref{D105} \newline
             \textbullet\ Frameworks have been extended reinforcement learning \cellref{D208}

           & \textbullet\ Deep learning frameworks (Caffe and PADDLE) rely on GPU programming libraries such as cuDNN. \cellref{G1014} \newline
             \textbullet\ As architectures evolve, underlying code needs to be re-optimized. This is standardized by NVidia as they understand better how the GPU architecture works.\cellref{G1013}
           & \uline{\textbf{Critical in many domains}}\newline 
           \textbullet\ GPU programming libraries are used in a more narrow domain, however DNNs have broader applicability in areas such as reinforcement learning. 
           \newline
           \textbullet\ NVidia understands well the GPU architecture and can provide better optimizations in critical areas.\\

           \midrule
		   MF4 
           & \textbullet\ Data centers are inherently homogenous. BytePS can leverage spare CPU and bandwidth resources to accelerate distributed training running on GPUs. \cellref{D104} \newline
             \textbullet\ Modern systems can leverage mobile devices, tablets, and thousands of GPU cards. \cellref{D201}
           & GPU programming libraries expose a C language API to communicate with the host CPU. \cellref{G1015}
           & \uline{\textbf{Heterogenous hardware}}\newline 
           There is limited support for CPU-GPU interaction in GPU programming libraries. However, heterogeneous hardware plays a more important role in DNNs as the ability to fully utilize available resources is critical. \\
 
           \midrule
		   MF5
           & DDNs have powered a wide range of applications including image recognition, language translation, anomaly detection, and more. \cellref{D106} \newline
             \textbullet\ Replication of published results can involve months of work by researchers. \cellref{G1041}
           & GPU libraries meet user's needs by reducing the need to write custom code, allowing developers to focus on higher-level issues, improved portability. \cellref{G1016} \newline
             \textbullet\ Few toolboxes offer truly off-the-shelf deployment of state-of-the-art models that are computationally efficient. \cellref{G1041}
           & \uline{\textbf{Requirements and applications}}\newline 
           Both topics aim to make it easier for developers to take advantage of parallel hardware. GPU programming facilitates the development of new architectures and DNNs scale models to achieve better accuracy. The need for efficient deployment and replication of research results drives development in both areas. \\

           \midrule
		   MF6
           & Colossal-AI builds on existing open-source frameworks such as PipeDream, GPipe and Chimera. \cellref{D107}, \cellref{D207}
           & \textbullet\ CuDNN relies on the CUDA toolkit, specifically the cuBLAS library. \cellref{G1016} \newline \textbullet\ CuPy is specifically designed to work with NVidia GPUs. \cellref{G1062}
           & \uline{\textbf{Leverage existing frameworks}}\newline 
           Since DNN frameworks are generally open-source, this encourages community involvement, which leads to innovation. \newline 
           GPU programming libraries are proprietary. Nonetheless, internally libraries such as cuDNN rely on the CUDA toolkit. \\

           \midrule
		   MF7
           & \textbullet\ Inter-GPU communication frameworks require minimal code changes and support multiple frontends. \cellref{D110}, \cellref{D112} \newline
            \textbullet\ Libraries can generally be integrated into existing frontend frameworks (i.e. PyTorch). \cellref{D211}
           & \textbullet\ GPU programming libraries provide lower-level primitives and are generally self-contained. \cellref{G1017} \newline
             \textbullet\ Libraries emphasize compatibility (CuPy with NumPy) or ease of development (Torch7). \cellref{G1062}, \cellref{G1071} \newline
             \textbullet\ Research-focused libraries prioritize configurability and flexibility. \cellref{G1031}
           & \uline{\textbf{Cross-framework use}}\newline 
           Open-source DNN libraries promote usability and cross-framework compatibility, fostering innovation. GPU libraries vary between low-level primitives (cuDNN) and user-friendly interfaces (CuPy, Torch7), though being closed-source limits community-driven innovation. This highlights the importance of code sharing for research. \\

           % recent advancements in deep learning
           % rapidly evolving field

        

		\bottomrule
	\end{longtable}
}

\twocolumn

% DONE D101 internal need for scalability
% DONE D102 increasingly complex datasets (Done also from D105 and D111), Improved performance
% DONE D103 critical in many domains (emerging applications)
% DOING D104 utilization of heterogeneous hardware
% D105
% howpublished = \{(https?://.*?)\}
% howpublished = {\url{$1}}
% url = \{(https?://.*?)\}
% howpublished = {\url{$1}}
% DONE D201 utilization of heterogenous hardware


% Not taken into account:
% D108
\clearpage
\onecolumn

{\footnotesize
	\begin{longtable}{|l|p{5cm}|p{5cm}|p{5cm}|}
		\caption{Translations of the critical factors}\label{tab:translations_critical_factors}   \\

		\toprule
		ID & Distributed Neural Networks & GPU Programming & Translation \\
		\midrule
		\endfirsthead

		\multicolumn{4}{c}{Table \thetable{} -- continued from previous page}           \\
		\toprule
		ID & Distributed Neural Networks & GPU Programming & Translation \\
		\midrule
		\endhead
		\midrule
		CF1
		   & Generally DNNs are most flexible and use the most common programming style supported by the host language. \cellref{D202}, \cellref{D205}
           & \textbullet\ Although most GPU frameworks work using the host language as C++, there exist frontend frameworks that enable users to use Python. \cellref{G2061} \newline
             \textbullet\ Cudnn exposes a C API \cellref{G1015} \newline
             \textbullet\ Torch7 is designed for ease of programming through Lua \cellref{G2021} \newline
             \textbullet\ CuPy implements NumPy-like API \cellref{G2061}
           & \uline{\textbf{Paradigms, programming ease.}}\newline
           Imperative and declarative programming styles are both supported. DNNs use Python as the most popular host language, while GPU frameworks generally work with C++ and CUDA. 
           Nonetheless, there do exist frontend frameworks that enable users to write code in higher-level languages (i.e. Lua, Python), and compatibility libraries (NumPy-like API) to improve accessability.\\
           \midrule

    CF2
    & \textbullet\ Distributing neural network layers across multiple GPUs is architecture-specific. \cellref{D203} \newline
      \textbullet\ Scaling is expensive in terms of cost, time and code integration. \cellref{D209} \newline
      \textbullet\ There is a separation of concerns between GPU programming and DNNs, as there is no need for custom C++ code or compiler required to distribute neural networks over cluster nodes. \cellref{D211}
        & \textbullet\ Optimized code using NVIDIA GPUs ensures high performance (freeing up auxiliary memory) \cellref{G2011} \newline
          \textbullet\ CuDNN provides separation of concerns by enabling developers to focus on higher-level optimizations instead of low-level architecture specific code \cellref{G2012} \newline
          \textbullet\ Caffe implements separation of representation and implementation \cellref{G2041}
        & \uline{\textbf{Scalability, separation of concerns.}}\newline
        \textbullet\ Scalability challenges are related to distributing parts of the network or dataset across multiple nodes. Frameworks emphasize separation of concerns, allowing developers to focus on higher-level optimizations while library providers handle hardware-specific optimizations.\\
        \midrule

    CF3
    & \textbullet\ Specialized techniques for distributed training include: bucketing, overlapping communication with computation, etc. \cellref{D206} \newline
      \textbullet\ Megatron-LM extends optimization techniques to the transformer model. \cellref{D211}
        & \textbullet\ Libraries optimize for wide range of use cases \cellref{G2011} \newline
          \textbullet\ Frameworks like Torch7 leverage SSE and support multiple parallelization methods \cellref{G2021}
        & \uline{\textbf{Performance optimization.}}\newline
        Both domains focus heavily on performance optimization through various techniques. DNNs use specialized distributed training techniques, while GPU frameworks optimize for different architectures and use cases through various parallelization methods.\\
        \midrule
    
    CF4
    & There exist algorithms that can optimize network latency. \cellref{D210} \cellref{D204}
        & \textbullet\ Multi-GPU training is an outstanding challenge \cellref{G4011} \newline
          \textbullet\ GPUs can read/write directly to each other's memory \cellref{G2051} \newline
          \textbullet\ Inter-GPU communication is optimized for specific layers \cellref{G2051}
        & \uline{\textbf{Network and hardware communication.}}\newline
        While optimal algorithms exist for network latency optimization, multi-GPU training presents ongoing challenges. AlexNet optimized inter-GPU communication through direct memory access and selective layer communication. The CuDNN leaves multi-GPU communication to the user.\\
        \midrule

    CF5
    & The Transformers library provides modular components that greatly simplify the extension and ease of use of the library \cellref{D212}
        & \textbullet\ CuPy is NumPy compatible \cellref{G1062} \newline
          \textbullet\ CuDNN requires more specialized C and CUDA knowledge \cellref{G1015} \newline
          \textbullet\ Caffe provides easy CPU/GPU switching and clean Python/MATLAB bindings \cellref{G2041}
        & \uline{\textbf{Ease of use and hardware flexibility.}}\newline
          \textbullet\ DNN libraries emphasize modularity and ease of extension. \newline
          \textbullet\ GPU frameworks vary in accessibility - some require specialized knowledge while others provide familiar APIs and easy hardware switching capabilities.\\
        \midrule

		\bottomrule
	\end{longtable}
}

\twocolumn


\clearpage
\onecolumn

{\footnotesize
	\begin{longtable}{|l|p{5cm}|p{5cm}|p{5cm}|}
		\caption{Translations of the evaluation metrics}\label{tab:translations_evaluation_metrics}   \\

		\toprule
		\textbf{ID} & \textbf{Distributed Neural Networks} & \textbf{GPU Programming} & \textbf{Translation} \\
		\midrule
		\endfirsthead

		\multicolumn{4}{c}{Table \thetable{} -- continued from previous page}           \\
		\toprule
		\textbf{ID} & \textbf{Distributed Neural Networks} & \textbf{GPU Programming} & \textbf{Translation} \\
		\midrule
		\endhead
		\midrule
    EM1
        & \textbullet\ Evaluation was initially performed behind closed doors for internal processes (speech recognition systems) and subsequently for external applications (Google Search). \cellref{D301}
        & \textbullet\ In many frameworks, the GPU libraries can be switched on and off at compile time using a single flag. \cellref{G1014} \newline
          \textbullet\ Some are designed for both research and industry, run on both CPU and GPU, have bindings for both Python and Matlab. For model architecture portability, Protocol Buffer files are used. \cellref{G3041}
        & \uline{\textbf{Deployment:}} \newline
          \textbullet\ Large companies employ staged deployment: first testing internally before external applications, enabling safe evaluation. \newline
          \textbullet\ Framework designers can rollback through compile-time flags. CUDA code written in C, with Python and Matlab bindings ensure portability.
        \\
        \midrule

    EM2
        & \textbullet\ The evaluation was done by scaling complex networks -- based on Mixture of Experts -- to 600B parameters using automatic sharding. \cellref{D305}
        & \textbullet\ The cuDNN library is assessed by measuring time and memory usage for convolutional layers. 
        Mini-batch performance is also assessed.
        Scalability is not as much of a concern as different GPU architectures are benchmarked instead of assessing performance across GPU clusters. \cellref{G3011}
        & \uline{\textbf{Model Architectures:}} \newline
          \textbullet\ Scalability testing is a concern for DNNs, as these are the type of problems that are encountered in the real world. \newline
          \textbullet\ However, for GPU programming, performance is measured by optimizing resource usage on a single GPU. 
          The difficulty stands in optimizing performance as new NN architectures are developed.
        \\
        \midrule

    EM3
        & \textbullet\ Evaluation tasks include: image classification \cellref{D303}, machine translation \cellref{D303}, \cellref{D305}, 
          NLP \cellref{D306} \cellref{D311}, RL \cellref{D308}.\newline
          \textbullet\ MoE models can be scaled up to 600 billion parameters for machine translation.
        & \textbullet\ CuDNN can be used in deep learning, CNNs, speech and language. \cellref{G3012} \newline
          \textbullet\ CuPy can be extended to scientific computing and probabilistic modelling. \cellref{G3061}
        & \uline{\textbf{Task Domains:}} \newline
          \textbullet\ DNN libraries focus on deep learning tasks. \newline
          \textbullet\ CuDNN was designed for deep learning. CuPy can be used in broader domains.
        \\
        \midrule

    EM4
        & \textbullet\ Impressive improvements in performance over older methods. \cellref{D304} \newline
          \textbullet\ Evaluation is done against vision and NLP tasks \cellref{D306} and RL \cellref{D308}. \newline
          \textbullet\ Wikipedia dataset is often used for evaluation. \cellref{D307} \newline
          \textbullet\ Scaling gives consistent improvements in performance. \cellref{D311}
        & \textbullet\ Evaluation metric involves assessing performance and matrix multiplication. Speedup reaches up to 36\% improvements. \cellref{G3013} \newline
          \textbullet\ CuDNN is assessed against libraries like cuda-convnet2 and Caffe. Achieves portability across GPU architectures. \cellref{G3013} \newline
          \textbullet\ Qualitative evaluation can offer valuable insights into performance. \cellref{G3051}
        & \uline{\textbf{Evaluation:}} \newline
          \textbullet\ Potential for gains through hardware-specific optimizations. \newline
          \textbullet\ Broad applicability of DNNs, while GPU programming focuses on specific architectures. \newline
          \textbullet\ Consistent scaling suggests promising future for DNNs, while GPU advances focus on specific implementations.
        \\
        \midrule

    LF1
      & \textbullet\ To address usability, many libraries provide common APIs with other frameworks. \cellref{D402} \newline
        \textbullet\ Training DNNs require special algorithms that are often architecture-specific. \cellref{D403} \newline
        \textbullet\ Optimizations are challenging and error prone, requirement intimate knowledge of the network architecture. \cellref{D411} \newline
        \textbullet\ Reinforcement learning libraries have bindings that allow both task-parallel and actor-based parallelism. \cellref{D408}
      & \textbullet\ Replication of results is challenging (can take months of work). \cellref{G4041} \newline
        \textbullet\ The requirement to manually fine-tune architectures is time-consuming and requires deep knowledge of the GPU architecture. \cellref{G4012} \newline
        \textbullet\ The memory profiles are used to assess performance. \cellref{G4012}
      & \uline{\textbf{Usability:}} \newline
      \textbullet\ To address the replication of SOTA results common APIs are used (Transformers~\cite{wolf_huggingfaces_2020} addresses this issue as it interfaces with \cite{falcon_pytorch_2019}). \newline
      \textbullet\ Being closed source, open-source frameworks cannot reliably match NVidia SOTA performance, as they have better knowledge of the GPU architecture. \newline
      \textbullet\ Profiles are key to efficient debugging and optimization in both cases.\\
      \midrule

    LF2
        & \textbullet\ No single algorithm that can perform optimally across all cases. \cellref{D406} \newline
          \textbullet\ Communication overhead leads to resource under-utilization and solutions do not transfer across architectures. \cellref{D403}, \cellref{D405} \newline
          \textbullet\ Data parallelism models are designed for homogeneous setups. \cellref{D404}
        & \textbullet\ Main issues relate to memory management around matrix multiplication algorithms. Problems also relate to hyperparameter choice, as some stride sizes perform sub-optimally. \cellref{G4013}
        & \uline{\textbf{Algorithmic Limitations:}} \newline
          \textbullet\ No algorithms perform optimally across all cases. \newline
          \textbullet\ Memory optimizations remain a challenge for both GPU programming and DNNs.
        \\
        \midrule
        
    LF3
        & \textbullet\ Tensorflow performs node placement and communication management which results in overhead. \cellref{D401} \newline
          \textbullet\ Deepspeed incurs communication overhead by allocating data to CPU memory. \cellref{D407} \newline
          \textbullet\ Some papers emphasize collaboration in the research community to ensure innovation. \cellref{D410}
        & \textbullet\ Sophisticated techniques to manage communication overhead by not updating parameters across GPUs on each layer. \cellref{G4051} \newline
          \textbullet\ The cost of transferring data to the GPU outweighs the benefits of using a GPU. \cellref{G4061} \newline
          \textbullet\ The very existence of CuDNN implies that cross-GPU programming is challenging which requires thorough understanding of the GPU architecture. \cellref{G4012}, \cellref{G4011}
        & \uline{\textbf{Communication Overhead \& Scalability:}} \newline
          \textbullet\ Tradeoffs between communication overhead and performance. \newline
          \textbullet\ Both GPUs and DNNs face similar bottlenecks in terms of memory allocation that impacts performance. \newline
          \textbullet\ There are no simple universal solutions and choosing the right approach depends on model architectures and hardware. \newline
          \textbullet\ Community is key to success for DNNs.
        \\
        \midrule


        
		\bottomrule
	\end{longtable}
}

\twocolumn



\textbf{CF3. Performance.}
Performance is a key concern for both DNNs and GPU programming. DNNs are motivated by the need to
minimize network bandwidth latency and achieve better scalability, while GPU programming provides
optimized primitives by implementing large-matrix operations. To do this, thorough
knowledge of the GPU architecture is necessary.

\textbf{CF4. Network and hardware communication.}
Multi-GPU training is particularly relevant in DNNs. There exist multiple algorithms to minimize
network latency, however GPU programming frameworks still struggle with multi-GPU communication
training, as libraries like cuDNN leave this management to the user. This indicates an ongoing
challenge in this area.

\textbf{CF5. Ease of use and hardware flexibility.}
The main challenge here resides between ease of use to the developer and hardware flexibility.
Considering the broad community of developers, DNN libraries prioritize modularity and ease of
extension to facilitate broader community involvement. Some GPU programming libraries sacrifice ease of use for lower-level control
and potentially higher performance (cuDNN), while others strive for more user-friendly APIs (CuPy, Caffe).

\paragraph{EM1. Deployment.}
DNN libraries follow a staged deployment process for safe evaluation (i.e. Google products), which
offers a degree of safety in real world scenarios. Conversely, GPU programming frameworks are
designed with configurable deployment in mind, providing features like compile-time flags to
integrate more easily into ML frameworks.

\textbf{EM2. Model architectures.}
In some evaluating scenarios, special care was payed to assess performance across different architectures.
For DNNs, this involves assessing scalability with increasingly complex model architectures. On the other hand,
for GPU programming, evaluation is geared towards optimizing performance by providing efficient primitives
for the most popular operations (convolutions, self-attention, fully connected layers, etc.).

\textbf{EM3. Task domains.}
Evaluation metrics are strongly dependent on the task domains. DNNs are assessed across
a broad range of deep learning applications, while GPU programming libraries are geared towards
a more narrow field  (like cuDNN), with others being more general purpose (like CuPy in scientific
computing).

\textbf{EM4. Evaluation}
DNN evaluation focuses on overall performance gains and broad applicability in different domains (vision,
NLP, etc.). Conversely, GPU programming focuses on the potential performance gains achievable through
hardware-specific optimizations. Both fields are continuously refined when new hardware architectures
become available and new deep learning algorithms are developed.

\textbf{LF1. Usability.}
DNN libraries attempt to improve usability through common APIs to facilitate broader
reproducibility. On the other hand, GPU programming frameworks have performance optimizations as
the main objective, which sacrifices ease of use and hides internal implementation details.
To circumvent this, many GPU frameworks provide profiling tools and specialized debuggers to aid
developer productivity.

\textbf{LF2. Algorithmic limitations.}
The primary algorithmic limitation concerns memory management in both domains.
DNNs pose problems related to data and model parallelism, while GPU programming faces issues
related to matrix multiplication algorithms and hyperparameter choices being suboptimal in some edge cases
(i.e. small batch size).

\textbf{LF3. Communication Overhead and Scalability.}
Another limitation is the communication overhead and its impact on scalability. Both areas struggle
with this, leading to performance bottlenecks that are challenging to overcome. There are not universally
optimal solutions, as the best approaches are dependent on model architectures and hardware configurations.
Community involvement is essential for progress as this can promote innovation.
