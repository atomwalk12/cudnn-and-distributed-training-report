In short, studies about distributed neural networks are quite recent and offer interesting insights
into the key factors that motivate the development of DNNs. The key factors for developing such
frameworks range from the drive to improve performance and create general purpose intelligent
systems by leveraging increasingly abundant computational resources and data
availability~\cite{chen_mxnet_2015,lepikhin_gshard_2020,shoeybi_megatron-lm_2020}.

\paragraph{C1. Key Motivating Factors.}
The motivating factors for training DNNs include the pursuit of better performance through more
efficient training and increased usability in practical scenarios. It has been widely regarded that
by scaling the architectures to a large number of parameters and leveraging larger datasets, the
evaluation accuracy on many benchmarks would be improved \cite{hestness_deep_2017}. However, it was
recently demonstrated by Deepseek R1
\cite{deepseekai2025deepseekr1incentivizingreasoningcapability} that scaling up the computational
power is not the only possible method of progress. As a result, future research is likely to focus
not only on distributed training, but also on innovating existing architectures. Nonetheless, below
is a review of the key properties

\textbf{Performance.}
For example, this was demonstrated in GShard \cite{kaplan_scaling_2020} and GPipe
\cite{huang_gpipe_2019} also shows that "by increasing the model capacity from 400M params to 1.3B,
and further to 6B, leads to significant quality improvements across all languages". In particular,
the NLP field is particularly sensitive to the model capacity, as shown in Megatron-LM
\cite{shoeybi_megatron-lm_2020}, noting that "empirical evidence indicates that larger language
models are dramatically more useful for NLP tasks such as article completion, question answering
and natural language inference". The need for larger models is not limited to text generation
tasks, as GPipe was also motivated by the desire to improve translation quality through increased
model size in low-resource languages \cite{huang_gpipe_2019}.

\textbf{Resource Utilization.}
However, the pursuit of larger models introduces challenges related to efficient training. The
BytePS paper \cite{jiang_unified_nodate} shows that existing frameworks that utilize solely the
CPU or the GPU are suboptimal. It demonstrates that utilizing all available resources leads to
significant performance improvements, where the CPU is used to perform summation of gradients,
while the parameter updates are performed on the GPU. BytePS outperforms all-reduce (traditional)
methods significantly both when CPUs are or are not available.

\textbf{Efficiency.}
The need for efficient training is also reflected in the development of systems like Tensorflow
\cite{abadi_tensorflow_2016} and Ray \cite{moritz_ray_2018}, both of which aimed to provide
general-purpose deep learning frameworks to address the needs of large communities. DeepSpeed
\cite{rasley_deepspeed_2020} was also created to facilitate training of large models with over 100
billion parameters. Deepspeed utilizes PyTorch \cite{noauthor_pytorchpytorch_nodate} as the
underlying framework in order to "make distributed training and inference easy, efficient, and
effective".

\textbf{Model and Pipeline Parallelism.}
In frameworks such as Pytorch DDP \cite{li_pytorch_2020}, the technique they used for distributed
training involved data parallelism (see \ref{sec:related_work}), which required the entire model to
fit on a single GPU device. This had the inconvenience that if a model did not fit in memory, the
approach would not be feasible. To address this issue, techniques such as pipeline parallelism
\cite{huang_gpipe_2019} and model parallelism \cite{shoeybi_megatron-lm_2020} were developed to
allow the effective training of models that are too large to fit on a single device.

\textbf{Usability.}
Another key aspect of motivation is practical usability. Frameworks like PyTorch \cite{li_pytorch_2020}
and Horovod \cite{sergeev_horovod_2018} ensured that distributed training could easily be integrated
in existing code with minimal changes. Huggingface Transformers \cite{wolf_huggingfaces_2020} was
explicitly designed to make NLP models easy to download, fine-tune and share. Colossal-AI
\cite{li_colossal-ai_2023} sought to create a unified system to simplify the complexities of
large scale distributed training. It was designed to improve on existing work such as the Alpa library
\cite{noauthor_alpa-projectsalpa_nodate}, which was discontinued in 2024.

Finally, MxNet \cite{chen_mxnet_2015} aimed to provide a uniform programming interface by bridging
the gap between imperative and declarative programming, allowing users to express computations in a
variety of styles.

\paragraph{C2. Critical Factors and Guidelines.}
This section aims to identify which factors were critical in the development of each library, and
to some degree what each library excels at.

\textbf{Review of key concepts.}
Scalability can be achieved through two inter-related ideas. First, it is possible to increase the number of parameters in the
model, yielding more powerful architectures, that tend to be more general-purpose. Second, it is also possible to increase the
amount of data being used for training, which yields better generalization capabilities and a lesser chance of
overfitting\footnote{Overfitting relates to the idea that the model learns the training data too well, and as a result,
	performs poorly on unseen examples.}.

The former approach is implemented through a technique called {\em{Model parallelism}} (and its
extension {\em{Pipeline parallelism}}) by distributing layers of the model across the GPU cluster,
where each individual GPU is responsible for computing the gradients of a portion of the model.
Pipeline parallelism is a technique that extends model parallelism by also dividing the processing
of a batch of data across the GPUs. This means that while one GPU is working on a segment of the
data batch through its assigned model layers, the other GPUs may work on different segments through
the layers that were assigned to them.

On the other hand, the later approach is implemented through {\em{Data parallelism}}. In this case,
the model is replicated (exact copy) across the GPU cluster, and each GPU is responsible for
computing the gradients of portions of the dataset. However, in this case, the model must be small
enough to fit in the memory of a single GPU.

For more information on the differences the reader is invited to refer to
\cite{dehghani_distributed_2023}.

\textbf{Scalability.}
GShard \cite{lepikhin_gshard_2020} uses model parallelism to scale up to large
datasets, innovating existing frameworks with techniques such as conditional computation and
automatic sharding. The former allows to activate only a sub-network of the larger model on
per-input basis, while the later allows to automatically partition a neural network model across
multiple GPUs without the need for human intervention.

Similarly, Colossal-AI \cite{li_colossal-ai_2023} and Megatron-LM \cite{shoeybi_megatron-lm_2020}
show methods to scale up large architecture through model parallelism, leading to models of up to
8.3 billion parameters.

Conversely, GPipe \cite{huang_gpipe_2019} introduces pipeline parallelism allowing to "split the
model into several chunks of consecutive layers and each chunk is allocated to a device" and "as a
result, this method reduces cross-node communication", leading to higher throughput. This leads to
training models of up to 83 billion parameters.

\textbf{Performance.}
The immediate thought that comes to mind when pondering over performance optimizations of a GPU cluster involves
efficient communication protocols that reduce the communication overhead or optimizations involving the
both the model or the data. Pytorch DDP \cite{li_pytorch_2020} attempts to optimize data parallelism by
using techniques like bucketing gradients and overlapping communication with computation.

Megatron-LM \cite{shoeybi_megatron-lm_2020}, aims to reduce communication and keep the GPUs busy by
duplicating computations across GPUs, technique called fused operations. GShard
\cite{lepikhin_gshard_2020} aims to give the programmer enough flexibility to easily change the
parallelisation strategy (through automatic sharding), without having to heavily modify existing
code.

\textbf{Automatic Configuration.}
This allows the library to adapt to diverse hardware setups and automatically configure the best
backend library when executing code. This is a connection between GPU programming and DNN frameworks.

For instance, Tensorflow \cite{abadi_tensorflow_2016} and MxNet \cite{chen_mxnet_2015} can
autonomously choose the most efficient algorithms considering different hardware configurations,
allowing seamless integration for mobile devices or large GPU clusters.

However, BytePS \cite{jiang_unified_nodate} attempts to improve the communication overhead by
leveraging heterogeneous GPU/CPU clusters. This is done by analyzing the network traffic for
finding the optimal way to allocate resources. Similarly, GPipe \cite{huang_gpipe_2019} can also
leverage automatic communication strategies achieving task independent parallelism in heterogeneous
environments (clusters featuring different hardware configurations).

\textbf{Ease of use.}
This category is primarily focused towards creating tools that enable practitioners and researchers to easily
use state-of-the-art models. The Huggingface Transformers \cite{wolf_huggingfaces_2020} is a prime example
that provides access to open-source models through the Huggingface hub \cite{noauthor_hugging_2025}.
Finally, Tensorflow \cite{abadi_tensorflow_2016} provides Tensorboard, which is a tool for recording
experimental data in a distributed cluster.

\paragraph{C3. Practical Evaluation Scenarios.}
The evaluation factors can be classified into the following categories:

\textbf{Scalability and Performance.}
For scaling neural networks across GPU clusters, the ideal scenario involves achieving linear speedup
with respect to the setting involving a single machines (with possibly multiple GPUs). The difficulties
that arise as a result of network communication are due to limitations involving the bandwidth of the
network making linear speedup challenging to achieve. Some libraries have shown demonstrated
effective techniques for reducing the communication overhead.

% Concepts definition checkpoint
%\textbf{Near Linear Speedup.}
GPipe \cite{huang_gpipe_2019}, GShard \cite{lepikhin_gshard_2020} and Pytorch DDP
\cite{li_pytorch_2020} achieve "near linear" speedup by leveraging pipeline parallelism and model
parallelism. For example, Pytorch DDP achieved this by aggregating gradients into buckets for
communication and overlapping communication with computation, over a cluster with 256 GPUs. GShard
took a practical approach by assessing the trade-offs between model size, training time and
resulting accuracy.

%\textbf{Heterogeneous Clusters.}
Other frameworks such as BytePS \cite{jiang_unified_nodate} excel at optimizing communication
overhead in heterogenous GPU/CPU clusters. They outperformed Parameter Server and All-Reduce
methods by scaling GPU clusters to 2048 devices and training for a time period of 4 days.

%\textbf{Efficiency Techniques.}
Effective techniques such as automatic parallelization, sharding and offloading have been proposed
by Colossal-AI \cite{li_colossal-ai_2023}, effectively training models of up to 13B parameters.

%\textbf{Reinforcement Learning.}
Other libraries such as Ray \cite{moritz_ray_2018} focused on specialized techniques for training
reinforcement learning models.

\begin{table*}[ht]

	\centering
	\caption{Concepts recurring in the papers.}
	\label{tab:concepts}
	\resizebox{\textwidth}{!}{%
		\footnotesize
		\begin{tabular}{lll} % NOTE: concepts definition checkpoint
			\hline
			\textbf{Area}    & \textbf{Concept}                           & \textbf{Definition} \\
			\hline
			\multirow{8}{*}{DNNs}
			                 & Bucket Gradient Aggregation                & \makecell[l]{...}   \\ % Pytorch DDP \cite{li_pytorch_2020}
			                 & Overlapping Communication with Computation & ...                 \\ % Pytorch DDP \cite{li_pytorch_2020}
			                 & Heterogeneous Clusters                     & ...                 \\ % BytePS \cite{jiang_unified_nodate}
			                 & Parameter Server                           & ...                 \\ % BytePS \cite{jiang_unified_nodate}
			                 & All-Reduce                                 & ...                 \\ % BytePS \cite{jiang_unified_nodate}
			                 & Automatic Parallelization                  & ...                 \\ % Colossal-AI \cite{li_colossal-ai_2023}
			                 & Offloading                                 & ...                 \\ % Colossal-AI \cite{li_colossal-ai_2023}
			                 & Sharding                                   & ...                 \\ % Colossal-AI \cite{li_colossal-ai_2023}
			\hline
			\multirow{2}{*}{GPU Programming}
			                 & Model Parallelism                          &                     \\
			                 & Model Parallelism                          &                     \\
			\hline
			Data Parallelism & Data Parallelism                           &                     \\
			\hline
		\end{tabular}
	}
\end{table*}
