% ======== CORE THEME 1: SCALABILITY ========
\section*{Scalability Relationships}
\subsection*{Motivation (MF1)}
\textbf{MF1. Scalability.} The connection is that scalability is a major shared motivating factor for both DNNs and GPU programming.
The increasing scale of data and complexity of DNNs necessitates scalable solutions. GPU programming is
motivated by providing the tools and optimizations needed to achieve this scalability, enabling DNNs to
handle larger workloads, improve productivity, and become more cost-effective.

\subsection*{Critical Factors (CF2)}
\textbf{CF2. Scalability.} Scalability is achieved by implementing a modular programming style. DNN frameworks
abstract away the distributed infrastructure complexity -- by being able to easily select distributed
strategies when executing the code -- while ML frameworks leveraging GPU acceleration to hide
low-level hardware details, allowing developers to focus on the application logic.
% Related to: MF1 (motivation), LF3 (limitation)

\subsection*{Limitations (LF3)}
\textbf{LF3. Communication Overhead and Scalability.} Another limitation is the communication overhead and its impact on scalability. Both areas struggle
with this, leading to performance bottlenecks that are challenging to overcome. There are not universally
optimal solutions, as the best approaches are dependent on model architectures and hardware configurations.
Community involvement is essential for progress as this can promote innovation.
% Emerges from: CF2 implementation challenges

% ======== CORE THEME 2: HARDWARE COMMUNICATION ======== 
\section{Hardware Communication Challenges}
\subsection*{Motivation (MF4)}
\textbf{MF4. Heterogeneous hardware.} While DNNs are motivated to use heterogenous hardware to ensure broader applicability and performance, GPU programming
acknowledge its importance by providing C APIs for CPU-GPU communication. Nonetheless, there do exist limitations
due to latency and sub-optimal bandwidth utilization in both domains. A related concern is that GPU libraries like cuDNN do not provide
integrated support for multi-GPU training, which must be achieved manually by the user. This highlights existing challenges
in both domains.

\subsection*{Critical Factors (CF4)}
\textbf{CF4. Network and hardware communication.} Multi-GPU training is particularly relevant in DNNs. There exist multiple algorithms to minimize
network latency, however GPU programming frameworks still struggle with multi-GPU communication
training, as libraries like cuDNN leave this management to the user. This indicates an ongoing
challenge in this area.
% Direct progression from MF4 motivation

% ======== CORE THEME 3: PERFORMANCE OPTIMIZATION ========
\section{Performance Considerations}
\subsection*{Motivation (MF2)}
\textbf{MF2. Complexity and performance.} The key shared motivator is to manage computational complexity while at the same time ensure higher accuracy
in common applications (i.e. NLP tasks). The increased accuracy is due to the guarantee that performance
is likely to increase thanks to the scaling laws that neural networks exhibit.
GPU programming directly provides the necessary primitives to facilitate the scaling laws and DNNs build on
top of them to improve performance.

\subsection*{Critical Factors (CF3)}
\textbf{CF3. Performance.} Performance is a key concern for both DNNs and GPU programming. DNNs are motivated by the need to
minimize network bandwidth latency and achieve better scalability, while GPU programming provides
optimized primitives by implementing large-matrix operations. To do this, thorough
knowledge of the GPU architecture is necessary.
% Implements MF2 motivations

\subsection*{Evaluation (EM2)}
\textbf{EM2. Model architectures.} In some evaluating scenarios, special care was payed to assess performance across different architectures.
For DNNs, this involves assessing scalability with increasingly complex model architectures. On the other hand,
for GPU programming, evaluation is geared towards optimizing performance by providing efficient primitives
for the most popular operations (convolutions, self-attention, fully connected layers, etc.).
% Assesses CF3 implementations

% ======== CORE THEME 4: COMMUNITY & ECOSYSTEM ========
\section{Community and Tooling Landscape}
\subsection*{Motivation (MF3, MF6, MF7)}
\textbf{MF3. Critical in many domains.}
GPU programming itself might be considered a more specialized domain, but its motivation is connected to the
critical applicability of DNNs across many fields. GPU programming enable DNNs to function efficiently in
these domains by providing the building blocks. A specific example involves Nvidia providing the optimized libraries,
as they can optimize the code better than the general community due to better knowledge of the underlying GPU architecture.

\textbf{MF6. Leveraging existing tools...}
DNN development benefits from open-source frameworks, promoting community-driven innovation. Conversely, GPU programming,
while often proprietary, builds on-top of other low-level libraries like cuBLAS. This shows a reliance on other
proprietary software, which ensures high-level performance. This can stifle innovation in the long term due
to the lack of competition.

\textbf{MF7. Cross-framework use...}
Open-source DNN framework aim for cross-framework compatibility and usability to foster community innovation.
GPU programming libraries show a trade-off between low-level, highly optimized primitives (like cuDNN), and offering
user-friendly interfaces (like CuPy and Torch7). Community involvement is important in both areas, however
less so for GPU programming due to the proprietary nature of the frameworks.

\subsection*{Critical Factors (CF1)}
\textbf{CF1. Paradigms, programming ease...}
DNNs are generally flexible and often use popular interpreted programming languages like Python to
promote ease of use. On the other hand, GPU programming usually relies on C++ and CUDA, which are
critical in areas where speed is a concern. Many GPU programming libraries provide bindings to
popular languages like Python to promote broad user involvement. This shows that community is
important in both areas, however GPU programming must rely on low-level languages to squeeze out
the best performance.
% Connects to MF7's cross-framework focus

% ======== CORE THEME 5: USABILITY & FLEXIBILITY ========
\section{Usability Tradeoffs}
\subsection*{Motivation (MF5)}
\textbf{MF5. Applications...} A shared motivation is to simplify development and improve the practical utility of both DNNs and GPU programming.
Both fields are driven by the need to make life easier for developers to leverage parallel hardware effectively,
and the motivation is to fulfill practical requirements of simplified deployment and reproducible research.

\subsection*{Critical Factors (CF5)}
\textbf{CF5. Ease of use and hardware flexibility...}
The main challenge here resides between ease of use to the developer and hardware flexibility.
Considering the broad community of developers, DNN libraries prioritize modularity and ease of
extension to facilitate broader community involvement. Some GPU programming libraries sacrifice ease of use for lower-level control
and potentially higher performance (cuDNN), while others strive for more user-friendly APIs (CuPy, Caffe).

\subsection*{Limitations (LF1)}
\textbf{LF1. Usability...}
DNN libraries attempt to improve usability through common APIs to facilitate broader
reproducibility. On the other hand, GPU programming frameworks have performance optimizations as
the main objective, which sacrifices ease of use and hides internal implementation details.
To circumvent this, many GPU frameworks provide profiling tools and specialized debuggers to aid
developer productivity.
% Result of CF5 tensions

% ======== CORE THEME 6: DEPLOYMENT & EVALUATION ========
\section{Deployment Strategies}
\subsection*{Evaluation (EM1, EM3, EM4)}
\textbf{EM1. Deployment...}
DNN libraries follow a staged deployment process for safe evaluation (i.e. Google products), which
offers a degree of safety in real world scenarios. Conversely, GPU programming frameworks are
designed with configurable deployment in mind, providing features like compile-time flags to
integrate more easily into ML frameworks.

\textbf{EM3. Task domains...}
Evaluation metrics are strongly dependent on the task domains. DNNs are assessed across
a broad range of deep learning applications, while GPU programming libraries are geared towards
a more narrow field  (like cuDNN), with others being more general purpose (like CuPy in scientific
computing).

\textbf{EM4. Evaluation...}
DNN evaluation focuses on overall performance gains and broad applicability in different domains (vision,
NLP, etc.). Conversely, GPU programming focuses on the potential performance gains achievable through
hardware-specific optimizations. Both fields are continuously refined when new hardware architectures
become available and new deep learning algorithms are developed.

% ======== CORE THEME 7: ALGORITHMIC CHALLENGES ========
\section{Algorithmic Limitations}
\subsection*{Limitations (LF2)}
\textbf{LF2. Algorithmic limitations...}
The primary algorithmic limitation concerns memory management in both domains.
DNNs pose problems related to data and model parallelism, while GPU programming faces issues
related to matrix multiplication algorithms and hyperparameter choices being suboptimal in some edge cases
(i.e. small batch size).
% Connects to CF3 (Performance)

% ======== UPDATED CROSS-THEME CONNECTIONS ========
\begin{itemize}
	\item \textbf{MF1 → CF2 → LF3}: Scalability motivation (MF1) leads to modular implementation approaches (CF2), ultimately resulting in communication limitations (LF3)
	\item \textbf{MF4 → CF4 → LF3}: Hardware heterogeneity motivation leads to multi-GPU challenges (CF4), ultimately creating scalability limitations (LF3)
	\item \textbf{MF2 → CF3 → EM2}: Performance motivation requires architectural knowledge (CF3), evaluated through model-specific metrics (EM2)
	\item \textbf{MF6 → CF1 → LF1}: Tool leverage (MF6) enables programming paradigms (CF1) but creates usability limitations (LF1)
	\item \textbf{MF5 → CF5 → EM1}: Application motivation (MF5) drives flexibility needs (CF5) evaluated through deployment (EM1)
	\item \textbf{MF3 → EM3}: Domain criticality (MF3) influences task-specific evaluation (EM3)
	\item \textbf{CF3 → LF2}: Performance requirements (CF3) reveal algorithmic limitations (LF2)
\end{itemize}

% ======== PRESERVED ORIGINAL STRUCTURE ========
\clearpage
\onecolumn
\nopagebreak

{\footnotesize
	\begin{longtable}{|l|p{5cm}|p{5cm}|p{5cm}|}
		\caption{Translations of the motivating factors}\label{tab:translations_motivating_factors}   \\

		\toprule
		\textbf{ID} & \textbf{Distributed Neural Networks} & \textbf{GPU Programming} & \textbf{Translation} \\
		\midrule
		\endfirsthead

		\multicolumn{4}{c}{Table \thetable{} -- continued from previous page}           \\
		\toprule
		\textbf{ID} & \textbf{Distributed Neural Networks} & \textbf{GPU Programming} & \textbf{Translation} \\
		\midrule
		\endhead
		\midrule
		MF1
		   & \textbullet\ Google internally requires their deep learning frameworks to be scalable. \cellref{D101} \newline
             \textbullet\ Internally, other organizations (i.e. Facebook) become more and more reliant on neural networks. \cellref{D106}
           & \textbullet\ Optimizing kernels is difficult and time-consuming. \cellref{G1011}              
           & \uline{\textbf{Scalability}}\newline 
           %Internal need for scalability
           \textbullet\ There is a surging need for scalability, likely due to the increasingly abundant data availability which is time consuming to process. \newline
             \textbullet\ The reliance on neural networks has increased productivity and reduced costs.            \\
           \midrule
		   MF2 
           & \textbullet\ The trend to scale datasets and computational resources yields increased performance in ImageNet competitions. \cellref{D102}, \cellref{D105}, \cellref{D103}
            \newline
            \textbullet\ The abundance of computation and data are particularly effective in Natural Language Processing (NLP) tasks. \cellref{D111}
           & \textbullet\ Natural parallelizability of deep learning techniques enables training higher capacity networks on larger datasets. \cellref{G1012} \newline
             \textbullet\ Early open-source GPU implementations of CNNs set precedent for code sharing. \cellref{G1051}
           & \uline{\textbf{Complexity and performance}}\newline 
           \textbullet\ Effective training parallelization leads to increased performance. 
           \newline
           \textbullet\ Larger networks consistently provide better performance, especially in NLP tasks. 
           \newline
           \textbullet\ Open-source implementations have accelerated progress. \\
           \midrule
		   MF3 
           &
             \textbullet\ The deep learning applications are critical in many domains. \cellref{D103}, \cellref{D105} \newline
             \textbullet\ Frameworks have been extended reinforcement learning \cellref{D208}

           & \textbullet\ Deep learning frameworks (Caffe and PADDLE) rely on GPU programming libraries such as cuDNN. \cellref{G1014} \newline
             \textbullet\ As architectures evolve, underlying code needs to be re-optimized. This is standardized by NVidia as they understand better how the GPU architecture works.\cellref{G1013}
           & \uline{\textbf{Critical in many domains}}\newline 
           \textbullet\ GPU programming libraries are used in a more narrow domain, however DNNs have broader applicability in areas such as reinforcement learning. 
           \newline
           \textbullet\ NVidia understands well the GPU architecture and can provide better optimizations in critical areas.\\

           \midrule
		   MF4 
           & \textbullet\ Data centers are inherently homogenous. BytePS can leverage spare CPU and bandwidth resources to accelerate distributed training running on GPUs. \cellref{D104} \newline
             \textbullet\ Modern systems can leverage mobile devices, tablets, and thousands of GPU cards. \cellref{D201}
           & \textbullet\ GPU programming libraries expose a C language API to communicate with the host CPU. \cellref{G1015}
           & \uline{\textbf{Heterogenous hardware}}\newline 
           \textbullet\ There is limited support for CPU-GPU interaction in GPU programming libraries. 
           \newline
           \textbullet\ Heterogeneous hardware plays a more important role in DNNs as the ability to fully utilize available resources is critical. \\
 
           \midrule
		   MF5
           & DDNs have powered a wide range of applications including image recognition, language translation, anomaly detection, and more. \cellref{D106} \newline
             \textbullet\ Replication of published results can involve months of work by researchers. \cellref{G1041}
           & \textbullet\ GPU libraries meet user's needs by reducing the need to write custom code, allowing developers to focus on higher-level issues, improved portability. \cellref{G1016} \newline
             \textbullet\ Few toolboxes offer truly off-the-shelf deployment of state-of-the-art models that are computationally efficient. \cellref{G1041}
           & \uline{\textbf{Requirements and applications}}\newline 
           \textbullet\ Both topics aim to make it easier for developers to take advantage of parallel hardware. 
           \newline
           \textbullet\ GPU programming facilitates the development of new architectures and DNNs scale models to achieve better accuracy. 
           \newline
           \textbullet\ The need for efficient deployment and replication of research results drives development in both areas. \\

           \midrule
		   MF6
           & \textbullet\ Colossal-AI builds on existing open-source frameworks such as PipeDream, GPipe and Chimera. \cellref{D107}, \cellref{D207}
           & \textbullet\ CuDNN relies on the CUDA toolkit, specifically the cuBLAS library. \cellref{G1016} \newline \textbullet\ CuPy is specifically designed to work with NVidia GPUs. \cellref{G1062}
           & \uline{\textbf{Leverage existing frameworks}}\newline 
           \textbullet\ Since DNN frameworks are generally open-source, this encourages community involvement, which leads to innovation. \newline 
           \textbullet\ GPU programming libraries are proprietary. Nonetheless, internally libraries such as cuDNN rely on the CUDA toolkit. \\

           \midrule
		   MF7
           & \textbullet\ Inter-GPU communication frameworks require minimal code changes and support multiple frontends. \cellref{D110}, \cellref{D112} \newline
            \textbullet\ Libraries can generally be integrated into existing frontend frameworks (i.e. PyTorch). \cellref{D211}
           & \textbullet\ GPU programming libraries provide lower-level primitives and are generally self-contained. \cellref{G1017} \newline
             \textbullet\ Libraries emphasize compatibility (CuPy with NumPy) or ease of development (Torch7). \cellref{G1062}, \cellref{G1071} \newline
             \textbullet\ Research-focused libraries prioritize configurability and flexibility. \cellref{G1031}
           & \uline{\textbf{Cross-framework use}}\newline 
           \textbullet\ Open-source DNN libraries promote usability and cross-framework compatibility, fostering innovation. 
           \newline
           \textbullet\ GPU libraries vary between low-level primitives (cuDNN) and user-friendly interfaces (CuPy, Torch7), though being closed-source limits community-driven innovation. 
           \newline
           \textbullet\ This highlights the importance of code sharing for research. \\

           % recent advancements in deep learning
           % rapidly evolving field

        

		\bottomrule
	\end{longtable}
}

\twocolumn

% DONE D101 internal need for scalability
% DONE D102 increasingly complex datasets (Done also from D105 and D111), Improved performance
% DONE D103 critical in many domains (emerging applications)
% DOING D104 utilization of heterogeneous hardware
% D105
% howpublished = \{(https?://.*?)\}
% howpublished = {\url{$1}}
% url = \{(https?://.*?)\}
% howpublished = {\url{$1}}
% DONE D201 utilization of heterogenous hardware


% Not taken into account:
% D108
\clearpage
\onecolumn

{\footnotesize
	\begin{longtable}{|l|p{5cm}|p{5cm}|p{5cm}|}
		\caption{Translations of the critical factors}\label{tab:translations_critical}   \\

		\toprule
		ID & Distributed Neural Networks & GPU Programming & Translation \\
		\midrule
		\endfirsthead

		\multicolumn{4}{c}{Table \thetable{} -- continued from previous page}           \\
		\toprule
		ID & Distributed Neural Networks & GPU Programming & Translation \\
		\midrule
		\endhead
		\midrule
		CF1
		   & Generally DNNs are most flexible and use the most common programming style supported by the host language. \cellref{D202}, \cellref{D205}
           & \textbullet\ Although most GPU frameworks work using the host language as C++, there exist frontend frameworks that enable users to use Python. \cellref{G2021} \newline
             \textbullet\ Cudnn exposes a C API \cellref{G1015}
           & \textbf{Programming paradigms, programming ease.} Imperative and declarative programming styles are both supported. DNNs use Python as the most popular host language, while GPU frameworks generally work with C++ and CUDA. 
           Nonetheless, there do exist frontend frameworks that enable users to write code in higher-level languages, such as Python.\\
           \midrule

    CF2
    & \textbullet\ Distributing neural network layers across multiple GPUs is architecture-specific. \cellref{D203} \newline
      \textbullet\ Scaling is expensive in terms of cost, time and code integration. \cellref{D209} \newline
      \textbullet\ There is a separation of concerns between GPU programming and DNNs, as there is no need for custom C++ code or compiler required to distribute neural networks over cluster nodes. \cellref{D211}
        & \textbullet\ Optimized code using NVIDIA GPUs ensures high performance (freeing up auxiliary memory) \cellref{G2011} \newline
          \textbullet\ CuDNN provides separation of concerns by enabling developers to focus on higher-level optimizations instead of low-level architecture specific code. \cellref{G2012}
        & \textbf{Scalability, cost, usability.} DNNs: Scalability challenges are related to distributing parts of the network or dataset across multiple nodes. Since the CUDA toolkit provides a separation of concerns, developers generally
        do not write low-level code for inter-GPU communication. They can focus on higher-level optimizations (i.e. bucket aggregation, overlapping communication with computation, etc).
        \\
        \midrule

    CF3
    & \textbullet\ Specialized techniques for distributed training include: bucketing, overlapping communication with computation, etc. \cellref{D206} \newline
      \textbullet\ Megatron-LM extentrds optimization techniques to the transformer model. \cellref{D211}
        & It is a challenge to provide consistent performance as new architectures emerge. \cellref{G2013}
        & \textbf{!!!Performance.} Although, a lot has been discovered, there exist future challenges in both domains as new architectures are developed.  \newline
          \\
        \midrule
    
    CF4
    & There exist algorithms that can optimize network latency. \cellref{D210}
        & Multi-GPU training is an outstanding challenge. \cellref{G2014}
        & \textbf{Network latency.} Some challenging problems have optimal algorithms for distributed training. Conversely, multi-GPU training is outstanding work (concurrency is handled by the user). \\
        \midrule

    CF5
    & The Transformers library provides modular components that greatly simplify the extension and ease of use of the library \cellref{D212}
        & \textbullet\ CuPy is NumPy compatible. \cellref{G1022} \newline
          \textbullet\ CuDNN requires more specialized C and CUDA knowledge. \cellref{G1015}
        & \textbf{Ease of use.} \textbullet\ Given the rich ecosystem, there exist DNN libraries that are particularly easy to build on and extend (i.e. Pytorch). \newline
          \textbullet\ In general GPU programming libraries are more challenging to get started with as they require more specialized knowledge. However, CuPy attempts to bridge the gap by being NumPy compatible.\\
        \midrule

		\bottomrule
	\end{longtable}
}

\twocolumn


\clearpage
\onecolumn

{\footnotesize
	\begin{longtable}{|l|p{5cm}|p{5cm}|p{5cm}|}
		\caption{Translations of the evaluation metrics}\label{tab:translations_evaluation_metrics}   \\

		\toprule
		ID & Distributed Neural Networks & GPU Programming & Translation \\
		\midrule
		\endfirsthead

		\multicolumn{4}{c}{Table \thetable{} -- continued from previous page}           \\
		\toprule
		ID & Distributed Neural Networks & GPU Programming & Translation \\
		\midrule
		\endhead
		\midrule
    EM1
        & \textbullet\ Evaluation was initially performed behind closed doors for internal processes (speech recognition systems) and subsequently for external applications (Google Search). \cellref{D301}
        & \textbullet\ In many frameworks, the GPU libraries can be switched on and off at compile time using a single flag. \cellref{G1014}
        & \uline{\textbf{Deployment:}} \newline
          \textbullet\ Large companies can assess new technologies in production without letting the user know about the deployment of new techniques. Allows to rollback to the previous version if the new technique is not effective.
          \textbullet\ GPU libraries are transparent, but assessing the effectiveness in the real world is not as straightforward as for other companies (i.e. Google).
        \\
        \midrule

    EM2
        & \textbullet\ The evaluation was done by scaling complex networks -- based on Mixture of Experts -- to 600B parameters using automatic sharding. \cellref{D305}
        & \textbullet\ The cuDNN library is assessed by measuring time and memory usage for convolutional layers. \newline
          \textbullet\ Mini-batch performance is assessed. \newline
          \textbullet\ Scalability is not so much a concern as different GPU architectures are benchmarked instead of GPU clusters.
        & \uline{\textbf{Model Architectures:}} \newline
          \textbullet\ ...
        \\
        \midrule

    EM3
        & 
        & 
        & \uline{\textbf{Task Doamins:}} \newline
          \textbullet\ ...
        \\
        \midrule

    EM4
        & 
        & 
        & \uline{\textbf{Evaluation:}} \newline
          \textbullet\ ...
        \\
        \midrule

    EM5
        & 
        & 
        & \uline{\textbf{Infrastructure:}} \newline
          \textbullet\ ...
        \\
        \midrule
		\bottomrule
	\end{longtable}
}

\twocolumn






% ======== ORIGINAL STRUCTURE ========
\subsection{M.6 -- Relationship between concepts}

\textbf{MF1. Scalability.}
The connection is that scalability is a major shared motivating factor for both DNNs and GPU programming.
The increasing scale of data and complexity of DNNs necessitates scalable solutions. GPU programming is
motivated by providing the tools and optimizations needed to achieve this scalability, enabling DNNs to
handle larger workloads, improve productivity, and become more cost-effective.

\textbf{MF2. Complexity and performance.}
The key shared motivator is to manage computational complexity while at the same time ensure higher accuracy
in common applications (i.e. NLP tasks). The increased accuracy is due to the guarantee that performance
is likely to increase thanks to the scaling laws that neural networks exhibit.
GPU programming directly provides the necessary primitives to facilitate the scaling laws and DNNs build on 
top of them to improve performance.

\textbf{MF3. Critical in many domains.}
GPU programming itself might be considered a more specialized domain, but its motivation is connected to the
critical applicability of DNNs across many fields. GPU programming enable DNNs to function efficiently in
these domains by providing the building blocks. A specific example involves Nvidia providing the optimized libraries,
as they can optimize the code better than the general community due to better knowledge of the underlying GPU architecture.

\textbf{MF4. Heterogeneous hardware.}
While DNNs are motivated to use heterogenous hardware to ensure broader applicability and performance, GPU programming
acknowledge its importance by providing C APIs for CPU-GPU communication. Nonetheless, there do exist limitations
due to latency and sub-optimal bandwidth utilization in both domains. A related concern is that GPU libraries like cuDNN do not provide
integrated support for multi-GPU training, which must be achieved manually by the user. This highlights existing challenges
in both domains.

\textbf{MF5. Applications.}
A shared motivation is to simplify development and improve the practical utility of both DNNs and GPU programming.
Both fields are driven by the need to make life easier for developers to leverage parallel hardware effectively,
and the motivation is to fulfill practical requirements of simplified deployment and reproducible research.

\textbf{MF6. Leveraging existing tools.}
DNN development benefits from open-source frameworks, promoting community-driven innovation. Conversely, GPU programming,
while often proprietary, builds on-top of other low-level libraries like cuBLAS. This shows a reliance on other
proprietary software, which ensures high-level performance. This can stifle innovation in the long term due
to the lack of competition.

\textbf{MF7. Cross-framework use.}
Open-source DNN framework aim for cross-framework compatibility and usability to foster community innovation.
GPU programming libraries show a trade-off between low-level, highly optimized primitives (like cuDNN), and offering
user-friendly interfaces (like CuPy and Torch7). Community involvement is important in both areas, however
less so for GPU programming due to the proprietary nature of the frameworks.

\paragraph{CF1. Paradigms, programming ease.}
DNNs are generally flexible and often use popular interpreted programming languages like Python to
promote ease of use. On the other hand, GPU programming usually relies on C++ and CUDA, which are
critical in areas where speed is a concern. Many GPU programming libraries provide bindings to
popular languages like Python to promote broad user involvement. This shows that community is
important in both areas, however GPU programming must rely on low-level languages to squeeze out
the best performance.

\textbf{CF2. Scalability.}
Scalability is achieved by implementing a modular programming style. DNN frameworks
abstract away the distributed infrastructure complexity -- by being able to easily select distributed
strategies when executing the code -- while ML frameworks leveraging GPU acceleration to hide
low-level hardware details, allowing developers to focus on the application logic.

\clearpage
\onecolumn
\nopagebreak

{\footnotesize
	\begin{longtable}{|l|p{5cm}|p{5cm}|p{5cm}|}
		\caption{Translations of the motivating factors}\label{tab:translations_motivating_factors}   \\

		\toprule
		\textbf{ID} & \textbf{Distributed Neural Networks} & \textbf{GPU Programming} & \textbf{Translation} \\
		\midrule
		\endfirsthead

		\multicolumn{4}{c}{Table \thetable{} -- continued from previous page}           \\
		\toprule
		\textbf{ID} & \textbf{Distributed Neural Networks} & \textbf{GPU Programming} & \textbf{Translation} \\
		\midrule
		\endhead
		\midrule
		MF1
		   & \textbullet\ Google internally requires their deep learning frameworks to be scalable. \cellref{D101} \newline
             \textbullet\ Internally, other organizations (i.e. Facebook) become more and more reliant on neural networks. \cellref{D106}
           & \textbullet\ Optimizing kernels is difficult and time-consuming. \cellref{G1011}              
           & \uline{\textbf{Scalability}}\newline 
           %Internal need for scalability
           \textbullet\ There is a surging need for scalability, likely due to the increasingly abundant data availability which is time consuming to process. \newline
             \textbullet\ The reliance on neural networks has increased productivity and reduced costs.            \\
           \midrule
		   MF2 
           & \textbullet\ The trend to scale datasets and computational resources yields increased performance in ImageNet competitions. \cellref{D102}, \cellref{D105}, \cellref{D103}
            \newline
            \textbullet\ The abundance of computation and data are particularly effective in Natural Language Processing (NLP) tasks. \cellref{D111}
           & \textbullet\ Natural parallelizability of deep learning techniques enables training higher capacity networks on larger datasets. \cellref{G1012} \newline
             \textbullet\ Early open-source GPU implementations of CNNs set precedent for code sharing. \cellref{G1051}
           & \uline{\textbf{Complexity and performance}}\newline 
           \textbullet\ Effective training parallelization leads to increased performance. 
           \newline
           \textbullet\ Larger networks consistently provide better performance, especially in NLP tasks. 
           \newline
           \textbullet\ Open-source implementations have accelerated progress. \\
           \midrule
		   MF3 
           &
             \textbullet\ The deep learning applications are critical in many domains. \cellref{D103}, \cellref{D105} \newline
             \textbullet\ Frameworks have been extended reinforcement learning \cellref{D208}

           & \textbullet\ Deep learning frameworks (Caffe and PADDLE) rely on GPU programming libraries such as cuDNN. \cellref{G1014} \newline
             \textbullet\ As architectures evolve, underlying code needs to be re-optimized. This is standardized by NVidia as they understand better how the GPU architecture works.\cellref{G1013}
           & \uline{\textbf{Critical in many domains}}\newline 
           \textbullet\ GPU programming libraries are used in a more narrow domain, however DNNs have broader applicability in areas such as reinforcement learning. 
           \newline
           \textbullet\ NVidia understands well the GPU architecture and can provide better optimizations in critical areas.\\

           \midrule
		   MF4 
           & \textbullet\ Data centers are inherently homogenous. BytePS can leverage spare CPU and bandwidth resources to accelerate distributed training running on GPUs. \cellref{D104} \newline
             \textbullet\ Modern systems can leverage mobile devices, tablets, and thousands of GPU cards. \cellref{D201}
           & \textbullet\ GPU programming libraries expose a C language API to communicate with the host CPU. \cellref{G1015}
           & \uline{\textbf{Heterogenous hardware}}\newline 
           \textbullet\ There is limited support for CPU-GPU interaction in GPU programming libraries. 
           \newline
           \textbullet\ Heterogeneous hardware plays a more important role in DNNs as the ability to fully utilize available resources is critical. \\
 
           \midrule
		   MF5
           & DDNs have powered a wide range of applications including image recognition, language translation, anomaly detection, and more. \cellref{D106} \newline
             \textbullet\ Replication of published results can involve months of work by researchers. \cellref{G1041}
           & \textbullet\ GPU libraries meet user's needs by reducing the need to write custom code, allowing developers to focus on higher-level issues, improved portability. \cellref{G1016} \newline
             \textbullet\ Few toolboxes offer truly off-the-shelf deployment of state-of-the-art models that are computationally efficient. \cellref{G1041}
           & \uline{\textbf{Requirements and applications}}\newline 
           \textbullet\ Both topics aim to make it easier for developers to take advantage of parallel hardware. 
           \newline
           \textbullet\ GPU programming facilitates the development of new architectures and DNNs scale models to achieve better accuracy. 
           \newline
           \textbullet\ The need for efficient deployment and replication of research results drives development in both areas. \\

           \midrule
		   MF6
           & \textbullet\ Colossal-AI builds on existing open-source frameworks such as PipeDream, GPipe and Chimera. \cellref{D107}, \cellref{D207}
           & \textbullet\ CuDNN relies on the CUDA toolkit, specifically the cuBLAS library. \cellref{G1016} \newline \textbullet\ CuPy is specifically designed to work with NVidia GPUs. \cellref{G1062}
           & \uline{\textbf{Leverage existing frameworks}}\newline 
           \textbullet\ Since DNN frameworks are generally open-source, this encourages community involvement, which leads to innovation. \newline 
           \textbullet\ GPU programming libraries are proprietary. Nonetheless, internally libraries such as cuDNN rely on the CUDA toolkit. \\

           \midrule
		   MF7
           & \textbullet\ Inter-GPU communication frameworks require minimal code changes and support multiple frontends. \cellref{D110}, \cellref{D112} \newline
            \textbullet\ Libraries can generally be integrated into existing frontend frameworks (i.e. PyTorch). \cellref{D211}
           & \textbullet\ GPU programming libraries provide lower-level primitives and are generally self-contained. \cellref{G1017} \newline
             \textbullet\ Libraries emphasize compatibility (CuPy with NumPy) or ease of development (Torch7). \cellref{G1062}, \cellref{G1071} \newline
             \textbullet\ Research-focused libraries prioritize configurability and flexibility. \cellref{G1031}
           & \uline{\textbf{Cross-framework use}}\newline 
           \textbullet\ Open-source DNN libraries promote usability and cross-framework compatibility, fostering innovation. 
           \newline
           \textbullet\ GPU libraries vary between low-level primitives (cuDNN) and user-friendly interfaces (CuPy, Torch7), though being closed-source limits community-driven innovation. 
           \newline
           \textbullet\ This highlights the importance of code sharing for research. \\

           % recent advancements in deep learning
           % rapidly evolving field

        

		\bottomrule
	\end{longtable}
}

\twocolumn

% DONE D101 internal need for scalability
% DONE D102 increasingly complex datasets (Done also from D105 and D111), Improved performance
% DONE D103 critical in many domains (emerging applications)
% DOING D104 utilization of heterogeneous hardware
% D105
% howpublished = \{(https?://.*?)\}
% howpublished = {\url{$1}}
% url = \{(https?://.*?)\}
% howpublished = {\url{$1}}
% DONE D201 utilization of heterogenous hardware


% Not taken into account:
% D108
\clearpage
\onecolumn

{\footnotesize
	\begin{longtable}{|l|p{5cm}|p{5cm}|p{5cm}|}
		\caption{Translations of the critical factors}\label{tab:translations_critical}   \\

		\toprule
		ID & Distributed Neural Networks & GPU Programming & Translation \\
		\midrule
		\endfirsthead

		\multicolumn{4}{c}{Table \thetable{} -- continued from previous page}           \\
		\toprule
		ID & Distributed Neural Networks & GPU Programming & Translation \\
		\midrule
		\endhead
		\midrule
		CF1
		   & Generally DNNs are most flexible and use the most common programming style supported by the host language. \cellref{D202}, \cellref{D205}
           & \textbullet\ Although most GPU frameworks work using the host language as C++, there exist frontend frameworks that enable users to use Python. \cellref{G2021} \newline
             \textbullet\ Cudnn exposes a C API \cellref{G1015}
           & \textbf{Programming paradigms, programming ease.} Imperative and declarative programming styles are both supported. DNNs use Python as the most popular host language, while GPU frameworks generally work with C++ and CUDA. 
           Nonetheless, there do exist frontend frameworks that enable users to write code in higher-level languages, such as Python.\\
           \midrule

    CF2
    & \textbullet\ Distributing neural network layers across multiple GPUs is architecture-specific. \cellref{D203} \newline
      \textbullet\ Scaling is expensive in terms of cost, time and code integration. \cellref{D209} \newline
      \textbullet\ There is a separation of concerns between GPU programming and DNNs, as there is no need for custom C++ code or compiler required to distribute neural networks over cluster nodes. \cellref{D211}
        & \textbullet\ Optimized code using NVIDIA GPUs ensures high performance (freeing up auxiliary memory) \cellref{G2011} \newline
          \textbullet\ CuDNN provides separation of concerns by enabling developers to focus on higher-level optimizations instead of low-level architecture specific code. \cellref{G2012}
        & \textbf{Scalability, cost, usability.} DNNs: Scalability challenges are related to distributing parts of the network or dataset across multiple nodes. Since the CUDA toolkit provides a separation of concerns, developers generally
        do not write low-level code for inter-GPU communication. They can focus on higher-level optimizations (i.e. bucket aggregation, overlapping communication with computation, etc).
        \\
        \midrule

    CF3
    & \textbullet\ Specialized techniques for distributed training include: bucketing, overlapping communication with computation, etc. \cellref{D206} \newline
      \textbullet\ Megatron-LM extentrds optimization techniques to the transformer model. \cellref{D211}
        & It is a challenge to provide consistent performance as new architectures emerge. \cellref{G2013}
        & \textbf{!!!Performance.} Although, a lot has been discovered, there exist future challenges in both domains as new architectures are developed.  \newline
          \\
        \midrule
    
    CF4
    & There exist algorithms that can optimize network latency. \cellref{D210}
        & Multi-GPU training is an outstanding challenge. \cellref{G2014}
        & \textbf{Network latency.} Some challenging problems have optimal algorithms for distributed training. Conversely, multi-GPU training is outstanding work (concurrency is handled by the user). \\
        \midrule

    CF5
    & The Transformers library provides modular components that greatly simplify the extension and ease of use of the library \cellref{D212}
        & \textbullet\ CuPy is NumPy compatible. \cellref{G1022} \newline
          \textbullet\ CuDNN requires more specialized C and CUDA knowledge. \cellref{G1015}
        & \textbf{Ease of use.} \textbullet\ Given the rich ecosystem, there exist DNN libraries that are particularly easy to build on and extend (i.e. Pytorch). \newline
          \textbullet\ In general GPU programming libraries are more challenging to get started with as they require more specialized knowledge. However, CuPy attempts to bridge the gap by being NumPy compatible.\\
        \midrule

		\bottomrule
	\end{longtable}
}

\twocolumn


\clearpage
\onecolumn

{\footnotesize
	\begin{longtable}{|l|p{5cm}|p{5cm}|p{5cm}|}
		\caption{Translations of the evaluation metrics}\label{tab:translations_evaluation_metrics}   \\

		\toprule
		ID & Distributed Neural Networks & GPU Programming & Translation \\
		\midrule
		\endfirsthead

		\multicolumn{4}{c}{Table \thetable{} -- continued from previous page}           \\
		\toprule
		ID & Distributed Neural Networks & GPU Programming & Translation \\
		\midrule
		\endhead
		\midrule
    EM1
        & \textbullet\ Evaluation was initially performed behind closed doors for internal processes (speech recognition systems) and subsequently for external applications (Google Search). \cellref{D301}
        & \textbullet\ In many frameworks, the GPU libraries can be switched on and off at compile time using a single flag. \cellref{G1014}
        & \uline{\textbf{Deployment:}} \newline
          \textbullet\ Large companies can assess new technologies in production without letting the user know about the deployment of new techniques. Allows to rollback to the previous version if the new technique is not effective.
          \textbullet\ GPU libraries are transparent, but assessing the effectiveness in the real world is not as straightforward as for other companies (i.e. Google).
        \\
        \midrule

    EM2
        & \textbullet\ The evaluation was done by scaling complex networks -- based on Mixture of Experts -- to 600B parameters using automatic sharding. \cellref{D305}
        & \textbullet\ The cuDNN library is assessed by measuring time and memory usage for convolutional layers. \newline
          \textbullet\ Mini-batch performance is assessed. \newline
          \textbullet\ Scalability is not so much a concern as different GPU architectures are benchmarked instead of GPU clusters.
        & \uline{\textbf{Model Architectures:}} \newline
          \textbullet\ ...
        \\
        \midrule

    EM3
        & 
        & 
        & \uline{\textbf{Task Doamins:}} \newline
          \textbullet\ ...
        \\
        \midrule

    EM4
        & 
        & 
        & \uline{\textbf{Evaluation:}} \newline
          \textbullet\ ...
        \\
        \midrule

    EM5
        & 
        & 
        & \uline{\textbf{Infrastructure:}} \newline
          \textbullet\ ...
        \\
        \midrule
		\bottomrule
	\end{longtable}
}

\twocolumn



\textbf{CF3. Performance.}
Performance is a key concern for both DNNs and GPU programming. DNNs are motivated by the need to
minimize network bandwidth latency and achieve better scalability, while GPU programming provides
optimized primitives by implementing large-matrix operations. To do this, thorough
knowledge of the GPU architecture is necessary.

\textbf{CF4. Network and hardware communication.}
Multi-GPU training is particularly relevant in DNNs. There exist multiple algorithms to minimize
network latency, however GPU programming frameworks still struggle with multi-GPU communication
training, as libraries like cuDNN leave this management to the user. This indicates an ongoing
challenge in this area.

\textbf{CF5. Ease of use and hardware flexibility.}
The main challenge here resides between ease of use to the developer and hardware flexibility.
Considering the broad community of developers, DNN libraries prioritize modularity and ease of
extension to facilitate broader community involvement. Some GPU programming libraries sacrifice ease of use for lower-level control
and potentially higher performance (cuDNN), while others strive for more user-friendly APIs (CuPy, Caffe).

\paragraph{EM1. Deployment.}
DNN libraries follow a staged deployment process for safe evaluation (i.e. Google products), which
offers a degree of safety in real world scenarios. Conversely, GPU programming frameworks are
designed with configurable deployment in mind, providing features like compile-time flags to
integrate more easily into ML frameworks.

\textbf{EM2. Model architectures.}
In some evaluating scenarios, special care was payed to assess performance across different architectures.
For DNNs, this involves assessing scalability with increasingly complex model architectures. On the other hand,
for GPU programming, evaluation is geared towards optimizing performance by providing efficient primitives
for the most popular operations (convolutions, self-attention, fully connected layers, etc.).

\textbf{EM3. Task domains.}
Evaluation metrics are strongly dependent on the task domains. DNNs are assessed across
a broad range of deep learning applications, while GPU programming libraries are geared towards
a more narrow field  (like cuDNN), with others being more general purpose (like CuPy in scientific
computing).

\textbf{EM4. Evaluation}
DNN evaluation focuses on overall performance gains and broad applicability in different domains (vision,
NLP, etc.). Conversely, GPU programming focuses on the potential performance gains achievable through
hardware-specific optimizations. Both fields are continuously refined when new hardware architectures
become available and new deep learning algorithms are developed.

\textbf{LF1. Usability.}
DNN libraries attempt to improve usability through common APIs to facilitate broader
reproducibility. On the other hand, GPU programming frameworks have performance optimizations as
the main objective, which sacrifices ease of use and hides internal implementation details.
To circumvent this, many GPU frameworks provide profiling tools and specialized debuggers to aid
developer productivity.

\textbf{LF2. Algorithmic limitations.}
The primary algorithmic limitation concerns memory management in both domains.
DNNs pose problems related to data and model parallelism, while GPU programming faces issues
related to matrix multiplication algorithms and hyperparameter choices being suboptimal in some edge cases
(i.e. small batch size).

\textbf{LF3. Communication Overhead and Scalability.}
Another limitation is the communication overhead and its impact on scalability. Both areas struggle
with this, leading to performance bottlenecks that are challenging to overcome. There are not universally
optimal solutions, as the best approaches are dependent on model architectures and hardware configurations.
Community involvement is essential for progress as this can promote innovation.
